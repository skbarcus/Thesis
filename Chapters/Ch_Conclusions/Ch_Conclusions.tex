% Introduction
\chapter{Conclusions} % Main chapter title
\label{ch:conclusions} % For referencing the chapter elsewhere, use 

A new $^3$He elastic cross section was extracted from JLab Hall A experiment E08-014. This required isolating the elastically scattered electrons from a large quasielastic background. This analysis found an elastic electron yield of 627 electrons. The small number of elastic events contributed strongly to the uncertainty in this new cross section. The elastic electron yield was simulated using the Monte Carlo SIMC which contained an older model of elastic $^3$He from ~\cite{Article:Amroun}. The elastic SIMC results were then summed with a fit of the quasieastic background of the experimental data so as to be comparable to the full experimental yield (elastic and quasielastic events). The SIMC yield was then scaled to match the experimental yield so that SIMC would produce the experimental cross section. Bin centering corrections were then applied to find the Q$^2$ at which to place the new cross section, and finally the various sources of uncertainty were compiled. This resulted in a new $^3$He elastic cross section of 1.335 * 10$^{-6}$ $\mu$b/sr $\pm$ 0.086 at a Q$^2$ value of 34.19 fm$^{-2}$. 

The world data for both $^3$H and $^3$He elastic cross sections were then collected with the addition of new JLab high Q$^2$ data and this analysis' cross section. These cross sections were then fit with a sum of Gaussians parametrization which allowed the charge and magnetic form factors to be extracted. Charge densities were then calculated along with charge radii for both targets. The charge and magnetic form factors for $^3$H remained fairly consistent with past fits from ~\cite{Article:Amroun} since no new data has yet to be added to the world data. The charge radius for $^3$H was found to be much larger than past measurements, but this is easily attributed to not forcing the $\sum$Q$_{i_{ch}}$ to sum to unity (i.e. requiring F$_{ch}(0)$ = 1 or that the sum of the electric charges is one). Not restricting the free parameters in this manner caused the magnitude of the negative slope of F$_{ch}$ at Q$^2$ of zero to increase causing a larger radius to be found.

The charge form factor for $^3$He also remained unchanged with past fits since there is an abundance of excellent data influencing F$_{ch}$. The magnetic form factor for $^3$He did change significantly with the addition of new high Q$^2$ data. The first diffractive minimum in F$_m$ shifted higher in Q$^2$ by several fm$^{-2}$, and the magnitude of F$_m$ decreased somewhat after the first minimum. The charge radius for $^3$He was found to be in decent agreement with past measurements. Overall, our knowledge of the form factors declines considerably at higher Q$^2$ ($<$ $\approx$ 25-30 fm$^{-2}$). Collecting more high Q$^2$ and back angle data would help improve our understanding of the form factors, especially F$_m$. JLab's Hall A is quite well equipped to make these measurements with its maximum back angle of 150$^{\circ}$ and maximum beam energy of 12 GeV.

These new fits were compared to past fits of the data from Amroun \textit{et al}. ~\cite{Article:Amroun} and theory predictions from Marcucci \textit{et al}. ~\cite{Article:Marcucci}. The new fits broadly agreed with the past Amroun fits except for the shift in the $^3$He magnetic form factor discussed above. Theory is doing a good job predicting the charge form factors with a conventional approach accounting for two and three-body nucleon interactions with relativistic corrections as are $\chi$EFT predictions. However, theory struggled to predict the magnetic form factors of either $^3$H or $^3$He. 

