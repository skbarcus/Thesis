% Introduction
\chapter{Conclusions} % Main chapter title
\label{ch:conclusions} % For referencing the chapter elsewhere, use 

A new $^3$He elastic cross section was extracted from JLab Hall A experiment E08-014. This required isolating the elastically scattered electrons from a large quasielastic background. This analysis found an elastic electron yield of 627 electrons. The small number of elastic events contributed strongly to the uncertainty in this new cross section. The elastic electron yield was simulated using the Monte Carlo SIMC which contained an older model of elastic $^3$He form factors from \cite{Article:Amroun}. The elastic SIMC results were then summed with a fit of the quasieastic background of the experimental data so as to be comparable to the full experimental yield (elastic and quasielastic events). The SIMC yield was then scaled to match the experimental yield so that SIMC would produce the experimental cross section. Bin centering corrections were then applied to find the $Q^2$ at which to place the new cross section, and finally the various sources of uncertainty were compiled. This resulted in a new $^3$He elastic cross section measurement of 1.335 $\times$ 10$^{-6}$ $\mu$b/sr $\pm$ 0.086 $\times$ 10$^{-6}$ $\mu$b/sr at a $Q^2$ value of 34.19 fm$^{-2}$. 

The world data for both $^3$H and $^3$He elastic cross sections were then collected with the addition of new JLab high $Q^2$ data and this analysis' cross section. These cross sections were then fit with a sum of Gaussians parametrization which allowed the charge and magnetic form factors to be extracted. Representative charge and magnetic form factor parametrizations were chosen from the center of each set of new SOG form factor fits for $^3$H and $^3$He. Error bands were then defined about these representative fits spanning the width of all possible `good' SOG fits. Figure \ref{fig:3h_rep_fit} shows the new representative form factor fits for $^3$H and the fit parameters are given in Table \ref{tab:3h_rep_fit_pars}. Figure \ref{fig:3he_rep_fit} shows the new representative form factor fits for $^3$He and the fit parameters are given in Table \ref{tab:3he_rep_fit_pars}.

The charge and magnetic form factors for $^3$H remained consistent with past fits from \cite{Article:Amroun} since no new data has been added to the world data yet. The charge form factor for $^3$He also remained unchanged with past fits since there is an abundance of excellent data influencing $F_{ch}$. The magnetic form factor for $^3$He did change significantly with the addition of new high $Q^2$ data. The first diffractive minimum in $F_m$ shifted higher in $Q^2$ by several fm$^{-2}$, and the magnitude of $F_m$ decreased somewhat after the first minimum. Overall, our knowledge of the form factors declines considerably at higher $Q^2$ ($<$ $\approx$ 25-30 fm$^{-2}$). 

The new fits were compared to past fits of the data from Amroun \textit{et al}. \cite{Article:Amroun} and theory predictions from Marcucci \textit{et al}. \cite{Article:Marcucci}. The $^3$H charge form factor fits and theory predictions can be found in Figure \ref{fig:3h_fch_theory}, and the $^3$H magnetic form factor fits and theory predictions can be found in Figure \ref{fig:3h_fm_theory}. The $^3$He charge form factor fits and theory predictions can be found in Figure \ref{fig:3he_fch_theory}, and the $^3$He magnetic form factor fits and theory predictions can be found in Figure \ref{fig:3he_fm_theory}. The new form factor fits broadly agreed with the past Amroun fits except for the shift in the $^3$He magnetic form factor discussed above. Theory is doing a good job predicting the charge form factors using a conventional approach, which accounts for two and three-body nucleon interactions and relativistic corrections, and $\chi$EFT. However, theory struggles to predict the magnetic form factors of either $^3$H or $^3$He. 

Charge densities were then calculated along with charge radii for both targets using the new form factor fits. The average charge radius for $^3$He was found to be 1.90 fm with a standard deviation of 0.00144 fm. This value is in decent agreement with past measurements (Saclay 1.96 fm $\pm$ 0.03 fm and Bates 1.87 fm $\pm$ 0.03 fm). However, the average charge radius for $^3$H was found to be 2.02 fm with a standard deviation of 0.0133 fm. This value is much larger than past measurements (Saclay 1.76 fm $\pm$ 0.09 fm and Bates 1.68 fm $\pm$ 0.03 fm). Recall that there is an additional uncertainty on each of the charge radius results due to not forcing the $\sum Q_i$ to equal unity as discussed in Section \ref{ssec:3he_fits}, and this additional uncertainty is much larger for $^3$H. Not restricting the free parameters in this manner caused the magnitude of the negative slope of the $^3$H $F_{ch}$ at a $Q^2$ of zero to increase causing a larger radius to be found, but this choice also gives us another method to evaluate how well the new fits comply with our physical expectation that $\sum Q_{i_{ch}}$ = 1 as discussed in Section \ref{sec:sog}. Unfortunately, this analysis was unable to quantify the additional uncertainty on the charge radii due to not forcing the $\sum Q_{i_{ch}}$ = 1, but if this uncertainty were accounted for the $^3$H results in particular would likely be in much better agreement with past measurements. %This result is easily attributed to not forcing the $\sum Q_{i_{ch}}$ to sum to unity (i.e. requiring $F_{ch}(0)$ = 1 or that the sum of the electric charges is one).

The updated representative form factors were then used to replace the older parametrization of the form factors in SIMC. Using the updated SIMC Monte Carlo the $^3$He elastic cross section from experiment E08-014 was recalculated using the elastic electron yield from the modified SIMC code. These two cross section values, the newer and older parametrizations, should be in agreement with one another using the cross section extraction technique discussed in Chapter \ref{ch:xs}. In fact, the updated form factor SIMC model finds a cross section of 1.345 $\times$ 10$^{-6}$ $\mu$b/sr $\pm$ 0.086 $\times$ 10$^{-6}$ $\mu$b/sr at a $Q^2$ value of 34.19 fm$^{-2}$ as opposed to the older model which estimated 1.335 $\times$ 10$^{-6}$ $\mu$b/sr.

Going forward, there are several logical extensions to this work. It should be relatively straightforward to expand the fitting code to fit the cross sections of other light nuclei. If the code were to be expanded to heavier nuclei a full phase shift correction would need to be applied in place of $Q^2_{eff}$. Performing these fits with different functions like Fourier-Bessel functions would make for an interesting point of comparison to the SOG fits. Collecting more high $Q^2$ and back angle data would considerably improve our understanding of the form factors, especially $F_m$. JLab's Hall A is well equipped to make these measurements with its maximum back angle of 150$^{\circ}$ and maximum beam energy of 12 GeV.

The discrepancy between theory and experiment on the location of the first diffractive minimum for the $^3$He $F_m$ could also be resolved by JLab. By performing an asymmetry measurement using a polarized $^3$He target and a polarized electron beam the location of the $F_m$ minimum can be found. This is because the asymmetry will flip sign when passing the diffractive minimum. This asymmetry measurement is given in Equation \ref{eq:asymmetry}, where $N^+$ is the normalized counting rate for positive beam helicity and $N^-$ is the normalized counting rate for negative beam helicity \cite{Asymmetry}. This experiment would determine whether theory is wrong, experiment is wrong, or both are wrong on the location of the first $F_m$ minimum.   

\begin{equation} \label{eq:asymmetry}
	A_{meas} = \frac{N^+-N^-}{N^++N^-}
	\text{\equationlabels{Asymmetry Measurement}}
\end{equation}