% Introduction
\chapter{Experiment E08-014} % Main chapter title
\label{ch:experiment} % For referencing the chapter elsewhere, use 

Experiment E08-014 ran in Jefferson Lab's Hall A in 2011. The experiment electron scattering to measure the inclusive cross sections of various targets using both of Hall A's high resolution spectrometers (HRSs). E08-014 aimed to compare heavy targets to two and three-nucleon targets to study the short range correlations (SRC) between these two and three-nucleon clusters. To this end inclusive cross sections for $^2$H, $^3$He, $^4$He, $^12$C, $^{40}$Ca, and $^{48}$Ca were measured in the region of 1.1 GeV/c $<$ Q$^2$ $<$ 2.5 GeV/c. This experiment covered the $x_{Bj}$ range encompassing the quasielastic (QE) region up to $x_{Bj}$ greater than 3. [May want to ask to borrow Zhihong's figure 2.23 to identify kin 3.2 which I used]  [ref Zhihong's ch 2.5 and 3 and possibly the SRC main page http://hallaweb.jlab.org/experiment/E08-014/]

While experiment E08-014 focused on the QE region of electron scattering, one kinematic region, KIN 3.2, also included elastically scattered electrons [Maybe add plot showing elastic band passing through this region]. These elastic events were scattered from a gaseous $^3$He target allowing for the extraction of an elastic cross section. This cross section is located in the little studied region of Q$^2 = $ 34.[whatever it turns out to be] fm$^{-2}$. This understudied region is interesting because it has the potential to constrain and improve previous fits of the $^3$He form factors. 
\section{This is a section}

And here is a Table

\begin{table}
  \centering
    \begin{tabular}{c  c c}
    \toprule
      table & \multicolumn{2}{c}{Slope (\si{\meter\giga\electronvolt\squared\per\micro\radian})} \\
      \midrule
      1 & 2 \\
      5 & 6\\
      \bottomrule
    \end{tabular}
    \caption[Here is a table]{Table}
    \label{tab:table}
\end{table}


How about and Equation? Remove equation labels if you would like in the text and in the main file where the table of equations is printed.
%Example equation
\begin{equation}
    1 + 1 = 2
    \label{eq:sum1}
  \text{\equationlabels{1st sum}}
\end{equation}


How about a figure:
\begin{figure}[!htb]
    \centering
    \includegraphics{Chapters/Ch_Introduction/Figure_Placeholder.png}
    \caption{Caption}
    \label{fig:my_label}
\end{figure}

Use Units \SI{15}{\ampere}: Change the units.tex to your liking.
How about a citation~\cite{Book:PeskinSchroeder1995,Book:Jackson} or a reference Eq.~\ref{eq:sum1}.

\lipsum[6-7]