\chapter{Global Fits} % Main chapter title
\label{ch:global_fits} % For referencing the chapter elsewhere, use 

This chapter will discuss the world data for $^3$H and $^3$He elastic cross sections. This data will then be fit using a sum of Gaussians (SOG) technique. These new global fits will incorporate modern data sets added to the world data since the last global fits were performed. This will include new high Q$^2$ data from JLab for $^3$He as well as the $^3$He cross section extracted in this thesis. The SOG fitting technique allows the electric and magnetic form factors to be easily extracted as well as for charge radii to be calculated. These new results will then be compared to past fits as well as some theory predictions.

\section{World Data}
\label{sec:world_data}

The world data for $^3$H and $^3$He elastic cross sections spans numerous decades, laboratories, and continents. Due to the expansiveness of the data set there are many inconsistent methodologies employed in the different analyses collected. Efforts were taken to make these comparisons as consistent as reasonably possible, however it was often impossible with the existing literature to be certain which techniques were used. Methodological differences in modifications like radiative corrections and Dirac wave Born approximation (DWBA) techniques would be extremely time consuming to force all data sets into complete agreement. As such some of the datasets fit together are not completely apples-to-apples comparisons. Fortunately, the methodological differences result in very minor changes to the final cross sections, and thus do not significantly impact the efficacy of the new global fits.

Table ~\ref{tab:world_data_3h} lists the literature comprising the current world data of $^3$H elastic cross sections and Table ~\ref{tab:world_data_3he} contains the $^3$He world data compiled for this analysis. The table is organized chronologically from oldest dataset to most recent. The table lists the title of each publication, the first author listed on each publication, the journal the publication appeared in, and the location of the measurement with the year it was published. The table also contains physics data on each experiment. This physics data includes the rough Q$^2$ range covered by the experiment, whether the paper lists cross sections explicitly, whether the paper lists form factors explicitly, if the paper applied a phase shift correction to account for the plane wave approximation, and finally a brief note on the radiative corrections each paper used. Whenever a table entry wasn't listed or was unclear in the literature a `?' was used.
 

%\begingroup
%\begin{sideways}
\begin{landscape}
\pagestyle{empty}
\small
% Text layout
\topmargin 2.75cm      %2.75cm + shifts left.
\oddsidemargin 0.5cm  %0.5cm - shifts up.
\evensidemargin 0.5cm  %0.5cm Does nothing I can see.
\textwidth 40cm       %16cm
\textheight 21cm       %21cm
\hoffset 1.cm          %1.cm + shifts down
\voffset -2.25cm        %- shifts right

%\Rotatebox{90}{
%\thispagestyle{empty}
\begin{longtable}{c c c c c c c c c c}%[!h]
\caption{\bf{Accumulated World Data for $^3$H Elastic Scattering}}\\
%\begin{tabular}%{l c c c c c c c c c c c}
\hline
\hline
\textbf{Title} & \textbf{Authors} & \textbf{Journal} & \textbf{\thead{Date/\\Location}} & \textbf{\thead{Q$^2$ Range \\ (fm$^{-2}$)}} & \textbf{\thead{Cross \\ Sections}} & \textbf{\thead{Form \\ Factors}} & \textbf{\thead{Phase \\ Shift}} & \textbf{\thead{Radiative \\ Corrections}} \\
\hline

\thead{Elastic Electron Scattering\\from Tritium and Helium-3} & \makecell{Collard} & \makecell{Phys. Rev.\\ Vol. 138, No. 1B \cite{Article:Collard}} & \makecell{1965*\\SLAC} & 1-8 & Yes & Yes & ? & Tsai \\

\thead{Triton Form Factor\\ from 0.29-1.00 fm$^{-2}$} & \makecell{Beck\\Asai} & \makecell{Phys. Rev. C\\ Vol. 25, No. 3, 1152-1155 \cite{Article:Beck82}} & \makecell{1982\\ Saskatchewan} & 0.29-1 & \makecell{Yes} & \makecell{Yes\\ ($G_E$)} & \makecell{?} & \makecell{Meister\\Yennie} \\

\thead{Tritium Form Factors\\at Low q} & \makecell{Beck} & \makecell{Phys. Rev. C\\ Vol. 30, No. 5, 1403-1408 \cite{Article:Beck84}} & \makecell{1984*\\ NBS MIT} & 0.05-3 & \makecell{Yes} & \makecell{Yes} & \makecell{Yes\\ ($q_{eff}$)} & \makecell{Mo/Tsai} \\

\thead{Tritium Electromagnetic\\ Form Factors} & \makecell{Juster} & \makecell{Phys. Rev. Letters\\ Vol. 55, No. 21, 2261-2264 \cite{Article:Juster}} & \makecell{1985\\Saclay} & 0.3-31 & \makecell{In Amroun\\1994} & \makecell{Yes\\ (SOG)} & \makecell{?} & \makecell{Auffret} \\

\thead{Isoscalar and Isovector Form\\ Factors of $^3$H and $^3$He for Q\\below 2.9 fm$^{-1}$ from Electron-\\Scattering Measurements} & \makecell{Beck} & \makecell{Phys. Rev. Letters\\ Vol. 59, No. 14, 1537-1540 \cite{Article:Beck87}} & \makecell{1987\\Bates} & 0.03-9 & \makecell{No} & \makecell{Yes\\ (Iso)} & \makecell{Yes} & \makecell{Mo/Tsai} \\

\thead{$^3$H and $^3$He \\ Electromagnetic \\ Form Factors} & \makecell{Amroun} & \makecell{Nuc. Phys.\\ A579 596-626 \cite{Article:Amroun}} & \makecell{1994*\\Saclay} & 1-47 & Yes & Yes & Yes & \makecell{Mo/Tsai, Schwinger \\ and bremsstrahlung +\\ Landau Straggling} \\
%\hline
%\hline
%\endhead
\hline
\hline
%\endfoot
%\end{tabular}
\label{tab:world_data_3h}
\end{longtable}

%}%End rotate box.
\end{landscape}
%\end{sideways}
%\endgroup
%\pagestyle{plain}



\begin{landscape}
\pagestyle{empty}
\small
% Text layout
\topmargin 2.75cm
\oddsidemargin 0.5cm
\evensidemargin 0.5cm
\textwidth 16cm 
\textheight 21cm
\hoffset 1.cm
\voffset -1.75cm

%\Rotatebox{90}{
%\thispagestyle{empty}
\begin{longtable}{c c c c c c c c c c}%[!h]
\caption{\bf{Accumulated World Data for $^3$He Elastic Scattering}}\\
%\begin{tabular}%{l c c c c c c c c c c c}
\hline
\hline
\textbf{Title} & \textbf{Authors} & \textbf{Journal} & \textbf{\thead{Date/\\Location}} & \textbf{\thead{Q$^2$ Range \\ (fm$^{-2}$)}} & \textbf{\thead{Cross \\ Sections}} & \textbf{\thead{Form \\ Factors}} & \textbf{\thead{Phase \\ Shift}} & \textbf{\thead{Radiative \\ Corrections}} \\
\hline

\thead{Elastic Electron Scattering\\ from Tritium and Helium-3} & \makecell{Collard} & \makecell{Phys. Rev.\\ Vol. 138, No. 1B \cite{Article:Collard}} & \makecell{1965*\\SLAC} & 1-8 & Yes & Yes & ? & Tsai \\

\thead{Elastic Electron Scattering\\from $^3$He at High\\ Momentum Transfer} & \makecell{Bernheim} & \makecell{Lettere Al Nuovo Cimento\\ Vol. 5, No. 5, 431-434 \cite{Article:Bernheim}} & \makecell{1972\\Orsay} & 9-16 & \makecell{No} & \makecell{Yes} & \makecell{?} & \makecell{``Usual"} \\

\thead{Electromagnetic Structure\\of the Helium Isotopes} & \makecell{McCarthy} & \makecell{Phys. Rev. C\\ Vol. 15, No. 4, 1396-1414 \cite{Article:McCarthy}} & \makecell{1977\\ Stanford HEPL} & 0.3-20 & No & Yes & \makecell{Yes} & Mo/Tsai \\

\thead{Low-Momentum-Transfer\\ Elastic Electron\\ Scattering from $^3$He} & \makecell{Szalata} & \makecell{Phys. Rev. C\\ Vol. 15, No. 4, 1200-1203 \cite{Article:Szalata}} & \makecell{1977*\\National Bureau\\ of Standards} & 0.03-0.33 & \makecell{Yes\\$^3$He/$^{12}$C\\Exp.} & \makecell{Yes \\($F_{ch}^2$)} & \makecell{Yes} & \makecell{``In the\\Standard\\Fashion"}\\

\thead{Elastic Scattering\\ from $^3$He and $^4$He at\\ High Momentum Transfer} & \makecell{Arnold} & \makecell{Phys. Rev. Letters\\ Vol. 40, No. 22 \cite{Article:Arnold}} & \makecell{1978*\\SLAC} & 18-103 & No & \makecell{Yes \\($A^{1/2}$)} & \makecell{?} & \makecell{?} \\

\thead{Magnetic Form\\ Factor of $^3$He} & \makecell{Cavedon} & \makecell{Phys. Rev. Letters\\ Vol. 49, No. 14, 986-989 \cite{Article:Cavedon}} & \makecell{1982\\Saclay} & 7-32 & \makecell{In Amroun\\ 1994} & \makecell{Yes\\ ($F_M^2$)} & \makecell{Yes\\(HADES)} & \makecell{Yes} \\

\thead{$^3$He Magnetic\\ Form Factor} & \makecell{Dunn} & \makecell{Phys. Rev. C\\ Vol. 27, No. 1, 71-82 \cite{Article:Dunn}} & \makecell{1983*\\Bates} & 0.08-11 & \makecell{Yes} & \makecell{Yes} & \makecell{Yes} & \makecell{Bergstrom +\\ Mo/Tsai} \\

\thead{Elastic Electron Scattering\\from $^3$He and $^4$He} & \makecell{Otterman} & \makecell{Nuclear Physics\\ A436 688-698 \cite{Article:Otterman}} & \makecell{1985\\Mainz} & 0.2-3.7 & \makecell{No} & \makecell{Yes} & \makecell{Yes\\(HADES)} & \makecell{Mo/Tsai} \\

\thead{Isoscalar and Isovector Form\\Factors of $^3$H and $^3$He for Q\\below 2.9 fm$^{-1}$ from Electron-\\Scattering Measurements} & \makecell{Beck} & \makecell{Phys. Rev. Letters\\ Vol. 59, No. 14, 1537-1540 \cite{Article:Beck87}} & \makecell{1987\\Bates} & 0.03-9 & \makecell{No} & \makecell{Yes\\ (Iso)} & \makecell{Yes} & \makecell{Mo/Tsai} \\

\thead{Isospin Separation of Three-\\Nucleon Form Factors} & \makecell{Amroun} & \makecell{Phys. Rev. Letters\\ Vol. 69, No. 2, 253-256 \cite{Article:Amroun92}} & \makecell{1992*\\Saclay} & 2.6-37 & \makecell{In Amroun\\ 1994} & \makecell{No} & \makecell{Yes} & \makecell{``Standard"} \\

\thead{$^3$H and $^3$He \\ Electromagnetic \\ Form Factors} & \makecell{Amroun} & \makecell{Nuc. Phys.\\A579  596-626 \cite{Article:Amroun}} & \makecell{1994*\\Saclay} & 2-48 & Yes & Yes & Yes & \makecell{Mo/Tsai, Schwinger \\ and bremsstrahlung +\\ Landau Straggling} \\

\thead{Measurement of the Elastic\\Magnetic Form Factor of 3He\\at High Momentum Transfer} & \makecell{Nakagawa} & \makecell{Phys. Rev. Letters\\ Vol. 86, No. 24, 5446-5449 \cite{Article:Nakagawa}} & \makecell{2001*\\ Bates} & 6-43 & \makecell{Yes} & \makecell{Yes\\ ($|F_M|^2$)} & \makecell{Yes} & \makecell{Mo/Tsai} \\

\thead{JLab Measurements of the\\$^3$He Form Factors at Large\\ Momentum Transfers} & \makecell{Camsonne} & \makecell{Phys. Rev. Letters\\ Vol. 119, No. 162501, 1-6 \cite{Article:Alex}} & \makecell{2016*\\ JLab} & 25-61 & \makecell{Yes} & \makecell{Yes} & \makecell{Yes\\ ($q_{eff}$)} & \makecell{Yes} \\

%\hline
%\hline
%\endhead
\hline
\hline
%\endfoot
%\end{tabular}
\label{tab:world_data_3he}
\end{longtable}

%}%End rotate box.
\end{landscape}



The datasets used in the SOG global fits in this thesis are marked with a * after the listed dates. Not all of the datasets were able to be used in this analysis for various reasons. The most common reason a dataset was not used was simply that the publication did not list its cross section data points explicitly in the publication so they could not be added to the fit. Another common reason was publications listing only the extracted form factors and not cross sections. This is not an issue when the publication also lists the beam energy and scattering angle for each data point (or Q$^2$ and one of either the energy or angle) as the cross section can be computed using these values. However, numerous publications list only the form factors without energies or angles making it impossible to calculate a cross section for each data point to be used in the global fit. Some publications like Arnold 1978 ~\cite{Article:Arnold} used different ways to parametrize form factors, and whenever possible these methods were converted to cross sections.

\section{Sum of Gaussians Parametrization}
\label{sec:sog}

The sum of Gaussians (SOG) parametrization is a powerful method for fitting nuclear cross section data developed by Ingo Sick in the early 1970s \cite{Article:SOG}. It attempts to fit elastically scattered electron cross sections by representing the electric form factor (sometimes referred to as the charge form factor, F$_{ch}$) and the magnetic form factor (F$_{m}$) as the sum of numerous Gaussians. The technique attempts to remain model independent while taking several physical requirements for the form factors and nuclear wave functions in to account. However, a model dependence of sorts does enter the fits in the form of the radii at which the various Gaussians are situated. SOG fits make the extraction of the charge and magnetic form factors easy to extract, and along with them the charge density and charge radii of the target.

Sick outlines the rules for removing a global model dependence when fitting cross section data as follows,

\begin{quote}
	\begin{enumerate}
		\item ``Accept some clearly specified limitation to generality (accept some model dependence), since data with infinite q$_{max}$ are not available (wavelengths smaller than $\lambda = \frac{2\pi}{q_{max}}$ are not determined by experiment).
		\item Choose a restriction to generality which can be justified by physical arguments.
		\item Write the density in a manner which decouples densities at different radii as much as possible." \cite{Article:SOG}
	\end{enumerate}
\end{quote} 

One of the first physical restrictions that can be applied by the SOG parametrization is on the nuclear charge densities. No structures in the nuclear charge densities are allowed to be larger than the RMS radii of the proton \cite{Article:SOG}. As this thesis is often using \cite{Article:Amroun} as a point of comparison this work employs the same minimum size allowed for structure used by Amroun \textit{et al.} of 0.8 fm, or slightly less than the proton's radius. 

A second restriction comes from the fact that the amplitudes of the high frequency components of the nuclear wave functions have limits placed on them. Paraphrasing Sick this can be seen by noting that the Schr{\"o}dinger equation strongly couples the second derivative of the radial wave functions to the proton's separation energy. This implies that the proton will become less bound for larger amplitudes and shorter wavelengths of the radial wave functions. This is yet another reason to forbid certain structures in the nuclear charge density smaller than a certain width \cite{Article:SOG}.

Gaussians are used to build the structure of the fits since they mimic the peaks of the radial wave functions. They also fall off quickly enough so as to not strongly interfere with other Gaussians not nearby them satisfying rule 2 above. Gaussians also work well with the rules and limitations imposed earlier. One can write the nuclear charge density as shown in Equation ~\ref{eq:sog_rho_no_tail}, where the charge density is represented as a sum of numerous Gaussians set at different radii R$_i$. The A$_i$ amplitudes are fitted to the cross sectional data. These Gaussians have their full width at half maximum restricted by the parameter $\gamma$ as required by the physical restrictions imposed above. The smallest width structure allowed is given by $\Gamma$ where $\Gamma = 2\gamma \sqrt{ln(2)}$ \cite{Article:SOG}. 

\begin{equation} \label{eq:sog_rho_no_tail}
	\rho(r) \propto \sum_{i=1}^N A_i e^{-\left( r-R_i \right)^2/\gamma^2}
\end{equation}

As previously mentioned, the R$_i$, representing the radii at which different Gaussians are located, form their own sort of model dependence. Since we are unable to study what happens above q$_{max}$ the R$_i$ are analogous to a model of how the charge density behaves above q$_{max}$. This issue can be resolved by choosing many different R$_i$ values randomly and fitting the A$_i$ to the data for each set of R$_i$ chosen randomly. The choice of R$_i$ may be random but it does have numerous conditions applied. More will be said on the selection of the R$_i$ later on***(maybe have an Ri selection section to link to later). 

Once a large number of fits of the data using different sets of R$_i$ have been generated the `good' fits must be distinguished from the `bad'. This is done in several ways which will be discussed in more detail in Section ***, and include finding lower $\chi^2$ fits as well as making sure the fit's form factors appear physical. Once the `good' fits are identified an error band can be built up by plotting each of the fits on top of one another. After a sufficient number of different `good' R$_i$ sets have been fitted to the data the whole of the available model space has been explored. 

The charge density is expected to have a derivative of zero at a $r=0$. This is not accounted for in Equation ~\ref{eq:sog_rho_no_tail}. To resolve this issue a tail can be added to each Gaussian that represents the Gaussian's behavior at $r<0$. This modified definition of the charge density is given in Equation ~\ref{eq:sog_rho} \cite{Article:SOG}.

\begin{equation} \label{eq:sog_rho}
	\rho(r) = \frac{Ze}{2 \pi^{3/2}\gamma^3} \sum_{i=1}^N \frac{Q_i}{1+\frac{2R_i^2}{\gamma^2}} \left( e^{-\left( r-R_i \right)^2/\gamma^2} + e^{-\left( r+R_i \right)^2/\gamma^2} \right)
\end{equation}

\noindent Equation \ref{eq:sog_rho} is normalized by Equation ~\ref{eq:normalization}. The Q$_i$ are now the parameters fitted to the data. The Q$_i$ are required to be positive as they represent the fraction of electric or magnetic charge carried by each Gaussian. $\sum Q_i=1$ is also required of the Q$_i$ terms as all of the charge fractions must sum to the total charge (one). $Z$ is the atomic number of the target and $e$ is the elementary charge \cite{Article:SOG}.

\begin{equation} \label{eq:normalization}
	4 \pi \int_0^{\infty} \rho(r) r^2 dr = Ze
\end{equation}

When using the plane wave Born approximation (PWBA) the electric and magnetic form factors can be parametrized as in Equation ~\ref{eq:sog_ffs} \cite{Article:SOG}.

\begin{equation} \label{eq:sog_ffs}
	F_{(ch,m)}(q) = exp \left(-\frac{1}{4} q^2 \gamma^2 \right) \sum^{n}_{i=1} \frac{Q_i{_{(ch,m)}}}{1+2R^2_i/\gamma^2} \left( \cos(qR_i) + \frac{2R^2_i}{\gamma^2} \frac{\sin(qR_i)}{qR_i} \right)
\end{equation}

\noindent At this point we will follow the procedure laid out in \cite{Article:Amroun} and note that there is a typo in the reference. In \cite{Article:Amroun} Equation (1) the -1/2 in the exponent should be a -1/4. Again the Q$_i$ are fitted to the data and represent the fraction of the electric or magnetic charge carried by each Gaussian. The R$_i$ are the radii at which the Gaussians are placed. $q$ is the four-momentum transferred via the virtual photon as discussed in Section ~\ref{sec:kinematics}. Lastly $\gamma$ is defined as $\gamma \sqrt{\frac{3}{2}}=0.8$ fm \cite{Article:Amroun}. 

The cross section can be represented in PWBA with the SOG parametrization as shown in Equation ~\ref{eq:xs}.

\begin{equation} \label{eq:xs}
	\frac{d\sigma}{d\Omega} = \left( \frac{d\sigma}{d\Omega} \right)_{Mott} \frac{1}{\eta} \left[ \frac{q^2}{\boldsymbol{q}^2}F_{ch}^2(q) + \frac{\mu^2q^2}{2M^2} \left( \frac{1}{2} \frac{q^2}{\boldsymbol{q}^2} + \tan^2 \left( \frac{\theta}{2} \right) \right)F_{m}^2(q) \right]
\end{equation}

\noindent Here $\eta = 1 + q^2/4M^2$, $q^2$ is the squared four-momentum transfer from ~\ref{eq:Q^2}, $\boldsymbol{q}^2$ is the three-momentum, $\mu$ is the magnetic moment of the target ($\mu_{^3He}$ = -2.1275*(3.0/2.0) and $\mu_{^3H}$ = 2.9788*(3.0/1.0)), and $M$ is the mass of the target (M$_{^3He}$ = 3.0160293 amu and M$_{^3H}$ = 3.0160492 amu) ~\cite{Article:Amroun}. 

The Mott cross section is shown in equation ~\ref{eq:mott}.

\begin{equation} \label{eq:mott}
	\left( \frac{d\sigma}{d\Omega} \right)_{Mott} = Z^2 \frac{E'}{E_0} \frac{\alpha^2 \cos^2 \left( \frac{\theta}{2} \right)^2}{4E_0^2 \sin^2 \left( \frac{\theta}{2} \right)^4} 
\end{equation}
 
\noindent $Z^2$ accounts for the charge of the target with $Z$ being the target's atomic number, $\frac{E'}{E_0}$ is the recoil factor with $E_0$ being the scattered electrons initial energy and $E'$ is the energy after scattering, $\alpha$ is the fine structure constant, and $\theta$ is the scattering angle. It is extremely important to be mindful of the units one is using when working with these equations. Be cautious of interchanging degrees and radians for the scattering angle, fm$^{-2}$ and GeV$^2$ for the squared four-momentum values, fm$^{-1}$ and GeV for the energies, and amus and GeV for the mass units. Equation ~\ref{eq:gev2fm} shows the equivalent amount of GeV$^2$ to fm$^{-2}$ in nuclear units.

\begin{equation} \label{eq:gev2fm}
	1 \; GeV^2 \approx 25.7 \; fm^{-2}
\end{equation}

The assumption that the wave functions of the electrons are plane waves is not entirely correct. The nucleus' charge distorts these wave functions due to the Coulomb interaction, and thus shifts the Q$^2$ value to Q$^2_{eff}$ given in Equation ~\ref{eq:qeff}. This leads to Q$^2_{eff}$ taking the place of Q$^2$ in the above equations in this section (i.e. Q$^2$ is taken from the literature and then Q$^2_{eff}$ is then calculated and used in the fits) \cite{Article:Alex}.

\begin{equation} \label{eq:qeff}
	Q^2_{eff} = Q^2 \left(  1+ \frac{1.5Z\alpha}{E_0*1.12*A^{\frac{1}{3}}}   \right)^2
\end{equation}

\noindent Here $A$ is the mass number and the other variables are defined above. The three-momentum, $\boldsymbol{q}^2$, is then given by Equation ~\ref{eq:three-momentum} where $\nu=E_0-E'$ as in ~\ref{eq:nu}.

\begin{equation} \label{eq:three-momentum}
	\boldsymbol{q}^2 = \nu^2 - Q^2_{eff}
\end{equation}

\section{New SOG Fits}
\label{sec:new_fits}

The world data for $^3$H and $^3$He described in Section ~\ref{sec:world_data} will be fitted with the sum of Gaussians parametrization described in Section ~\ref{sec:sog} in this section. This section will explain the choices made for each of the SOG fits such as the number of Gaussians used to fit the world data. It will also describe the choices made involving the Q$_i$ fit parameters. The placement and spacing of the R$_i$ radii at which the Gaussians are placed will also be discussed. A method used to try to optimize the fits by adjusting the R$_i$ spacing while attempting to minimize $\chi^2$ will be described. 

\subsection{Gaussian Radii Placement}
\label{ssec:radii}

As discussed in Section ~\ref{sec:sog}, the R$_i$ are the radii at which the SOG Gaussians are placed. This means that they represent a sort of model dependence. To explore all of the model space many different random R$_i$ combinations must be used to fit the world data. However, selecting the R$_i$ totally at random is extremely inefficient since we are only interested in R$_i$ combinations that yield reasonable fits and physical form factors. 

To explore the R$_i$ combinations we want to apply a few rules to their selection. The first of which is that there is some radii, R$_{max}$, beyond which the charge density has fallen almost to zero. Therefore, there is no reason to position any R$_i$ beyond R$_{max}$. For $^3$H and $^3$He R$_{max}$ is $\approx$ 5 fm, although it is allowed to diverge from this radii as the fits are optimized. For the majority of fits, after the optimization procedure discussed below, R$_{max}$ is found to be in the range of 4 fm $<$ R$_{max}$ $<$ 6 fm centered around 5 fm.

Once a reasonable upper limit on the radii is established the spacing separating the R$_i$ from one another must be determined. It has been found that the spacing of the R$_i$ for R$_i$ $<$ R$_{max}$/2 should be approximately half as far apart as the R$_i$ spacing for the radii positioned at R$_i$ $>$ R$_{max}$/2 \cite{Article:SOG}. This is done so that the charge density region with more charge, i.e. closer to the nucleus, is described by more Gaussians. This allows the structure to be better captured by the SOG fits. Further away from the nucleus, where there is less charge, fewer Gaussians are needed to accurately describe the structure of the charge density.   

Once the R$_i$ are selected each combination of R$_i$ are fitted using the SOG parametrization. Since we are interested in the fits that best describe the data it is logical to search for the lowest $\chi^2$ fits. We also want fits that produce form factors that conform to the physical priors we expect so it is important to inspect the form factors visually for physicality. We define $\chi^2$ as in Equation ~\ref{eq:chi2}, where $N$ is the number of data points being fit, $\sigma_{exp}$ is the experimentally measured cross section at a particular Q$^2$, $\sigma_{fit}$ is the cross section given by the global fit at the same Q$^2$ as the experimental cross section, and $\Delta$ is the total uncertainty attached to the experimental cross section at the given Q$^2$. A lower $\chi^2$ value naively indicates a better fit, but numerous flaws can occur when using only $\chi^2$ \cite{doug_stats}. Numerous other tests for the `goodness' of the fit were also applied to avoid this problem and are discussed in ~\ref{ssec:ngaus}.  

\begin{equation} \label{eq:chi2}
	\chi^2 = \sum_{n=1}^N \frac{\left( \sigma_{exp}-\sigma_{fit} \right)^2}{\Delta^2}
\end{equation}

Initial R$_i$ spacings tend to be fairly unfavorable and produce large $\chi^2$ values and strangely shaped form factors. To minimize the $\chi^2$ as much as possible the R$_i$ values need to be allowed to shift. This is accomplished by first fitting the data with an initial set of R$_i$ values. After the initial fit is done each of the R$_i$ values is then optimized. If R$_0$ was initially 0.2 fm the fit would then be redone with R$_0$ = 0.1 fm and then R$_0$ = 0.3 fm. The R$_i$ are each shifted up and down 0.1 fm until $\chi^2$ gets larger. The R$_i$ that yielded the smallest $\chi^2$ is then kept as the `optimal' R$_i$. Once this procedure is completed for each R$_i$ in ascending order the lowest, or at least close to the lowest, $\chi^2$ value for R$_i$ similar to the initial R$_i$ has been found.  

As an example, let us examine the initial R$_i$ spacings for $^3$H using eight Gaussians. While the order of the Gaussians is irrelevant it is easier to code the R$_i$ in ascending radii length. Next we choose R$_0$-R$_7$ to meet the rules defined above. We want all of the R$_i$ to sum to approximately 5 fm, and the R$_i$ spacing between consecutive Gaussians should be smaller at smaller radii. For $^3$H with eight Gaussians the initial R$_i$ spacing is produced within given ranges randomly and then optimized as previously described. The ranges for the R$_i$ spacings are divided in steps of 0.1 fm. The first Gaussian is placed near R = 0 fm and was chosen to be R$_0$ = 0.2-0.3 fm. This means that R$_0$ was randomly selected to initially be 0.2 fm or 0.3 fm. Note that an R$_0$ of 0 fm leads to poles in the parametrization. To avoid this issue a small number is used in place of 0 if R$_0$ = 0 fm is found to be the optimal radius. 

After the first Gaussian is placed at R$_0$ Gaussians R$_{1-7}$ are placed by semi-randomly choosing their distance from the radii prior to them. The spacing for R$_{1-4}$ = 0.5-0.6 fm and for R$_{5-7}$ = 0.8-0.9 fm chosen randomly in the same manner as R$_0$. Notice that the radii further from the nucleus are placed approximately twice as far apart as the inner radii in accordance with the rules previously described. We can take the average spacing of each consecutive radius and sum them to find R$_{max}$. Doing this we find 0.25 fm + 4*0.55 fm + 3*0.85 fm = 5 fm which is the target R$_{max}$. This process is then repeated for hundreds of semi-randomly generated R$_i$ sets which span the model space for $^3$H and $^3$He. 

\subsection{Number of Gaussians}
\label{ssec:ngaus}

To utilize the SOG parametrization it is necessary to select the number of Gaussians, N$_{Gaus}$, to use for each fit. This process involves balancing several competing interests. If too few Gaussians are used the structure of the form factors may not be described in enough detail, but if too many Gaussians are used the data may be overfit. Overfitting would lead to the statistical noise in the data being mistaken for signal. The goal is then to fit the data as well as possible with no more parameters than required. A commonly used tool for selecting the best model is to calculate the $\chi^2$ value for a fit, however $\chi^2$ alone is insufficient and can often be deceptive and lead to issues such as overfitting \cite{doug_stats}. 

To avoid this issue numerous other tests and metrics were applied when selecting N$_{Gaus}$ for $^3$H and $^3$He. Among these are the $\chi^2$ value, the reduced $\chi^2$ value, Bayesian information criterion, Akaike information criterion, the sums of the fractions of the electric and magnetic charges held by the Gaussians, the percentage of fits that were deemed `good', and finally a visual inspection of the form factors for known physical characteristics \cite{doug_stats}. By combining these different tests it is possible to determine the number of Gaussians that provide an optimal fit. Note that it is not uncommon for two consecutive numbers of Gaussians to yield reasonably similar fits.

Reduced $\chi^2$, or r$\chi^2$, is similar to $\chi^2$ from Equation ~\ref{eq:chi2} except that it takes the number of data points and the number of parameters used in the fit in to account. The equation for reduced $\chi^2$ used in this analysis is given in Equation ~\ref{eq:rchi2}, where $\chi^2$ is from ~\ref{eq:chi2}, $N$ is the number of data points in the fit, and $N_{var}$ is the number of free parameters, or variables, used in the fit. Note that while $\chi^2$ must always decrease with the number of parameters added r$\chi^2$ can increase if too many parameters have been added. This makes finding the fits with the lowest r$\chi^2$ an elementary, but still useful, test that the proper number of parameters are being use to describe the data.

\begin{equation} \label{eq:rchi2}
	r\chi^2 = \frac{\chi^2}{N-N_{var}-1}
\end{equation}

The next two tests applied to determine the number of Gaussians to use in the SOG fits are Akaike information criterion (AIC) defined in Equation ~\ref{eq:AIC} \cite{Article:AIC} and Bayesian information criterion (BIC) defined in Equation ~\ref{eq:BIC} \cite{Article:BIC} \cite{doug_stats}. AIC and BIC are both a more advanced type of statistical test useful for selecting the proper model to use. The primary difference between the two is that BIC applies a larger penalty based on the number of model parameters used to fit the data. The way to select the correct model is to find the lowest AIC and BIC values, while remembering that these tests may choose slightly different models than the other tests and each other. To determine how much more evidence there is for one model versus another we can look at the difference between their BIC values, $\Delta$BIC. A $\Delta$BIC of $0<\Delta$BIC$<2$ indicates no real difference between models, $2<\Delta$BIC$<6$ indicates that there is positive evidence for the lower valued model, $6<\Delta$BIC$<10$ indicates strong evidence for the lower valued model, and $\Delta$BIC$>10$ indicates very strong evidence for the lower valued model \cite{Article:Delta_BIC}.

\begin{equation} \label{eq:AIC}
	AIC = N \ln\left( \frac{\chi^2}{N} \right) + 2 N_{var}
\end{equation}

\begin{equation} \label{eq:BIC}
	BIC = N \ln\left( \frac{\chi^2}{N} \right) +  \ln\left( N \right) N_{var}
\end{equation}

When selecting the number of Gaussians to use the sum of the electric and magnetic charges is also examined. The sum of the Q$_i$ charges should sum to a charge of unity, however the fits do not enforce this requirement. Instead the sum of the Q$_i$ are allowed to fluctuate with the best fit values of the individual Q$_i$. This then makes the sums another sort of test of the goodness of each fit. A `better' fit, or one that complies more with our predetermined knowledge of the form factors, will have Q$_i$ sums closer to unity. Values further from unity can indicate a worse fit, but they also help to indicate where more data is needed.

Finally a visual inspection of the form factors is applied. It is known that the form factors should have sharp minima as discussed in Section ~\ref{sec:ffs}. Often the fits will have only a dip in the form factor where a sharp minimum should exist and can thus be discarded. An example of this is seen in Figure ~\ref{fig:3he_fch_no_cut}. These nonphysical dip only fits tend to have higher $\chi^2$ values so cutting on $\chi^2$ can generally eliminate them. More specifically, this is done by plotting the charge form factor, F$_{ch}$, and lowering the $\chi^2$ cut until all of the dip only minima fits are removed leaving only the sharp minima expected. These remaining fits are deemed to be the `good' fits. This process is done with the charge form factor as we have better data there. This procedure generally improves the corresponding magnetic form factors as well, but the lack of high Q$^2$ data for F$_m$ leads to more nonphysical or odd fits of F$_m$.

When fitting with any number of Gaussians many of the resulting fits do not meet the definition of a `good' fit described above. The ratio of the `good' fits to total fits attempted is representative of the likelihood of fits of N$_{Gaus}$ to converge to physical looking fits. Assume N$_{Gaus}$ = 9 gives a `good' fit 40$\%$ of the time, and N$_{Gaus}$ = 8 gives a `good' fit 5$\%$ of the time and has a slightly lower average BIC than N$_{Gaus}$ = 9. This analysis takes the low convergence rate as evidence against the slightly lower BIC results and may favor the marginally higher BIC results due to their better convergence rate assuming $\Delta$BIC between the two average BIC values is small. 

Previous analyses have also done a good job locating the first minima of the form factors and can be used to check the reasonableness of this analysis' fits. For example \cite{Article:Amroun} locates the first minima of both $^3$H and $^3$He fairly well, and \cite{Article:Alex} does the same for $^3$He. If this analysis' results diverge significantly in the previously well understood regions that is taken to be a strike against the model selected. Note that some movement in the $^3$He magnetic form factor is not unexpected since new high Q$^2$ data is being incorporated into this analysis.

Now that we have established the tools with which to select a model let us determine how many Gaussians to use when fitting $^3$H and $^3$He. $^3$He will be examined first due to there being more, and often higher quality, data for $^3$He than $^3$H. The method used to determine the number of Gaussians (i.e. the model) to use to fit the data was to run 100 fits of the $^3$He world data for each reasonable value of N$_{Gaus}$. Then the various tests and metrics laid out above were computed for that value of N$_{Gaus}$. These results were then compared and the `optimal' number of Gaussians was determined.

Table ~\ref{tab:3he_ngaus} shows the results of this model selection analysis for $^3$He. All of the values in the table are averages of the surviving `good' fits and the best values are bolded as is the final selection for N$_{Gaus}$. $\chi^2_{max}$ is the maximum $\chi^2$ cut that removed all of the nonphysical dip minima in the charge form factor, and `Good Fits' is the number of the 100 fits that survived this cut. Note that not all of these factors are weighted equally. The highest preference is given to BIC and AIC followed by r$\chi^2$ and visually inspecting the form factors. The other factors add more detailed information and are used more as tiebreakers and to raise red flags if something major is wrong.

%\begin{table}[!h]
%\centering
%\begin{tabular}{|c c c c c c c c c|}
%\hline
%%\makecell{\textbf{Absorption}\\ \textbf{Spectrum Shape}} & \textbf{Paint QE} & \makecell{\textbf{Visual}\\ \textbf{Opacity}} \\
%\textbf{N$_{Gaus}$} & \textbf{Avg. $\chi^2$} & \textbf{r$\chi^2$} & \textbf{BIC} & \textbf{AIC} & \textbf{$\sum$Q$_{i_{ch}}$} & \textbf{$\sum$Q$_{i_{m}}$} & \textbf{$\chi^2_{max}$} & \makecell{\textbf{`Good'}\\ \textbf{Fits}} \\
%\hline
%8 & 584.902 & 2.41695 & 255.440 & 223.228 & \textbf{1.00769} & 1.11065 & 765 & 11 \\
%9 & 470.435 & 1.96014 & 204.590 & 172.375 & 1.00851 & \textbf{1.02161} & 521 & 58 \\
%10 & 469.177 & 1.97133 & 209.454 & 173.793 & 1.00812 & 1.08196 & 519 & 66 \\
%11 & 445.136 & 1.88617 & \textbf{201.387} & 162.233 & 1.00843 & 1.04007 & 503 & 67 \\
%\textbf{12} & \textbf{436.264} & \textbf{1.86438} & 201.727 & \textbf{159.045} & 1.00839 & 1.02557 & 501 & \textbf{75} \\
%13 & 439.084 & 1.89260 & 208.924 & 162.685 & 1.00947 & 1.03975 & 500 & 56 \\
%\hline
%\end{tabular}
%\caption{\bf{Determination of N$_{Gaus}$ for $^3$He}}
%\label{tab:3he_ngaus}
%\end{table}

\vspace{6mm}
\begin{table}[!h]
\centering
\begin{tabular}{|c c c c c c c c c|}
\hline
%\makecell{\textbf{Absorption}\\ \textbf{Spectrum Shape}} & \textbf{Paint QE} & \makecell{\textbf{Visual}\\ \textbf{Opacity}} \\
\textbf{N$_{Gaus}$} & \textbf{Avg. $\chi^2$} & \textbf{r$\chi^2$} & \textbf{BIC} & \textbf{AIC} & \textbf{$\sum$Q$_{i_{ch}}$} & \textbf{$\sum$Q$_{i_{m}}$} & \textbf{$\chi^2_{max}$} & \makecell{\textbf{`Good'}\\ \textbf{Fits}} \\
\hline
8 & 584.9 & 2.417 & 255.4 & 223.2 & \textbf{1.008} & 1.111 & 765 & 11 \\
9 & 470.4 & 1.960 & 204.6 & 172.4 & 1.009 & \textbf{1.021} & 521 & 58 \\
10 & 469.2 & 1.971 & 209.5 & 173.8 & \textbf{1.008} & 1.082 & 519 & 66 \\
11 & 445.1 & 1.886 & \textbf{201.4} & 162.2 & \textbf{1.008} & 1.040 & 503 & 67 \\
\textbf{12} & \textbf{436.3} & \textbf{1.864} & 201.7 & \textbf{159.0} & \textbf{1.008} & 1.026 & 501 & \textbf{75} \\
13 & 439.1 & 1.893 & 208.9 & 162.7 & 1.009 & 1.040 & 500 & 56 \\
\hline
\end{tabular}
\caption{\bf{Determination of N$_{Gaus}$ for $^3$He}}
\label{tab:3he_ngaus}
\end{table}

Examining Table ~\ref{tab:3he_ngaus} it is clear that no model had the best value in every category so some further analysis is required to select N$_{Gaus}$ for $^3$He. N$_{Gaus}$ = 12 has the best value in both r$\chi^2$ and AIC which are both important metrics. Examining the BIC it is seen that N$_{Gaus}$ = 12 and N$_{Gaus}$ = 11 have nearly identical BIC values. In fact $\Delta$BIC$<0.4$ which indicates a negligible preference between the models. All of the $\sum$Q$_{i_{ch}}$ values are fairly close offering little insight. The $\sum$Q$_{i_{m}}$ value for N$_{Gaus}$ = 12 is on the better end of the spectrum as well. N$_{Gaus}$ = 12 also had the most fits converge to be designated good fits with physical looking charge form factors. N$_{Gaus}$ = 12 further had the lowest average $\chi^2$ value, but this metric can be misleading as $\chi^2$ must always decrease as the number of parameters increases. However, because of the different R$_i$ configurations and averaging the $\chi^2$ results for the fits the average $\chi^2$ shown in the table does not always have to decrease with increasing N$_{Gaus}$. Upon reviewing these metrics, it is fairly clear the N$_{Gaus}$ = 12 is the best model to use for fitting the $^3$He data.

Table ~\ref{tab:3h_ngaus} mirrors Table ~\ref{tab:3he_ngaus} and shows the results of this model selection analysis for $^3$H. It is immediately obvious that the $\chi^2$ values are larger for $^3$H than they were for $^3$He. This is because the world data for $^3$H is less complete than that of $^3$He, especially at higher Q$^2$, and the quality of the data is not as good as that of $^3$He. The $\sum$Q$_i$ values are also further from unity with the magnetic charges being especially far off. Once again, this is a product of the dearth of high Q$^2$ and back angle data in the world data. If more $^3$H data could be obtained at high Q$^2$ and large back angles the reduced $\chi^2$ would likely decrease. The poor $\sum$Q$_{i_{m}}$ agreement with the expectation of unity also demonstrates the analysis value of not forcing the Q$_i$ to sum to unity which could hide the need for more high Q$^2$ and back angle data.

There are two entries for N$_{Gaus}$ = 8 labelled close and wide. These refer to the initial spacing of the R$_i$ values. For the close entry R$_0$ = 0.2-0.3, R$_{1-4}$ = 0.3-0.4, and R$_{5-7}$ = 0.5-0.6, and for the wide entry R$_0$ = 0.2-0.3, R$_{1-4}$ = 0.5-0.6, and R$_{5-7}$ = 0.8-0.9 as explained in Section ~\ref{ssec:radii}. This meant for the close R$_i$ the average starting R$_{max}$ = 3.3 fm and for the wide spacing R$_{max}$ = 5 fm which is what we expect from previous analyses ~\cite{Article:Amroun}. This test was done to see if the final fit results depended strongly on the starting R$_i$ spacing, or if the R$_i$ optimization produced consistent results with less reasonable initial R$_i$ values.

Fortunately, the results for the closer and wider R$_i$ spacings come out very similar indicating that the initial choice of R$_i$ does not significantly change the final result. This test had also previously been done for $^3$He with N$_{Gaus}$ = 10 with similar results to $^3$H with N$_{Gaus}$ = 8. The major difference between the initial spacings was that the closer, less reasonable, R$_i$ took longer for the R$_i$ optimization code to process. This was because the code had to check more values for each R$_i$ before finding similar optimal values to the larger initial R$_i$ spacings. The close R$_i$ also had fewer fits converge to be designated `good' fits indicating that more of the initial R$_i$ were unfavorable models than the wider spacings. We can conclude that if the initial R$_i$ distributions are off the final fit results should generally still be reliable, but they may take longer to process and have fewer fits converge to be deemed `good'.  

%\begin{table}[!h]
%\centering
%\begin{tabular}{|c c c c c c c c c|}
%\hline
%%\makecell{\textbf{Absorption}\\ \textbf{Spectrum Shape}} & \textbf{Paint QE} & \makecell{\textbf{Visual}\\ \textbf{Opacity}} \\
%\textbf{N$_{Gaus}$} & \textbf{Avg. $\chi^2$} & \textbf{r$\chi^2$} & \textbf{BIC} & \textbf{AIC} & \textbf{$\sum$Q$_{i_{ch}}$} & \textbf{$\sum$Q$_{i_{m}}$} & \textbf{$\chi^2_{max}$} & \makecell{\textbf{`Good'}\\ \textbf{Fits}} \\
%\hline
%7 & 611.690 & 2.79310 & \textbf{263.039} & 238.851 & \textbf{1.08373} & 1.32730 & 611.7 & 1\\
%8 close & 601.836 & 2.77344 & 264.694 & 237.051 & 1.09013 & 1.32859 & 603 & 32\\
%\textbf{8 wide} & 601.752 & 2.79892 & 264.661 & \textbf{237.018} & 1.08970 & 1.33270 & 603 & 39\\
%9 & 601.768 & 2.82579 & 270.123 & 239.025 & 1.08849 & 1.31982 & 604 & \textbf{95}\\
%10 & 601.893 & 2.84416 & 275.627 & 241.074 & 1.09248 & \textbf{1.29611} & 603 & 78\\
%11 & \textbf{600.750} & \textbf{2.77305} & 280.637 & 242.629 & 1.08699 & 1.34100 & 602 & 88\\
%\hline
%\end{tabular}
%\caption{\bf{Determination of N$_{Gaus}$ for $^3$H}}
%\label{tab:3h_ngaus}
%\end{table}

\vspace{6mm}
\begin{table}[!h]
\centering
\begin{tabular}{|c c c c c c c c c|}
\hline
%\makecell{\textbf{Absorption}\\ \textbf{Spectrum Shape}} & \textbf{Paint QE} & \makecell{\textbf{Visual}\\ \textbf{Opacity}} \\
\textbf{N$_{Gaus}$} & \textbf{Avg. $\chi^2$} & \textbf{r$\chi^2$} & \textbf{BIC} & \textbf{AIC} & \textbf{$\sum$Q$_{i_{ch}}$} & \textbf{$\sum$Q$_{i_{m}}$} & \textbf{$\chi^2_{max}$} & \makecell{\textbf{`Good'}\\ \textbf{Fits}} \\
\hline
7 & 611.7 & 2.793 & \textbf{263.0} & 238.9 & \textbf{1.084} & 1.327 & 611.7 & 1\\
8 close & 601.8 & \textbf{2.773} & 264.7 & 237.1 & 1.090 & 1.329 & 603 & 32\\
\textbf{8 wide} & 601.8 & 2.799 & 264.7 & \textbf{237.0} & 1.090 & 1.333 & 603 & 39\\
9 & 601.8 & 2.826 & 270.1 & 239.0 & 1.088 & 1.320 & 604 & \textbf{95}\\
10 & 601.9 & 2.844 & 275.6 & 241.0 & 1.092 & \textbf{1.296} & 603 & 78\\
11 & \textbf{600.8} & \textbf{2.773} & 280.6 & 242.6 & 1.087 & 1.341 & 602 & 88\\
\hline
\end{tabular}
\caption{\bf{Determination of N$_{Gaus}$ for $^3$H}}
\label{tab:3h_ngaus}
\end{table}

Again, the agreement between the metrics is not unanimous, and in fact it is even less clear than for $^3$He. Let us begin by examining the lowest BIC value for N$_{Gaus}$ = 7. The other metrics also look decent until one notices that only one fit met the standards for a `good' fit. This failure for the vast majority of fits to look physical indicates that N$_{Gaus}$ = 7 is probably not a good choice for the best model. Examining the lowest AIC value for the wider R$_i$ spacings and N$_{Gaus}$ = 8 the other metrics look acceptable with a reasonable number of fits converging to `goodness'. $\Delta$BIC for the wider R$_i$ spacings and N$_{Gaus}$ = 8 compared to N$_{Gaus}$ = 7 is only 1.6 indicating that there is little reason to prefer one model over the other. The higher Gaussian fits look reasonable as well, but the AIC, and especially BIC, grow significantly as N$_{Gaus}$ increases ruling out these fits. The closer spacing for N$_{Gaus}$ = 8 was eliminated for the reasons discussed above like taking longer to process and having fewer fits converge but yielding similar results to the wider spaced R$_i$ models. Accounting for all of this we select N$_{Gaus}$ = 8 with the wider initial R$_i$ spacings.

\subsection{$^3$He Fits}
\label{ssec:3he_fits}

Now that we have selected N$_{Gaus}$ = 12 for $^3$He we can run several hundred fits with pseudorandom starting R$_i$ values along with the R$_i$ optimization procedure. The initial spacing of the R$_i$ values for these fits was R$_0$ = 0.1-0.2, R$_{1-6}$ = 0.3-0.4, and R$_{7-11}$ = 0.5-0.6 as explained in Section ~\ref{ssec:radii}. A total of 1352 individual fits using the pseudorandom R$_i$ values were generated. Table ~\ref{tab:3he_fits} shows the results of these 1352 fits without any $\chi^2_{max}$ cut and with a $\chi^2_{max}$ = 500 in the same fashion as Tables ~\ref{tab:3he_ngaus} and ~\ref{tab:3h_ngaus}. This $\chi^2$ cut was determined by decreasing the value of the cut until all of the F$_{ch}$ form factors had nicely defined sharp first minima and the unphysical form factors were eliminated as discussed in Section ~\ref{ssec:ngaus}. 852 fits survive the $\chi^2$ cut of 500, and the remaining fits have charge form factors with the desired sharp minima.

%\vspace{6mm}
%\begin{table}[!h]
%\centering
%\begin{tabular}{|c c c c c c c c c|}
%\hline
%%\makecell{\textbf{Absorption}\\ \textbf{Spectrum Shape}} & \textbf{Paint QE} & \makecell{\textbf{Visual}\\ \textbf{Opacity}} \\
%\textbf{N$_{Gaus}$} & \textbf{Avg. $\chi^2$} & \textbf{r$\chi^2$} & \textbf{BIC} & \textbf{AIC} & \textbf{$\sum$Q$_{i_{ch}}$} & \textbf{$\sum$Q$_{i_{m}}$} & \textbf{$\chi^2_{max}$} & \makecell{\textbf{Below}\\ \textbf{Cut}} \\
%\hline
%12 & 523.743 & 2.23822 & 249.063 & 184.771 & 1.01018 & 1.04558 & No Cut & 1352\\
%12 & 436.564 & 1.86566 & 201.908 & 159.223 & 1.00840 & 1.02235 & 500 & 852\\
%\hline
%\end{tabular}
%\caption{\bf{Metrics for Final $^3$He Fits}}
%\label{tab:3he_fits}
%\end{table}

\vspace{6mm}
\begin{table}[!h]
\centering
\begin{tabular}{|c c c c c c c c c|}
\hline
%\makecell{\textbf{Absorption}\\ \textbf{Spectrum Shape}} & \textbf{Paint QE} & \makecell{\textbf{Visual}\\ \textbf{Opacity}} \\
\textbf{N$_{Gaus}$} & \textbf{Avg. $\chi^2$} & \textbf{r$\chi^2$} & \textbf{BIC} & \textbf{AIC} & \textbf{$\sum$Q$_{i_{ch}}$} & \textbf{$\sum$Q$_{i_{m}}$} & \textbf{$\chi^2_{max}$} & \makecell{\textbf{Below}\\ \textbf{Cut}} \\
\hline
12 & 523.7 & 2.238 & 249.1 & 184.8 & 1.010 & 1.046 & No Cut & 1352\\
12 & 436.6 & 1.866 & 201.9 & 159.2 & 1.008 & 1.022 & 500 & 852\\
\hline
\end{tabular}
\caption{\bf{Metrics for Final $^3$He Fits}}
\label{tab:3he_fits}
\end{table}

Table ~\ref{tab:3he_fits} shows that the $\chi^2_{max}$ cut improves all of the metrics as expected. Of the 1352 fits 852, 63$\%$, survive the $\chi^2_{max}$ cut showing that the fits are not struggling to converge. There are 259 data points for $^3$He resulting in a $\chi^2$ of 436.6. This works out to a $\chi^2$ of 1.686 per data point. This value indicates a reasonably good fit, but could be decreased if the individual data sets were given a floating normalization. This analysis chose not to apply a floating normalization so as to better represent the current state of the world data and its uncertainties and disagreements.%436.564

The $\sum$Q$_{i_{ch}}$ = 1.008 when we expect it to equal unity from physical considerations of the form factors discussed in Section ~\ref{sec:ffs}. This means the fits on average see 0.8$\%$ more electric charge than expected. This occurs because we did not force the form factors to approach unity at the origin. Again, we hope to represent the world data as is and use the $\sum$Q$_{i}$ to indicate the quality and completeness of the available data. If we forced the form factors to unity at the origin their slope would be artificially decreased in magnitude near zero. Still, $\sum$Q$_{i_{ch}}$ = 1.008 is close to one and indicates that the world data describes the charge form factor for $^3$He well. $\sum$Q$_{i_{m}}$ = 1.022 shows an excess of 2.2$\%$ in the magnetic charge which is worse than the electric charge due again to the lack of high Q$^2$ data. Even so, the $\sum$Q$_{i_{m}}$ for $^3$He from the fits seems to correspond decently well with our prior expectations. %1.00840, 1.02235

Now let's examine the fits from Table ~\ref{tab:3he_fits} visually. Figure ~\ref{fig:3he_fch} shows the resulting charge form factors of each of the 1352 fits of the $^3$He world data. Figure ~\ref{fig:3he_fch_no_cut} shows the 1352 fits without any $\chi^2_{max}$ cut, and Figure ~\ref{fig:3he_fch_cut} shows the 852 fits surviving a $\chi^2_{max}$ = 500 cut. Plotted along with this analysis' fits in red is a blue line representing the average fit result from \cite{Article:Amroun} in the range that analysis considered its fits to be valid. 

\begin{figure}[!ht]
\begin{subfigure}{1.\textwidth}
  \centering
  \includegraphics[width=1.1\linewidth]{Fch_3He_n12_1352.png}
  \caption{\bf{Charge Form Factors from 1352 $^3$He Fits with no $\chi^2_{max}$ cut.}}
  \label{fig:3he_fch_no_cut}
\end{subfigure}\\
\begin{subfigure}{1.\textwidth}
  \centering
  \includegraphics[width=1.1\linewidth]{Fch_3He_n12_852.png}
  \caption{\bf{Charge Form Factors from 852 $^3$He Fits surviving a $\chi^2_{max}$ = 500 cut.}}
  \label{fig:3he_fch_cut}
\end{subfigure}
\caption[\bf{$^3$He Charge Form Factors}] {
{\bf{$^3$He Charge Form Factors.}} These charge form factors (red lines) were derived from 1352 fits of the $^3$He world data using pseudorandom initial R$_i$ values and the R$_i$ optimization procedure. The blue line is the fit from \cite{Article:Amroun}.}
\label{fig:3he_fch}
\end{figure}

Examining the 1352 fits with no $\chi^2_{max}$ cut in Figure ~\ref{fig:3he_fch_no_cut} the first feature to notice is that many of the fits are behaving in an unphysical manner. We observe numerous dip only minima where the first sharp minima is expected. Numerous fits also have first or second minima located in the already well understood region of 20-30 fm$^{-2}$ where there are not expected to be minima. We want to eliminate these kinds of fits as we know them to be nonphysical from theory and prior measurements. 

To eliminate these fits we can impose a cut, $\chi^2_{max}$, which removes fits with higher $\chi^2$ as described in Section ~\ref{ssec:ngaus}. Removing higher $\chi^2$ fits will generally remove the worse fits first, hopefully eliminating the least physical looking results. A cut of $\chi^2_{max}$ = 500 on these 1352 fits results in 852 fits surviving and is shown in the plot in Figure ~\ref{fig:3he_fch_cut}. By lowing $\chi^2_{max}$ until the last of the dip only minima were removed we have eliminated the nonphysical fits leaving only the desired `good' fits. Now it is clear that our results for the $^3$He charge form factor are in excellent agreement with \cite{Article:Amroun}. No changes were expected in the charge form factor due to the addition of more high Q$^2$ data as the charge form factor is dominated by the low Q$^2$ data which was already robust. If the new high Q$^2$ data is going to change our understanding of a form factor it is far more likely to be the magnetic form factor.

Figure ~\ref{fig:3he_fm} shows the resulting magnetic form factors of each of the 1352 fits of the $^3$He world data. Figure ~\ref{fig:3he_fm_no_cut} shows the 1352 fits without any $\chi^2_{max}$ cut plotted together, and Figure ~\ref{fig:3he_fm_cut} shows the 852 fits surviving the $\chi^2_{max}$ = 500 cut. Beginning with the fits without a $\chi^2_{max}$ cut from Figure ~\ref{fig:3he_fm_no_cut} there are many unphysical fits again. Most of these are identified by the location of their first minima being unreasonable, or by the minima not being sharp. It is expected that the magnetic form factor fits will be less clean than the charge form factor fits due to the relative lack of high Q$^2$ data. This is also why the $\chi^2_{max}$ cut was applied to the charge form factor instead of the magnetic. These fits are again greatly cleaned up once the $\chi^2_{max}$ cut is applied.

\begin{figure}[!ht]
\begin{subfigure}{1.\textwidth}
  \centering
  \includegraphics[width=1.1\linewidth]{Fm_3He_n12_1352.png}
  \caption{\bf{Magnetic Form Factors from 1352 $^3$He Fits with no $\chi^2_{max}$ cut.}}
  \label{fig:3he_fm_no_cut}
\end{subfigure}\\
\begin{subfigure}{1.\textwidth}
  \centering
  \includegraphics[width=1.1\linewidth]{Fm_3He_n12_852.png}
  \caption{\bf{Magnetic Form Factors from 852 $^3$He Fits surviving a $\chi^2_{max}$ = 500 cut.}}
  \label{fig:3he_fm_cut}
\end{subfigure}
\caption[\bf{$^3$He Magnetic Form Factors}] {
{\bf{$^3$He Magnetic Form Factors.}} These Magnetic form factors (red lines) were derived from 1352 fits of the $^3$He world data using pseudorandom initial R$_i$ values and the R$_i$ optimization procedure. The blue line is the fit from \cite{Article:Amroun}.}
\label{fig:3he_fm}
\end{figure}

Figure ~\ref{fig:3he_fm_cut} shows the resulting magnetic form factors of each of the 852 fits of the $^3$He world data that survive the $\chi^2_{max}$ = 500 cut plotted together. Immediately it is obvious that the cut has removed most of the nonphysical and bizarre looking results. However, unlike the charge form factor we are no longer in excellent agreement with the fit from \cite{Article:Amroun}. This is to be expected because the new data points from \cite{Article:Alex} and the point from this analysis add many new high Q$^2$ data points to the world data. The magnetic form factor is mostly influenced by these high Q$^2$ points unlike the charge form factor which is more influenced by low Q$^2$ points. So it is unsurprising that the charge form factor remains the same while the magnetic form factor evolved with the addition of new data. Indeed, the new fits with the new datasets added indicate that the first diffractive minima for the $^3$He magnetic form factor is a few femtometers higher in Q$^2$ than previously predicted. The consensus of fits places this minima at 19-20 fm as opposed to the previous location of 18 fm.  

Now that we have functions describing the form factors we can calculate the charge densities as discussed in Section ~\ref{sec:sog}. Figure ~\ref{fig:3he_charge_density} shows the charge densities for $^3$He prior to the $\chi^2_{max}$ cut's application in ~\ref{fig:3he_charge_density_no_cut} as well as the charge density for $^3$He with a $\chi^2_{max}$ cut = 500 in ~\ref{fig:3he_charge_density_cut}. Without the cut the charge density shapes are quite variable at small radii and vary widely in magnitude. However, once the nonphysical fits are removed by the $\chi^2_{max}$ cut the only charge densities remaining are extremely consistent in shape and magnitude. Below a radius of about 0.5 fm the charge density comes to a relatively stable plateau. We also note that the charge density tapers off by a radii of 5 fm justifying our guess of R$_{max}$ in Section ~\ref{ssec:radii}. %If we had artificially forced the Q$_i$ parameters to sum to one these charge densities would need to turn down slightly near the origin instead of showing the plateauing behavior we observe.

\begin{figure}[!ht]
\begin{subfigure}{1.\textwidth}
  \centering
  \includegraphics[width=1.1\linewidth]{Charge_Density_3He_n12_1352.png}
  \caption{Charge densities of 1352 $^3$He fits with no $\chi^2_{max}$ cut.}
  \label{fig:3he_charge_density_no_cut}
\end{subfigure}\\
\begin{subfigure}{1.\textwidth}
  \centering
  \includegraphics[width=1.1\linewidth]{Charge_Density_3He_n12_852.png}
  \caption{Charge densities of 852 $^3$He fits surviving a $\chi^2_{max}$ = 500 cut.}
  \label{fig:3he_charge_density_cut}
\end{subfigure}
\caption[\bf{$^3$He Charge Densities}]{\bf{$^3$He Charge Densities.}}
\label{fig:3he_charge_density}
\end{figure}

Using Equation ~\ref{eq:rms_derivative} we can also calculate the RMS charge radii for each of the individual fits. Figure ~\ref{fig:3he_rms_deriv} shows a plot of the 1352 charge radii resulting from the fits of $^3$He world data. Figure ~\ref{fig:3he_rms_deriv_no_cut} shows the 1352 fits without a $\chi^2_{max}$ cut, and ~\ref{fig:3he_rms_deriv_cut} shows the 852 fits surviving the $\chi^2_{max}$ = 500 cut. Notice that after the cut is applied the second peak at higher radii disappears indicating that it was nonphysical form factors that were yielding these larger radii. The average of all the RMS charge radii surviving the fit was 1.90 fm. A Gaussian fit to these radii finds a mean of 1.90 fm with a standard deviation of 0.00144 fm. We see that the radii are well grouped and form a reasonable semblance of a Gaussian distribution. %1.90086, 1.90102, 0.0014419.

\begin{figure}[!ht]
\begin{subfigure}{1.\textwidth}
  \centering
  \includegraphics[width=1.1\linewidth]{RMS_deriv_3He_n12_1352.png}
  \caption{RMS charge radii of 1352 $^3$He fits with no $\chi^2_{max}$ cut.}
  \label{fig:3he_rms_deriv_no_cut}
\end{subfigure}\\
\begin{subfigure}{1.\textwidth}
  \centering
  \includegraphics[width=1.1\linewidth]{RMS_deriv_3He_n12_852.png}
  \caption{RMS charge radii of 852 $^3$He fits surviving a $\chi^2_{max}$ = 500 cut.}
  \label{fig:3he_rms_deriv_cut}
\end{subfigure}
\caption[\bf{$^3$He RMS Charge Radii}]{\bf{$^3$He RMS Charge Radii.}}
\label{fig:3he_rms_deriv}
\end{figure}

The parameters of the $^3$He fits can be plotted as well for further information. Figure ~\ref{fig:3he_qi} shows the Q$_i$ fit parameters discussed in ~\ref{sec:sog} with Figure ~\ref{fig:3he_qch} showing the $Q_{i_{ch}}$ parameters and Figure ~\ref{fig:3he_qm} showing the $Q_{i_{m}}$ parameters. Each of the 12 parameters is shown up to a maximum value of 0.5. The structure of the parameter plots did not vary significantly after the $\chi^2_{max}$ cut was applied so only the plots with the cut are shown. One important thing to notice is that the Q$_i$ values are larger for the inner Gaussians placed at smaller radii. The inner Gaussians have far more influence over the fit than the outer ones which makes since as most of the structure of the charge density is located at smaller radii as well. A small change in Q$_{i_{1-3}}$ can have drastic effects on the form factors, but small changes to the higher Q$_i$ have almost no influence on the form factors. 

Also note that the inner Q$_i$ parameters have a much wider spread in values than the outer Q$_i$. This is because the fits are extremely sensitive to the final combination of R$_i$ selected which flow from the pseudorandom R$_i$ initially generated for each fit. This means that the Q$_i$ values vary widely from fit to fit as seen in the large spread of the inner parameters. As such, attributing an uncertainty to each parameter is fairly meaningless since each set of parameters is representative only of the specific `model' created by the R$_i$ of that single fit. It would be satisfying if the average value of each Q$_i$ and R$_i$ could be found and then used to plot an average set of form factors. However, due to the strong correlation between the Q$_i$ and R$_i$ when this was attempted the fit was very poor and nonphysical in nature.

\begin{figure}[!ht]
\begin{subfigure}{1.\textwidth}
  \centering
  \includegraphics[width=1.1\linewidth]{Qich_3He_n12_852.png}
  \caption{$Q_{i_{ch}}$ parameters for $^3$He with $\chi^2_{max}$ = 500.}
  \label{fig:3he_qch}
\end{subfigure}\\
\begin{subfigure}{1.\textwidth}
  \centering
  \includegraphics[width=1.1\linewidth]{Qim_3He_n12_852.png}
  \caption{$Q_{i_{m}}$ parameters for $^3$He with $\chi^2_{max}$ = 500.}
  \label{fig:3he_qm}
\end{subfigure}
\caption[\bf{$^3$He Q$_i$ Fit Parameters}]{\bf{$^3$He Q$_i$ Fit Parameters.}}
\label{fig:3he_qi}
\end{figure}

Plotted in Figure ~\ref{fig:3he_ri} are the R$_i$ values (~\ref{fig:3he_ri_ind}) and the separation between consecutive R$_i$ values (~\ref{fig:3he_ri_sep}) of the fits of the $^3$He world data in fm. Little difference was made after the $\chi^2_{max}$ cut was applied so only the final 852 fits surviving the cut were plotted. The individual R$_i$ are seen to be fairly well grouped around the same values. This is especially true for the lower radii which have more influence over the fits than the large radii due to their larger Q$_i$ values. The consistent positioning of the radii after the optimization procedure described in ~\ref{ssec:radii} demonstrates that the pseudorandom R$_i$ generated tend to converge to consistent values and not wander randomly. 

We also see that our prior that R$_{max}$ will be $\approx$ 5 fm agrees with the largest R$_i$ after the fitting procedure. Finally, we notice that the distance between consecutive R$_i$ values does fluctuate somewhat, but the separations still tend to converge to a central value, particularly at the lower radii. We had previously guessed that the separation between consecutive R$_i$ would roughly double for R$_i$ $>$ R$_{max}$/2 which is also consistent with these results.

\begin{figure}[!ht]
\begin{subfigure}{1.\textwidth}
  \centering
  \includegraphics[width=1.1\linewidth]{Ri_3He_n12_852.png}
  \caption{R$_i$ values (fm) of fits of $^3$He world data $\chi^2_{max}$ = 500.}
  \label{fig:3he_ri_ind}
\end{subfigure}\\
\begin{subfigure}{1.\textwidth}
  \centering
  \includegraphics[width=1.1\linewidth]{Ri_Sep_3He_n12_852.png}
  \caption{Separation of consecutive R$_i$ values (fm) of fits of $^3$He world data $\chi^2_{max}$ = 500.}
  \label{fig:3he_ri_sep}
\end{subfigure}
\caption[\bf{$^3$He RMS Charge Radii}]{\bf{$^3$He RMS Charge Radii.}}
\label{fig:3he_ri}
\end{figure}

The set of new world data fits for $^3$He have essentially created an error band for the form factors by spanning the set of R$_i$ models. To make these new fits applicable a specific parametrization of the form factors is required. To select this representative fit for $^3$He a single fit at the middle of the distribution of all new fits, with physical characteristics and reasonable values for the metrics listed in Table ~\ref{tab:3he_ngaus}, was selected. Table ~\ref{tab:3he_rep_fit_pars} and Table ~\ref{tab:3he_rep_fit_stats} show the parameters for the representative fit of the $^3$He world data along with the metrics used to test the `goodness' of the fit respectively.

%\vspace{6mm}
\begin{table}[!h]
\centering
\begin{tabular}{|c c l l|}
\hline
%\makecell{\textbf{Absorption}\\ \textbf{Spectrum Shape}} & \textbf{Paint QE} & \makecell{\textbf{Visual}\\ \textbf{Opacity}} \\
\makecell{\textbf{Parameter}\\ \textbf{Number}} & \textbf{R$_i$ (fm)} & \textbf{Q$_{i_{ch}}$} & \textbf{Q$_{i_{m}}$}\\
\hline
1 & 0.3 & 0.0996392 & 0.159649 \\
2 & 0.7 & 0.214304 & 0.0316168 \\
3 & 0.9 & 0.0199385 & 0.277843 \\
4 & 1.1 & 0.195676 & 0.0364955 \\
5 & 1.5 & 0.0785533 & 0.0329718 \\
6 & 1.6 & 0.167223 & 0.233469 \\
7 & 2.2 & 0.126926 & 0.117059 \\
8 & 2.7 & 0.0549379 & 0.0581085 \\
9 & 3.3 & 0.0401401 & 0.0485212 \\
10 & 4.2 & 0.0100803 & 1.77602e-12 \\
11 & 4.3 & 0.0007217 & 0.0240927 \\
12 & 4.8 & 4.98962e-12 & 8.94934e-12 \\           
\hline
\end{tabular}
\caption{\bf{Parameters for $^3$He World Data Representative Fit}}
\label{tab:3he_rep_fit_pars}
\end{table}

%\vspace{6mm}
%\begin{table}[!h]
%\centering
%\begin{tabular}{|c c c c c c c|}
%\hline
%%\makecell{\textbf{Absorption}\\ \textbf{Spectrum Shape}} & \textbf{Paint QE} & \makecell{\textbf{Visual}\\ \textbf{Opacity}} \\
%\textbf{N$_{Gaus}$} & \textbf{Avg. $\chi^2$} & \textbf{r$\chi^2$} & \textbf{BIC} & \textbf{AIC} & \textbf{$\sum$Q$_{i_{ch}}$} & \textbf{$\sum$Q$_{i_{m}}$}\\
%\hline
%12 & 436.097 & 1.86366 & 268.312 & 158.948 & 1.00814 & 1.01983\\
%\hline
%\end{tabular}
%\caption{\bf{Metrics for $^3$He World Data Representative Fit}}
%\label{tab:3he_rep_fit_stats}
%\end{table}

\begin{table}[!h]
\centering
\begin{tabular}{|c c c c c c c|}
\hline
%\makecell{\textbf{Absorption}\\ \textbf{Spectrum Shape}} & \textbf{Paint QE} & \makecell{\textbf{Visual}\\ \textbf{Opacity}} \\
\textbf{N$_{Gaus}$} & \textbf{Avg. $\chi^2$} & \textbf{r$\chi^2$} & \textbf{BIC} & \textbf{AIC} & \textbf{$\sum$Q$_{i_{ch}}$} & \textbf{$\sum$Q$_{i_{m}}$}\\
\hline
12 & 436.1 & 1.864 & 268.3 & 158.9 & 1.008 & 1.020\\
\hline
\end{tabular}
\caption{\bf{Metrics for $^3$He World Data Representative Fit}}
\label{tab:3he_rep_fit_stats}
\end{table}

Figure ~\ref{fig:3he_rep_fit} shows plots of the representative charge and magnetic form factors for the new $^3$He world data fits. The charge form factor error band is very well determined until Q$^2$ $\approx$ 55 fm$^{-2}$ where it begins to expand. The magnetic form factor is very well determined below  Q$^2$ $\approx$ 12 fm$^{-2}$. As F$_m$ approaches its first minimum the error band expands considerably producing a range of about 4 fm$^{-2}$ in Q$^2$ where the diffractive minimum is located. The error band then tightens again between 22 and 36 fm$^{-2}$ with the addition of the new $^3$He data from JLab and this analysis. The error band then expands again as a possible second diffractive minimum is approached due to leaving the Q$^2$ range in which robust experimental data exists. 

\begin{figure}[!ht]
\begin{subfigure}{1.\textwidth}
  \centering
  \includegraphics[width=1.\linewidth]{3He_Fch_Rep_Fit.png}
  \caption{\bf{$^3$He Charge Form Factor for the New Representative Fit of World Data.}}
  \label{fig:3he_fch_rep_fit}
\end{subfigure}\\
\begin{subfigure}{1.\textwidth}
  \centering
  \includegraphics[width=1.\linewidth]{3He_Fm_Rep_Fit.png}
  \caption{\bf{$^3$He Magnetic Form Factor for the New Representative Fit of World Data.}}
  \label{fig:3he_fm_rep_fit}
\end{subfigure}
\caption[\bf{$^3$He Form Factors for the New Representative Fit of World Data}] {
{\bf{$^3$He Form Factors for the New Representative Fit of World Data.}} The selected representative fit is shown in black and its error band is shaded in red.}
\label{fig:3he_rep_fit}
\end{figure}

This representative fit can be examined in even greater detail by comparing it to the world data distribution seen in Figure ~\ref{fig:3he_data_distribution}. Again, we find the comparison with the fit from ~\cite{Article:Amroun} to be in strong agreement for F$_{ch}$. We also see that the first minimum of F$_m$ has shifted up in Q$^2$. Looking at the distribution of the world data it becomes clear that most of the data exists below Q$^2$ $\approx$ 12 fm$^{-2}$ and above that value it becomes quite sparse. This analysis has previously discussed the need for more high Q$^2$ data, and Figure ~\ref{fig:3he_data_distribution} reinforces how the dearth of this data is restricting our understanding of the $^3$He form factors.

\begin{figure}[!ht]
	\begin{center}
	\includegraphics[width=1.0\linewidth]{3He_Data_Distribution.png}
	\end{center}
	\caption[\bf{$^3$He Representative Form Factors and World Data Distribution}]{
	{\bf{$^3$He Representative Form Factors and World Data Distribution.}} The top two plots show the $^3$He F$_{ch}$ and F$_m$ versus Q$^2$, and the lowest plot shows the $^3$He world data distribution in Q$^2$ used in this analysis.}
	\label{fig:3he_data_distribution}
\end{figure}

This representative fit has a total $\chi^2$ of 436.1 and a reduced $\chi^2$ of 1.864 for 259 $^3$He elastic cross sections. Recall that this analysis did not float the normalization of the different datasets leading to a slightly larger $\chi^2$. When examining how well the representative fit is describing the data it can be helpful to look at the $\chi^2$ values for each data individual point in each dataset as shown in Figure ~\ref{fig:3he_rep_fit_chi2_q2}. This plot indicates that much of the total $\chi^2$ comes from the older datasets found in Collard and Szalata. The point from this analysis (orange) has a $\chi^2$ of $\approx$ 4. 

\begin{figure}[!ht]
	\begin{center}
	\includegraphics[width=1.0\linewidth]{3He_Rep_Chi2_vs_Q2.png}
	\end{center}
	\caption[\bf{$^3$He Representative Fit $\chi^2$ vs. Q$^2$}]{
	{\bf{$^3$He Representative Fit $\chi^2$ vs. Q$^2$.}} }
	\label{fig:3he_data_distribution}
\end{figure}

It can also be illuminating to look at the residual of the fit for each data point in each dataset as shown in Figure ~\ref{fig:3he_rep_fit_residual}. Note that since the plot encompasses many orders of magnitude the residual is plotted as Equation ~\ref{eq:residual}, where $\sigma_{exp}$ is the experimental cross section measured and $\sigma_{fit}$ is the cross section predicted by the fit at those same kinematics. Firstly, there are two heavy outliers from Arnold, but this is not very surprising since these points were taken at very high Q$^2$. Zooming in on the remaining points they mostly look to be normally distributed about zero, although the points from Collard may lie a little high as do some of the high Q$^2$ points from Amroun. This indicates that floating the normalizations of the different datasets would not have improved the overall $\chi^2$ significantly. We also find that the point from this analysis lies below the representative fit.

\begin{equation} \label{eq:residual}
	Residual = \frac{\sigma_{exp}-\sigma_{fit}}{\sigma_{fit}}
\end{equation}

\begin{figure}[!ht]
\begin{subfigure}{1.\textwidth}
  \centering
  \includegraphics[width=1.\linewidth]{3He_Rep_Residual.png}
  \caption{\bf{No Zoom.}}
  \label{fig:3he_fch_rep_fit}
\end{subfigure}\\
\begin{subfigure}{1.\textwidth}
  \centering
  \includegraphics[width=1.\linewidth]{3He_Rep_Residual_Zoom.png}
  \caption{\bf{Zoomed In.}}
  \label{fig:3he_fm_rep_fit}
\end{subfigure}
\caption[\bf{$^3$He Representative Fit Residual vs. Q$^2$}] {
{\bf{$^3$He Representative Fit Residual vs. Q$^2$.}} The upper plot shows a view of each data point's residual, and the lower plot shows a zoomed in version without the two high residual points for clarity.}
\label{fig:3he_rep_fit_residual}
\end{figure}

\subsection{Revisiting the E08-014 Cross Section}
\label{ssec:xs_revisit}

Now that we have obtained a new representative fit for the expanded set of world data it is prudent to revisit the cross section extracted in Chapter ~\ref{ch:xs}. To find the elastic cross section for $^3$He presented earlier the elastic electrons were simulated with the Monte Carlo SIMC. This Monte Carlo had a previous fit for the $^3$He cross section from ~\cite{Article:Amroun}. We now have a fit that incorporates more high Q$^2$ data in the region of this analysis' cross section measurement. Let us now replace the previous fit in SIMC with our new fit and recalculate the cross section.

Once again the cross section in SIMC will be scaled by a constant value until the elastic electron yield from simulation matches that of the experiment. SIMC will then produce the average cross section for our $^3$He data point. Since the shape of the two fits should be similar, and the cross section clearly does not change based on the fit used, we expect to find a very similar cross section value with the new fit compared to the value from the previous fit. Matching the electron yields requires scaling the SIMC cross section by a constant value of C$_{SIMC}$ = 1.23197. This is a larger scale factor than previously and is due to the decreased magnitude of F$_m$ in the new fit at this point's Q$^2$ of 34.19 fm$^{-2}$. The cross section yielded by SIMC after the yields are matched is found to be 1.345 * 10$^{-6}$ $\pm$ 0.086 $\mu$b/sr. This is in extremely close agreement (better than 1$\%$) with the previous fit's cross section of 1.335 $\pm$ 0.086 $\mu$b/sr as we expected.

\subsection{$^3$H Fits}
\label{ssec:3h_fits}

Now that we have selected N$_{Gaus}$ = 8 for $^3$H we can run several hundred fits with the pseudorandom starting R$_i$ values along with the R$_i$ optimization procedure. The initial spacing of the R$_i$ values for these fits was R$_0$ = 0.2-0.3, R$_{1-4}$ = 0.5-0.6, and R$_{5-7}$ = 0.8-0.9 as explained in Section ~\ref{ssec:radii}. A total of 2600 individual fits using the pseudorandom R$_i$ values were generated. Table ~\ref{tab:3he_fits} shows the results of these 2600 fits without any $\chi^2_{max}$ cut and with a $\chi^2$ maximum of 603 in the same fashion as Tables ~\ref{tab:3he_ngaus} and ~\ref{tab:3h_ngaus}. 908 fits survive the $\chi^2$ cut of 603, and the remaining fits have charge form factors with the desired sharp minima.

%\vspace{6mm}
%\begin{table}[!h]
%\centering
%\begin{tabular}{|c c c c c c c c c|}
%\hline
%%\makecell{\textbf{Absorption}\\ \textbf{Spectrum Shape}} & \textbf{Paint QE} & \makecell{\textbf{Visual}\\ \textbf{Opacity}} \\
%\textbf{N$_{Gaus}$} & \textbf{Avg. $\chi^2$} & \textbf{r$\chi^2$} & \textbf{BIC} & \textbf{AIC} & \textbf{$\sum$Q$_{i_{ch}}$} & \textbf{$\sum$Q$_{i_{m}}$} & \textbf{$\chi^2_{max}$} & \makecell{\textbf{Below}\\ \textbf{Cut}} \\
%\hline
%8 & 611.385 & 2.81744 & 266.175 & 238.532 & 1.08866 & 1.33481 & No Cut & 2600\\
%8 & 601.840 & 2.77346 & 264.695 & 237.053 & 1.08991 & 1.32926 & 603 & 908\\
%\hline
%\end{tabular}
%\caption{\bf{Metrics for Final $^3$H Fits}}
%\label{tab:3h_fits}
%\end{table}

\vspace{6mm}
\begin{table}[!h]
\centering
\begin{tabular}{|c c c c c c c c c|}
\hline
%\makecell{\textbf{Absorption}\\ \textbf{Spectrum Shape}} & \textbf{Paint QE} & \makecell{\textbf{Visual}\\ \textbf{Opacity}} \\
\textbf{N$_{Gaus}$} & \textbf{Avg. $\chi^2$} & \textbf{r$\chi^2$} & \textbf{BIC} & \textbf{AIC} & \textbf{$\sum$Q$_{i_{ch}}$} & \textbf{$\sum$Q$_{i_{m}}$} & \textbf{$\chi^2_{max}$} & \makecell{\textbf{Below}\\ \textbf{Cut}} \\
\hline
8 & 611.4 & 2.817 & 266.2 & 238.5 & 1.089 & 1.335 & No Cut & 2600\\
8 & 601.8 & 2.773 & 264.7 & 237.1 & 1.090 & 1.329 & 603 & 908\\
\hline
\end{tabular}
\caption{\bf{Metrics for Final $^3$H Fits}}
\label{tab:3h_fits}
\end{table}

Table ~\ref{tab:3h_fits} shows that the $\chi^2_{max}$ cut again improves all of the metrics as expected. Of the 2600 fits 908, 35$\%$, survive the $\chi^2_{max}$ cut. This shows that the fits have more difficulty converging than they did for $^3$He, but they still converge with enough regularity to get a useful and reliable sample of fits. There are 234 data points for $^3$H resulting in a $\chi^2$ of 601.8. This works out to a $\chi^2$ of 2.572 per data point. This value is significantly higher than the value for $^3$He of 1.686 implying that the data for $^3$H is not as robust. Later in this section we will see that much of this larger $\chi^2$ is due to disagreement between the various experimental measurements. If the individual data sets were given a floating normalization the $\chi^2$ would lower significantly, but this would mask the actual state of the $^3$H world data. The fits still look very reasonable and agree well with past results.%601.840,

The $\sum$Q$_{i_{ch}}$ = 1.090 when we expect it to equal unity from physical considerations of the form factors discussed in Section ~\ref{sec:ffs}. This means the fits on average see 9$\%$ more electric charge than expected. Again, this occurs because we did not force the form factors to approach unity at the origin, but this demonstrates our relatively worse understanding of the charge form factor for $^3$H compared to $^3$He. If we forced the form factors to unity at the origin they would artificially turn over even more at small radii. $\sum$Q$_{i_{m}}$ = 1.329 shows an excess of 32.9$\%$ in the magnetic charge which is much worse than the charge form factor for $^3$H or either form factor for $^3$He. Now we see the analytical use of not forcing the Q$_i$ to sum to unity artificially. If we had done this it would not be nearly as clear that most of our uncertainty about the $^3$H form factors comes from our lack of understanding of the magnetic form factor. This continues to stress the need for more high Q$^2$ and back angle data for $^3$H to improve our understanding of the form factors.%1.08991, 1.32926

Figure ~\ref{fig:3h_fch} shows the charge form factors from each of the 2600 fits from Table ~\ref{tab:3h_fits}. Figure ~\ref{fig:3h_fch_no_cut} shows the 2600 fits without any $\chi^2_{max}$ cut plotted together, and Figure ~\ref{fig:3h_fch_cut} shows the 908 fits surviving a $\chi^2_{max}$ = 603 cut. Plotted along with this analysis' fits in red is a blue line representing the average result from \cite{Article:Amroun} in the range that analysis considered its fits to be valid. It is important to note that no new data has been added for $^3$H so we do not expect our results to diverge much from past results like \cite{Article:Amroun}. This makes the $^3$H results more useful to check the consistency of our methodology with past results and as a point of comparison to the $^3$He results. It was initially hoped that new $^3$H data would be available for this analysis from the planned experiment E12-14-009 \cite{3h_proposal} which was the original basis for this thesis. Unfortunately, due to budget uncertainty and continuing resolutions this experiment was cancelled. Happily, Hall A was able to take some elastic $^3$H data in the end, and once this data has been analyzed it should be a simple matter to add it to this analysis.

\begin{figure}[!ht]
\begin{subfigure}{1.\textwidth}
  \centering
  \includegraphics[width=1.1\linewidth]{Fch_3H_n8_2600.png}
  \caption{\bf{Charge Form Factors from 2600 $^3$H Fits with no $\chi^2_{max}$ cut.}}
  \label{fig:3h_fch_no_cut}
\end{subfigure}\\
\begin{subfigure}{1.\textwidth}
  \centering
  \includegraphics[width=1.1\linewidth]{Fch_3H_n8_908.png}
  \caption{\bf{Charge Form Factors from 908 $^3$H Fits surviving a $\chi^2_{max}$ = 603 cut.}}
  \label{fig:3h_fch_cut}
\end{subfigure}
\caption[\bf{$^3$H Charge Form Factors}] {
{\bf{$^3$H Charge Form Factors.}} These charge form factors (red lines) were derived from 2600 fits of the $^3$He world data using pseudorandom initial R$_i$ values and the R$_i$ optimization procedure. The blue line is the fit from \cite{Article:Amroun}.}
\label{fig:3h_fch}
\end{figure}

Looking at Figure ~\ref{fig:3h_fch_no_cut} we notice that many of the fits are again behaving in a nonphysical manner. We observe numerous dip only minima where the first sharp minimum is expected. These are even more shallow than the $^3$He dips and often almost miss the first minima entirely. There are even two fits that seemingly failed to converge at all and stand out wildly from the others. The location of the first minima look to be in good agreement with previous measurements even without the nonphysical fits being removed. The second minima, however, is barely pinned down at all. It seems as though it could be located anywhere from Q$^2$ = 26 - 60+ fm$^{-2}$ with a denser grouping around 30 fm$^{-2}$. Some of the fits indicate the presence of a second minimum at much higher Q$^2$ based on the current data. 

Now we apply the $\chi^2_{max}$ = 603 cut, and get Figure ~\ref{fig:3h_fch_cut} with the resulting charge form factors that survive. Immediately it is obvious that the cut has removed most of the nonphysical and bizarre looking results. The results remain in reasonable agreement with \cite{Article:Amroun} except the peak after the first minima is higher and falls off faster. The cluster of second minima near 30 fm$^{-2}$ has almost entirely disappeared. There is still no determination of the location of a secondary diffractive minima with the current data indicating it could be located anywhere from  Q$^2$ = 30 - 60+ fm$^{-2}$ or not exist at all. There is simply not enough high precision data in the world data to make any determinations beyond Q$^2$ $\approx$ 25 fm$^{-2}$.

Figure ~\ref{fig:3h_fm} shows the resulting magnetic form factors of each of the 2600 fits of the $^3$H world data. Figure ~\ref{fig:3h_fm_no_cut} shows the 2600 fits without any $\chi^2_{max}$ cut plotted together, and Figure ~\ref{fig:3h_fm_cut} shows the 908 fits surviving the $\chi^2_{max}$ = 603 cut. Beginning with the fits without a $\chi^2_{max}$ cut in Figure ~\ref{fig:3h_fm_no_cut} we see that as with the charge form factor there are many nonphysical fits. Most of these are dip only minima. The fits are decently consistent overall, and the location of the first minima is consistent with past results. 

\begin{figure}[!ht]
\begin{subfigure}{1.\textwidth}
  \centering
  \includegraphics[width=1.1\linewidth]{Fm_3H_n8_2600.png}
  \caption{\bf{Magnetic Form Factors from 2600 $^3$H Fits with no $\chi^2_{max}$ cut.}}
  \label{fig:3h_fm_no_cut}
\end{subfigure}\\
\begin{subfigure}{1.\textwidth}
  \centering
  \includegraphics[width=1.1\linewidth]{Fm_3H_n8_908.png}
  \caption{\bf{Magnetic Form Factors from 908 $^3$H Fits surviving a $\chi^2_{max}$ = 603 cut.}}
  \label{fig:3h_fm_cut}
\end{subfigure}
\caption[\bf{$^3$H Magnetic Form Factors}] {
{\bf{$^3$H Magnetic Form Factors.}} These Magnetic form factors (red lines) were derived from 2600 fits of the $^3$He world data using pseudorandom initial R$_i$ values and the R$_i$ optimization procedure. The blue line is the fit from \cite{Article:Amroun}.}
\label{fig:3h_fm}
\end{figure}

Next we examine the magnetic form factors of the 908 fits of the $^3$H world data that survive the $\chi^2_{max}$ = 603 cut shown in Figure ~\ref{fig:3h_fm_cut}. Once again, the cut has removed most of the nonphysical and bizarre looking results. After this cut our results still remain in agreement with \cite{Article:Amroun}. This is as we expect since no new $^3$H data was added to the world data. The fits indicate that a secondary diffractive minima could be anywhere from Q$^2$ = 36 - 60+ fm$^{-2}$ and offer little insight beyond this range. Again, the magnetic form factor is less well determined than the charge form factor fits due to the relative lack of high Q$^2$ and back angle data, and collecting more of this data is the best way to improve our knowledge of the $^3$H magnetic form factor. 

Using the functions describing the form factors we can calculate the charge densities as discussed in Section ~\ref{sec:sog}. Figure ~\ref{fig:3h_charge_density} shows the charge densities for $^3$H prior to the $\chi^2_{max}$ cut's application in ~\ref{fig:3h_charge_density_no_cut} as well as the charge density for $^3$H with a $\chi^2_{max}$ cut = 603 in ~\ref{fig:3h_charge_density_cut}. Without the cut the charge density appears to be bundled in two different groups, one predicting a higher density at a radius of zero and one predicting a lower density at zero. There is even a single fit which failed to converge well centered around 2 fm. Each of these groups has a fairly wide spread of magnitudes at low radii. However, once the nonphysical fits are removed by the $\chi^2_{max}$ cut the higher bundle of fits disappears leaving only the lower. 

\begin{figure}[!ht]
\begin{subfigure}{1.\textwidth}
  \centering
  \includegraphics[width=1.1\linewidth]{Charge_Density_3H_n8_2600.png}
  \caption{Charge densities of 2600 $^3$H fits with no $\chi^2_{max}$ cut.}
  \label{fig:3h_charge_density_no_cut}
\end{subfigure}\\
\begin{subfigure}{1.\textwidth}
  \centering
  \includegraphics[width=1.1\linewidth]{Charge_Density_3H_n8_908.png}
  \caption{Charge densities of 908 $^3$H fits surviving a $\chi^2_{max}$ = 603 cut.}
  \label{fig:3h_charge_density_cut}
\end{subfigure}
\caption[\bf{$^3$H Charge Densities}]{\bf{$^3$H Charge Densities.}}
\label{fig:3h_charge_density}
\end{figure}

The shapes of the $^3$H charge densities of the `good' fits are fairly consistent, but where they reach the origin still has a much wider spread than $^3$He's charge density. Once again this is due to the dearth of data for $^3$H, especially at higher kinematics. Below a radius of about 0.5 fm the charge density turns over and comes to to a plateau. This turnover occurs at smaller radii for fits with larger charge density at low radii. Again, the charge density tapers off by a radii of 5 fm justifying the guess of R$_{max}$ in Section ~\ref{ssec:radii}. %If we had artificially forced the Q$_i$ parameters to sum to one these charge densities would need to turn down significantly near the origin instead of showing the plateauing behavior we observe.

Using Equation ~\ref{eq:rms_derivative} we can also calculate the RMS charge radii for each of the individual fits. Figure ~\ref{fig:3h_rms_deriv} shows a plot of the 2600 charge radii resulting from the fits of $^3$H world data. Figure ~\ref{fig:3h_rms_deriv_no_cut} shows the 2600 fits without a $\chi^2_{max}$ cut, and ~\ref{fig:3h_rms_deriv_cut} shows the 908 fits surviving the $\chi^2_{max}$ = 603 cut. The cut removes some of the radii at the edges of the distribution, but it does not significantly alter its overall shape. The average of all the RMS charge radii surviving the cut was 2.020 fm. A Gaussian fit to these radii finds a mean of 2.019 fm with a standard deviation of 0.0132 fm. The $^3$H radii are less well grouped than the $^3$He radii and form a less well defined Gaussian distribution, but they still appear to be clustered roughly around a mean value without errant peaks separated from the prime distribution.%2.02038, 2.01915, 0.0132725

\begin{figure}[!ht]
\begin{subfigure}{1.\textwidth}
  \centering
  \includegraphics[width=1.1\linewidth]{RMS_deriv_3H_n8_2600.png}
  \caption{RMS charge radii of 2600 $^3$H fits with no $\chi^2_{max}$ cut.}
  \label{fig:3h_rms_deriv_no_cut}
\end{subfigure}\\
\begin{subfigure}{1.\textwidth}
  \centering
  \includegraphics[width=1.1\linewidth]{RMS_deriv_3H_n8_908.png}
  \caption{RMS charge radii of 908 $^3$H fits surviving a $\chi^2_{max}$ = 500 cut.}
  \label{fig:3h_rms_deriv_cut}
\end{subfigure}
\caption[\bf{$^3$H RMS Charge Radii}]{\bf{$^3$H RMS Charge Radii.}}
\label{fig:3h_rms_deriv}
\end{figure}

As was done with $^3$He, the parameters of the $^3$H fits can be plotted for further insight. Figure ~\ref{fig:3h_qi} shows the Q$_i$ fit parameters discussed in ~\ref{sec:sog} with Figure ~\ref{fig:3h_qch} showing the $Q_{i_{ch}}$ parameters and Figure ~\ref{fig:3h_qm} showing the $Q_{i_{m}}$ parameters. Each of the 8 parameters is shown up to a maximum value of 0.5. As previously, the structure of the parameter plots did not vary significantly after the $\chi^2_{max}$ cut was applied so only the plots with the cut are shown. The results are consistent with those of $^3$He in Section ~\ref{ssec:3he_fits} in that the Q$_i$ values are larger for the inner Gaussians placed at smaller radii, and thus they have the strongest influence on the overall fits. The Q$_i$ are still strongly coupled to the R$_i$ distribution, or `model', of each individual fit with wide spreads in the Q$_i$ parameter values, especially for the inner Q$_i$ .

\begin{figure}[!ht]
\begin{subfigure}{1.\textwidth}
  \centering
  \includegraphics[width=1.1\linewidth]{Qich_3H_n8_908.png}
  \caption{$Q_{i_{ch}}$ parameters for $^3$H with $\chi^2_{max}$ = 603.}
  \label{fig:3h_qch}
\end{subfigure}\\
\begin{subfigure}{1.\textwidth}
  \centering
  \includegraphics[width=1.1\linewidth]{Qim_3H_n8_908.png}
  \caption{$Q_{i_{m}}$ parameters for $^3$H with $\chi^2_{max}$ = 603.}
  \label{fig:3h_qm}
\end{subfigure}
\caption[\bf{$^3$H Q$_i$ Fit Parameters}]{\bf{$^3$H Q$_i$ Fit Parameters.}}
\label{fig:3h_qi}
\end{figure}

Plotted in Figure ~\ref{fig:3h_ri} are the R$_i$ values (~\ref{fig:3h_ri_ind}) and the separation between consecutive R$_i$ values (~\ref{fig:3h_ri_sep}) of the fits of the $^3$H world data in fm. The insights gleaned from these plots parallel those found for $^3$He in Section ~\ref{ssec:3he_fits}. The R$_i$ values are well grouped indicating the fitting procedure is finding similar results regardless of the pseudorandom initial conditions. Again we see that  R$_{max}$ $\approx$ 5 fm. Finally, the distance between consecutive R$_i$ values still fluctuates somewhat, but the separations tend to converge to a central value, and the separation between consecutive R$_i$ values roughly doubles for the larger half of the radii.

\begin{figure}[!ht]
\begin{subfigure}{1.\textwidth}
  \centering
  \includegraphics[width=1.1\linewidth]{Ri_3H_n8_908.png}
  \caption{R$_i$ values (fm) of fits of $^3$H world data $\chi^2_{max}$ = 500.}
  \label{fig:3h_ri_ind}
\end{subfigure}\\
\begin{subfigure}{1.\textwidth}
  \centering
  \includegraphics[width=1.1\linewidth]{Ri_Sep_3H_n8_908.png}
  \caption{Separation of consecutive R$_i$ values (fm) of fits of $^3$H world data $\chi^2_{max}$ = 500.}
  \label{fig:3h_ri_sep}
\end{subfigure}
\caption[\bf{$^3$H RMS Charge Radii}]{\bf{$^3$H RMS Charge Radii.}}
\label{fig:3h_ri}
\end{figure}

A representative fit for the $^3$H form factors was selected in the same manner as the one selected for $^3$He in Section ~\ref{ssec:3he_fits}. Table ~\ref{tab:3h_rep_fit_pars} and Table ~\ref{tab:3h_rep_fit_stats} show the parameters for the representative fit of the $^3$H world data along with the metrics used to test the `goodness' of the fit respectively. Figure ~\ref{fig:3h_rep_fit} shows plots of the representative charge and magnetic form factors for the new $^3$H world data fits. The charge form factor error band is well determined past the first minimum until about Q$^2$ $\approx$ 24 fm$^{-2}$ where it begins to expand. Above Q$^2$ $\approx$ 24 fm$^{-2}$ there is alomost no knowledge of the shape or magnitude of F$_{ch}$ due to a lack of high Q$^2$ data. The magnetic form factor is fairly well determined through the first minimum and up to Q$^2$ $\approx$ 30 fm$^{-2}$. As F$_m$ passes above Q$^2$ $\approx$ 30 fm$^{-2}$ there is almost no information about the form factor. Once again, to improve the understanding of the $^3$H form factors more high Q$^2$ and back angle data is required.

%\vspace{6mm}
\begin{table}[!h]
\centering
\begin{tabular}{|c c l l|}
\hline
%\makecell{\textbf{Absorption}\\ \textbf{Spectrum Shape}} & \textbf{Paint QE} & \makecell{\textbf{Visual}\\ \textbf{Opacity}} \\
\makecell{\textbf{Parameter}\\ \textbf{Number}} & \textbf{R$_i$ (fm)} & \textbf{Q$_{i_{ch}}$} & \textbf{Q$_{i_{m}}$}\\
\hline
1 & 0.3 & 0.151488 & 0.190646 \\
2 & 0.8 & 0.348372 & 0.301416 \\
3 & 1.4 & 0.29635 & 0.318972 \\
4 & 1.9 & 0.0978631 & 0.159433 \\
5 & 2.5 & 0.121983 & 0.173933 \\
6 & 3.3 & 0.0242654 & 0.106361 \\
7 & 4.1 & 0.049329 & 0.0665564 \\
8 & 4.8 & 4.40751e-11 & 0.0148866 \\          
\hline
\end{tabular}
\caption{\bf{Parameters for $^3$H World Data Representative Fit}}
\label{tab:3h_rep_fit_pars}
\end{table}

%\begin{table}[!h]
%\centering
%\begin{tabular}{|c c c c c c c|}
%\hline
%%\makecell{\textbf{Absorption}\\ \textbf{Spectrum Shape}} & \textbf{Paint QE} & \makecell{\textbf{Visual}\\ \textbf{Opacity}} \\
%\textbf{N$_{Gaus}$} & \textbf{Avg. $\chi^2$} & \textbf{r$\chi^2$} & \textbf{BIC} & \textbf{AIC} & \textbf{$\sum$Q$_{i_{ch}}$} & \textbf{$\sum$Q$_{i_{m}}$}\\
%\hline
%8 & 601.882 & 2.77365 & 264.712 & 237.069 & 1.08965 & 1.3322\\
%\hline
%\end{tabular}
%\caption{\bf{Metrics for $^3$H World Data Representative Fit}}
%\label{tab:3h_rep_fit_stats}
%\end{table}

\begin{table}[!h]
\centering
\begin{tabular}{|c c c c c c c|}
\hline
%\makecell{\textbf{Absorption}\\ \textbf{Spectrum Shape}} & \textbf{Paint QE} & \makecell{\textbf{Visual}\\ \textbf{Opacity}} \\
\textbf{N$_{Gaus}$} & \textbf{Avg. $\chi^2$} & \textbf{r$\chi^2$} & \textbf{BIC} & \textbf{AIC} & \textbf{$\sum$Q$_{i_{ch}}$} & \textbf{$\sum$Q$_{i_{m}}$}\\
\hline
8 & 601.9 & 2.774 & 264.7 & 237.1 & 1.089 & 1.332\\
\hline
\end{tabular}
\caption{\bf{Metrics for $^3$H World Data Representative Fit}}
\label{tab:3h_rep_fit_stats}
\end{table}

\begin{figure}[!ht]
\begin{subfigure}{1.\textwidth}
  \centering
  \includegraphics[width=1.\linewidth]{3H_Fch_Rep_Fit.png}
  \caption{\bf{$^3$H Charge Form Factor for the New Representative Fit of World Data.}}
  \label{fig:3h_fch_rep_fit}
\end{subfigure}\\
\begin{subfigure}{1.\textwidth}
  \centering
  \includegraphics[width=1.\linewidth]{3H_Fm_Rep_Fit.png}
  \caption{\bf{$^3$H Magnetic Form Factor for the New Representative Fit of World Data.}}
  \label{fig:3h_fm_rep_fit}
\end{subfigure}
\caption[\bf{$^3$H Form Factors for the New Representative Fit of World Data}] {
{\bf{$^3$H Form Factors for the New Representative Fit of World Data.}} The selected representative fit is shown in black and its error band is shaded in red.}
\label{fig:3h_rep_fit}
\end{figure}

Again, the representative fit can be studied against the distribution of the world data shown in Figure ~\ref{fig:3h_data_distribution}. Both the charge and magnetic form factors agree well with  ~\cite{Article:Amroun} until a Q$^2$ of approximately 35 to 40 fm$^{-2}$ which makes sense as no new data was added for $^3$H. Looking at the distribution of the world data it becomes clear that most of the data exists below Q$^2$ $\approx$ 10 and is sparse above there tapering off completely around Q$^2$ $\approx$ 32 fm$^{-2}$. As with $^3$He, the dearth of high Q$^2$ and back angle $^3$H elastic scattering data is restricting our understanding of the form factors.

\begin{figure}[!ht]
	\begin{center}
	\includegraphics[width=1.0\linewidth]{3H_Data_Distribution.png}
	\end{center}
	\caption[\bf{$^3$H Representative Form Factors and World Data Distribution}]{
	{\bf{$^3$H Representative Form Factors and World Data Distribution.}} The top two plots show the $^3$H F$_{ch}$ and F$_m$ versus Q$^2$, and the lowest plot shows the $^3$H world data distribution in Q$^2$ used in this analysis.}
	\label{fig:3h_data_distribution}
\end{figure}

This representative fit has a total $\chi^2$ of 601.9 for 234 $^3$H elastic cross sections. This results in an average $\chi^2$ per point of 2.572. Recall that this analysis did not float the normalization of the different datasets leading to a larger $\chi^2$. Figure ~\ref{fig:3h_rep_fit_chi2_q2} shows a plot of $\chi^2$ versus Q$^2$ for each data point in each dataset in the global fits. The highest Q$^2$ data points come from the older datasets of Collard and Beck. It is interesting to note that there seems to be some structure to the Amroun dataset which will be discussed more shortly. 

\begin{figure}[!ht]
	\begin{center}
	\includegraphics[width=1.0\linewidth]{3He_Rep_Chi2_vs_Q2.png}
	\end{center}
	\caption[\bf{$^3$He Representative Fit $\chi^2$ vs. Q$^2$}]{
	{\bf{$^3$He Representative Fit $\chi^2$ vs. Q$^2$.}} }
	\label{fig:3h_rep_fit_chi2_q2}
\end{figure}

The plot of the representative fit's residual, given by Equation ~\ref{eq:residual}, is shown in Figure ~\ref{fig:3h_rep_fit_residual}. The residual plot for $^3$H is more revealing than that of $^3$He. While there are no obvious outliers there is significant structure. The data points from Amroun seem to be low on average and the points from Collard appear to be high. The Beck data points seem fairly normally distributed. This means that if the normalization had been floated the $\chi^2$ could be reduced significantly as there do seem to be strong biases based on the experiments. 

\begin{figure}[!ht]
	\begin{center}
	\includegraphics[width=1.0\linewidth]{3H_Rep_Residual.png}
	\end{center}
	\caption[\bf{$^3$H Representative Fit Residual vs. Q$^2$}]{
	{\bf{$^3$H Representative Fit Residual vs. Q$^2$.}} }
	\label{fig:3h_rep_fit_residual}
\end{figure}

\subsection{$^3$He Comparison with Theory and Previous Measurements}
\label{ssec:3he_comparison_with_theory}

Now that we have new form factor fits for $^3$He and $^3$H it is important to see how they compare to previous fits and theoretical predictions. We have already shown the previous fit from Amroun \textit{et al}., but the error bands were previously excluded for clarity and will now be examined ~\cite{Article:Amroun}. Four theory predictions from Marcucci \textit{et al}. will also be compared to the current fits of the world data ~\cite{Article:Marcucci}. The four predictions include a conventional approach, two $\chi$EFT calculations, and finally a covariant spectator theorem prediction. These methods will be briefly described below, but for a more fulsome explanation of the theory predictions see the text in reference ~\cite{Article:Marcucci}.

Paraphrasing the description of the conventional approach used by Macucci \textit{et al}. this technique simulates 2 and 3-body nucleon interactions within the nucleus and applies relativistic corrections. It also models the nucleon's interactions with external electroweak forces through both one and many-body currents. It removes nucleon resonances and replaces these with effective potentials and currents ~\cite{Article:Marcucci}.

The $\chi$EFT predictions in Marcucci \textit{et al}. uses the chiral symmetry of QDC to describe the internal strong and EM interactions. This technique requires momentum space cutoffs to regularize operators with divergent behavior at large momenta. These cutoffs are selected at 500 MeV for one model and 600 MeV for the other $\chi$EFT model ~\cite{Article:Marcucci}.

The final theoretical model employed in Marcucci \textit{et al}. is the covariant spectator theorem (CST). Again, paraphrasing ~\cite{Article:Marcucci} CST is a covariant field theory where nucleons and light mesons are used as the effective degrees of freedom. Currents and form factors can be extracted from the field theory's approximate solutions. This theory is also fully relativistic.

Figure ~\ref{fig:3he_fch_theory} shows the new $^3$He F$_{ch}$ fits from this analysis, the representative fit from this analysis, Amroun \textit{et al}.'s previous fit and error band, and the four theory predications from Marcucci \textit{et al}. It is immediately clear that the new fits (red) are in excellent agreement with the fit from Amroun \textit{et al}. These fits are very tightly grouped and agree well until Q$^2$ $\approx$ 60 fm$^{-2}$. This makes sense as the abundance of high precision low Q$^2$ data for $^3$He means we expect the charge form factor to be well understood. 


\begin{figure}[!ht]
	\begin{center}
	\includegraphics[width=1.0\linewidth]{Fch_3He_n12_852_Errors_Rep.png}
	\end{center}
	\caption[\bf{$^3$He F$_{ch}$ Comparison of Results}]{
	{\bf{$^3$He F$_{ch}$ Comparison of Results.}} The red lines are the individual fits from this analysis. The black line is the representative fit for this analysis. The blue line and shaded region is the fit from ~\cite{Article:Amroun} and its error band. The theory predictions from ~\cite{Article:Marcucci} are the green, pink, purple, and blue lines for the conventional approach, CST, $\chi$EFT 500, and $\chi$EFT 600 methods respectively.}
	\label{fig:3he_fch_theory}
\end{figure}
	
Examining how well theory is predicting the data fits we can see that the conventional approach (green) is doing a good job. This method locates the first minima well and also approximates the magnitude of F$_{ch}$ successfully. It is not totally in agreement with the data fits on the location of the second minimum, but so little data exists in this region that this difference hold little useful meaning. CST on the other hand, is doing a far worse job predicting the fits to data. CST estimates the first minimum significantly higher in Q$^2$ than the fits find, and CST also overestimates the magnitude of F$_{ch}$ below the first minimum and underestimates it above. 

The $\chi$EFT predictions are in disagreement with one another. The $\chi$EFT prediction with a cutoff of 500 MeV finds the first minimum successfully and then underestimates the magnitude of F$_{ch}$ above that minimum. Whereas, $\chi$EFT 600 expects a minimum at higher Q$^2$ while also poorly predicting the magnitude of F$_{ch}$. Overall, the conventional approach from ~\cite{Article:Marcucci} does the best job predicting the data fits and does so quite accurately. 

Figure ~\ref{fig:3he_fm_theory} mirrors Figure ~\ref{fig:3he_fch_theory} but for the $^3$He magnetic form factor. The new world data fits (red) are much less tightly grouped than they were for the $^3$He F$_{ch}$. This is because the F$_{m}$ is more dependent on the scarce high Q$^2$ and back angle data. Above Q$^2$ of 40 fm$^{-2}$ the fits diverge and actually split in to two distinct paths, one of which finds a second minimum before Q$^2$ = 60 fm$^{-2}$, and the other which does not. When compared with the previous fits from ~\cite{Article:Amroun} we see that the first minimum has shifted higher in Q$^2$ by approximately one to four fm$^{-2}$. We also see that the F$_m$ magnitude of the new fits has decreased in magnitude above Q$^2$ of 25 fm$^{-2}$ with the addition of the new high Q$^2$ data.

\begin{figure}[!ht]
	\begin{center}
	\includegraphics[width=1.0\linewidth]{Fm_3He_n12_852_Errors_Rep.png}
	\end{center}
	\caption[\bf{$^3$He F$_{m}$ Comparison of Results}]{
	{\bf{$^3$He F$_{m}$ Comparison of Results.}} The red lines are the individual fits from this analysis. The black line is the representative fit for this analysis. The blue line and shaded region is the fit from ~\cite{Article:Amroun} and its error band. The theory predictions from ~\cite{Article:Marcucci} are the green, pink, purple, and blue lines for the conventional approach, CST, $\chi$EFT 500, and $\chi$EFT 600 methods respectively.}
	\label{fig:3he_fm_theory}
\end{figure}

Turning once more to the theory predictions in ~\cite{Article:Marcucci}, we see that all of the predictions predict a significantly lower Q$^2$ for the location of the first diffractive minimum. The conventional approach and $\chi$EFT 600 come closest to finding the location of the minimum but still fall short. The magnitude of F$_m$ estimated by the conventional approach is too large in the region of the minimum; however, if the minimum of the conventional approach were shifted up in Q$^2$ to match the world data fits it appears that the magnitude would then approximate F$_m$ relatively well. 

$\chi$EFT 500 and the CST predictions both fail to predict the $^3$He F$_m$ well in either minimum location or magnitude. Overall, theory is struggling to predict the world data fits for F$_m$. It is notable that the new minimum location has actually shifted further away from theory and not closer to theory. This development merits further study to understand why the theory is poorly predicting the data. 

This analysis found a $^3$He charge radius of 1.90 fm with a standard deviation of 0.00144 fm from the new fits to world data in Section ~\ref{ssec:3he_fits}. Previous measurements from Saclay and Bates found charge radii of 1.96 fm $\pm$ 0.03 fm and 1.87 fm $\pm$ 0.03 fm respectively ~\cite{3h_proposal}. The new world data fits are slotting nicely in between these two experimental results. Theoretical predictions of the $^3$He charge radius also exist. Green's function Monte Carlo (GFMC) methods predict a charge radius of 1.96 fm $\pm$ 0.01 fm and chiral effective field theory ($\chi$EFT) predicts a radius of 1.962 fm $\pm$ 0.004 fm ~\cite{3h_proposal}. Both of these predictions are larger than the radius found by this analysis, but the experimental fit results and theory predictions are still fairly close.

Recall that the charge radius is determined by the slope of the charge form factor at a Q$^2$ of zero, and that the value of F$_{ch}$ at Q$^2$ = 0 is the same as the value of $\sum$Q$_i{_{ch}}$. From physical considerations we expect $\sum$Q$_i{_{ch}}$ = 1 as discussed in Section ~\ref{sec:sog}. However, this analysis chose not to force the Q$_i$ parameters to sum to unity as not doing so provides another useful measure of the completeness of the world data without glossing over gaps and disagreements between the various measurements. For the $^3$He fits the average $\sum$Q$_i{_{ch}}$ = 1.00840. The data and fit are doing an excellent job agreeing with our prior expectation that $\sum$Q$_i{_{ch}}$ = 1 with $\sum$Q$_i{_{ch}}$ only being slightly too high. This slightly high $\sum$Q$_i{_{ch}}$ increases the negative magnitude of the F$_{ch}$ slope a small amount so the charge radius we find would have been very slightly smaller if we had forced $\sum$Q$_i{_{ch}}$ = 1. 

\subsection{$^3$H Comparison with Theory and Previous Measurements}
\label{ssec:3h_comparison_with_theory}

We can apply the same comparisons to past fits and theory predictions performed in Section ~\ref{ssec:3he_comparison_with_theory} to $^3$H as well. Figure ~\ref{fig:3h_fch_theory} shows the charge form factor fits and theory predictions for $^3$H in the same manner as was done for $^3$He. Comparing this analysis' fits to ~\cite{Article:Amroun} they are in good agreement. The first minimum is almost identical, and the error bands overlap nicely, although there is a slight difference in magnitude after the first minimum. This is likely due to this analysis having access to the data of fewer experiments than ~\cite{Article:Amroun} had. The fits diverge above Q$^2$ of 25 fm$^{-2}$ due to a lack of data in this region. Since no new elastic $^3$H data has been added to the world data we expect to have strong agreement with the previous fits of the same data, and the agreement of the two results demonstrates the consistency of our technique with past techniques. 

\begin{figure}[!ht]
	\begin{center}
	\includegraphics[width=1.0\linewidth]{Fch_3H_n8_908_Errors_Rep.png}
	\end{center}
	\caption[\bf{$^3$H F$_{ch}$ Comparison of Results}]{
	{\bf{$^3$H F$_{ch}$ Comparison of Results.}} The red lines are the individual fits from this analysis. The black line is the representative fit for this analysis. The blue line and shaded region is the fit from ~\cite{Article:Amroun} and its error band. The theory predictions from ~\cite{Article:Marcucci} are the green, pink, purple, and blue lines for the conventional approach, CST, $\chi$EFT 500, and $\chi$EFT 600 methods respectively.}
	\label{fig:3h_fch_theory}
\end{figure}

It should be noted that JLab has recently gathered new $^3$H elastic data that has yet to be analyzed. Integrating this new data with this analysis should be a simple matter after the data are analyzed. However, the new data is not at very high Q$^2$ so it may not influence the fits greatly. The new data are useful in that they overlap with data from ~\cite{Article:Beck82} and can be used to normalize that data such that it can be incorporated in future fits.

Turning to the theory predictions, the conventional approach is preforming best once more. It is successfully finding the first minimum, but it predicts a significantly larger F$_{ch}$ magnitude beyond the minimum. $\chi$EFT 600 similarly predicts the minimum while also overestimating the magnitude of the form factor. $\chi$EFT 500 underestimates the location of the minimum in Q$^2$ while overestimating the magnitude by an even greater value. Finally, CST greatly overestimates the Q$^2$ location of the first minimum while also underestimating the form factor magnitude. The conventional and $\chi$EFT 600 models predicted the minimum well, but it appears that there is still a need to better understand the magnitude of $^3$H's F$_{ch}$.

Figure ~\ref{fig:3h_fm_theory} shows the magnetic form factor fits and theory predictions for $^3$H in the same manner as was done previously. Like the charge form factor the magnetic form factor updated world data fits are in strong agreement with the previous fit in ~\cite{Article:Amroun}. The F$_m$ error band comprised by the new fits almost perfectly overlaps with Amroun \textit{et al}. Again, no new data was added so this is to be expected. The fits are grouped relatively well until Q$^2$ of 30 fm$^{-2}$ after which they begin to diverge from each other. 

\begin{figure}[!ht]
	\begin{center}
	\includegraphics[width=1.0\linewidth]{Fm_3H_n8_908_Errors_Rep.png}
	\end{center}
	\caption[\bf{$^3$H F$_{m}$ Comparison of Results}]{
	{\bf{$^3$H F$_{m}$ Comparison of Results.}} The red lines are the individual fits from this analysis. The black line is the representative fit for this analysis. The blue line and shaded region is the fit from ~\cite{Article:Amroun} and its error band. The theory predictions from ~\cite{Article:Marcucci} are the green, pink, purple, and blue lines for the conventional approach, CST, $\chi$EFT 500, and $\chi$EFT 600 methods respectively.}
	\label{fig:3h_fm_theory}
\end{figure}

Theory predictions for the magnetic form factor of $^3$H all anticipate a significantly lower Q$^2$ first minimum than is indicated by fits of the world data. The conventional approach and $\chi$EFT 600 come closest to finding the minimum but still underestimate it by three or four fm$^{-2}$. These two predictions also overestimate the magnitude of F$_m$, but if they were shifted up in Q$^2$ to match the location of the first minimum the magnitudes would line up with the new fits decently. The $\chi$EFT 500 and CST fits both predict a first minimum that is far too low while also overestimating the F$_m$ magnitude. 

Examining the success of the four theoretical models at predicting the experimental data it is clear that the conventional approach, modelling two and three-body interactions with relativistic corrections, was the most successful. $\chi$EFT predictions often came close to matching the conventional model's success, but the $\chi$EFT models were heavily dependent on their momentum space cutoffs. The CST model generally did a poor job at predicting the data. The successful theories are doing a good job predicting the charge form factor, F$_{ch}$, although they miss the magnitude a bit in $^3$H. However, all of the models seem to be having a more difficult time predicting the magnetic form factor both in minima location and magnitude.

This analysis found a $^3$H charge radius of 2.02 fm with a standard deviation of 0.0133 fm as discussed in Section ~\ref{ssec:3h_fits}. Past measurements from Saclay found a charge radius of 1.76 fm $\pm$ 0.09 fm, and measurements from Bates found a charge radius of 1.68 fm $\pm$ 0.03 fm ~\cite{3h_proposal}. GFMC predictions estimate a radius of 1.77 fm $\pm$ 0.01 fm, and $\chi$EFT predicts a radius of 1.756 fm $\pm$ 0.006 fm. Clearly the new world data fits are finding a significantly larger charge radius than past results and theory. However, it is easy to see why this analysis is in disagreement with past measurements. Because this analysis did not force $\sum$Q$_i{_{ch}}$ = 1 as discussed in Sections ~\ref{sec:sog} and ~\ref{ssec:3he_comparison_with_theory} the slope of the form factor at Q$^2$ = 0 is purely determined by the free parameters Q$_i$. 

For the $^3$H charge radius the the average $\sum$Q$_i{_{ch}}$ for the new world data fits was 1.08991, or almost 9$\%$ larger than if the free parameters were forced to sum to unity. This higher F$_{ch}$ value at Q$^2$ = 0 means that the negative slope of the form factor has a larger magnitude than if F$_{ch}$ were forced to unity at Q$^2$ = 0. Equation ~\ref{eq:rms_derivative} shows that a larger negative slope will yield a larger charge radius. This analysis' larger radius is due to not forcing $\sum$Q$_i{_{ch}}$ = 1 as past fits have done. Now it is clear that the $^3$H world data is either less complete and accurate and/or there is more disagreement between the different experiments' results (in this case it is both) than is the case for $^3$He. This would have been hard to identify if the $\sum$Q$_i{_{ch}}$ were forced to sum to unity or if we had floated the normalizations of the different experiments.

%\begin{equation}
%    1 + 1 = 2
%    \label{eq:sum1}
%  \text{\equationlabels{1st sum}}
%\end{equation}