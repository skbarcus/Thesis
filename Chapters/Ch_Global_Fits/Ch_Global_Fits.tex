\chapter{Global Fits} % Main chapter title
\label{ch:global_fits} % For referencing the chapter elsewhere, use 

This chapter will discuss the world data for $^3$H and $^3$He elastic cross sections. This data will then be fit using a sum of Gaussians (SOG) technique. These new global fits will incorporate modern data sets added to the world data since the last global fits were performed. This will include new high Q$^2$ data from JLab for $^3$He as well as the $^3$He cross section extracted in this thesis. The SOG fitting technique allows the electric and magnetic form factors to be easily extracted as well as for charge radii to be calculated. These new results will then be compared to past fits as well as some theory predictions.

\section{World Data}
\label{sec:world_data}

The world data for $^3$H and $^3$He elastic cross sections spans numerous decades, laboratories, and continents. Due to the expansiveness of the data set there are many inconsistent methodologies employed in the different analyses collected. Efforts were taken to make these comparisons as consistent as reasonably possible, however it was often impossible with the existing literature to be certain which techniques were used. Methodological differences in modifications like radiative corrections and Dirac wave Born approximation (DWBA) techniques would be extremely time consuming to force all data sets into complete agreement. As such some of the datasets fit together are not completely apples-to-apples comparisons. Fortunately, the methodological differences result in very minor changes to the final cross sections, and thus do not significantly impact the efficacy of the new global fits.

Table ~\ref{tab:world_data_3h} lists the literature comprising the current world data of $^3$H elastic cross sections and Table ~\ref{tab:world_data_3he} contains the $^3$He world data compiled for this analysis. The table is organized chronologically from oldest dataset to most recent. The table lists the title of each publication, the first author listed on each publication, the journal the publication appeared in, and the location of the measurement with the year it was published. The table also contains physics data on each experiment. This physics data includes the rough Q$^2$ range covered by the experiment, whether the paper lists cross sections explicitly, whether the paper lists form factors explicitly, if the paper applied a phase shift correction to account for the plane wave approximation, and finally a brief note on the radiative corrections each paper used. Whenever a table entry wasn't listed or was unclear in the literature a `?' was used.
 

%\begingroup
%\begin{sideways}
\begin{landscape}
\pagestyle{empty}
\small
% Text layout
\topmargin 2.75cm      %2.75cm + shifts left.
\oddsidemargin 0.5cm  %0.5cm - shifts up.
\evensidemargin 0.5cm  %0.5cm Does nothing I can see.
\textwidth 40cm       %16cm
\textheight 21cm       %21cm
\hoffset 1.cm          %1.cm + shifts down
\voffset -2.25cm        %- shifts right

%\Rotatebox{90}{
%\thispagestyle{empty}
\begin{longtable}{c c c c c c c c c c}%[!h]
\caption{\bf{Accumulated World Data for $^3$H Elastic Scattering}}\\
%\begin{tabular}%{l c c c c c c c c c c c}
\hline
\hline
\textbf{Title} & \textbf{Authors} & \textbf{Journal} & \textbf{\thead{Date/\\Location}} & \textbf{\thead{Q$^2$ Range \\ (fm$^{-2}$)}} & \textbf{\thead{Cross \\ Sections}} & \textbf{\thead{Form \\ Factors}} & \textbf{\thead{Phase \\ Shift}} & \textbf{\thead{Radiative \\ Corrections}} \\
\hline

\thead{Elastic Electron Scattering\\from Tritium and Helium-3} & \makecell{Collard} & \makecell{Phys. Rev.\\ Vol. 138, No. 1B \cite{Article:Collard}} & \makecell{1965*\\SLAC} & 1-8 & Yes & Yes & ? & Tsai \\

\thead{Triton Form Factor\\ from 0.29-1.00 fm$^{-2}$} & \makecell{Beck\\Asai} & \makecell{Phys. Rev. C\\ Vol. 25, No. 3, 1152-1155 \cite{Article:Beck82}} & \makecell{1982\\ Saskatchewan} & 0.29-1 & \makecell{Yes} & \makecell{Yes\\ ($G_E$)} & \makecell{?} & \makecell{Meister\\Yennie} \\

\thead{Tritium Form Factors\\at Low q} & \makecell{Beck} & \makecell{Phys. Rev. C\\ Vol. 30, No. 5, 1403-1408 \cite{Article:Beck84}} & \makecell{1984*\\ NBS MIT} & 0.05-3 & \makecell{Yes} & \makecell{Yes} & \makecell{Yes\\ ($q_{eff}$)} & \makecell{Mo/Tsai} \\

\thead{Tritium Electromagnetic\\ Form Factors} & \makecell{Juster} & \makecell{Phys. Rev. Letters\\ Vol. 55, No. 21, 2261-2264 \cite{Article:Juster}} & \makecell{1985\\Saclay} & 0.3-31 & \makecell{In Amroun\\1994} & \makecell{Yes\\ (SOG)} & \makecell{?} & \makecell{Auffret} \\

\thead{Isoscalar and Isovector Form\\ Factors of $^3$H and $^3$He for Q\\below 2.9 fm$^{-1}$ from Electron-\\Scattering Measurements} & \makecell{Beck} & \makecell{Phys. Rev. Letters\\ Vol. 59, No. 14, 1537-1540 \cite{Article:Beck87}} & \makecell{1987\\Bates} & 0.03-9 & \makecell{No} & \makecell{Yes\\ (Iso)} & \makecell{Yes} & \makecell{Mo/Tsai} \\

\thead{$^3$H and $^3$He \\ Electromagnetic \\ Form Factors} & \makecell{Amroun} & \makecell{Nuc. Phys.\\ A579 596-626 \cite{Article:Amroun}} & \makecell{1994*\\Saclay} & 1-47 & Yes & Yes & Yes & \makecell{Mo/Tsai, Schwinger \\ and bremsstrahlung +\\ Landau Straggling} \\
%\hline
%\hline
%\endhead
\hline
\hline
%\endfoot
%\end{tabular}
\label{tab:world_data_3h}
\end{longtable}

%}%End rotate box.
\end{landscape}
%\end{sideways}
%\endgroup
%\pagestyle{plain}



\begin{landscape}
\pagestyle{empty}
\small
% Text layout
\topmargin 2.75cm
\oddsidemargin 0.5cm
\evensidemargin 0.5cm
\textwidth 16cm 
\textheight 21cm
\hoffset 1.cm
\voffset -1.75cm

%\Rotatebox{90}{
%\thispagestyle{empty}
\begin{longtable}{c c c c c c c c c c}%[!h]
\caption{\bf{Accumulated World Data for $^3$He Elastic Scattering}}\\
%\begin{tabular}%{l c c c c c c c c c c c}
\hline
\hline
\textbf{Title} & \textbf{Authors} & \textbf{Journal} & \textbf{\thead{Date/\\Location}} & \textbf{\thead{Q$^2$ Range \\ (fm$^{-2}$)}} & \textbf{\thead{Cross \\ Sections}} & \textbf{\thead{Form \\ Factors}} & \textbf{\thead{Phase \\ Shift}} & \textbf{\thead{Radiative \\ Corrections}} \\
\hline

\thead{Elastic Electron Scattering\\ from Tritium and Helium-3} & \makecell{Collard} & \makecell{Phys. Rev.\\ Vol. 138, No. 1B \cite{Article:Collard}} & \makecell{1965*\\SLAC} & 1-8 & Yes & Yes & ? & Tsai \\

\thead{Elastic Electron Scattering\\from $^3$He at High\\ Momentum Transfer} & \makecell{Bernheim} & \makecell{Lettere Al Nuovo Cimento\\ Vol. 5, No. 5, 431-434 \cite{Article:Bernheim}} & \makecell{1972\\Orsay} & 9-16 & \makecell{No} & \makecell{Yes} & \makecell{?} & \makecell{``Usual"} \\

\thead{Electromagnetic Structure\\of the Helium Isotopes} & \makecell{McCarthy} & \makecell{Phys. Rev. C\\ Vol. 15, No. 4, 1396-1414 \cite{Article:McCarthy}} & \makecell{1977\\ Stanford HEPL} & 0.3-20 & No & Yes & \makecell{Yes} & Mo/Tsai \\

\thead{Low-Momentum-Transfer\\ Elastic Electron\\ Scattering from $^3$He} & \makecell{Szalata} & \makecell{Phys. Rev. C\\ Vol. 15, No. 4, 1200-1203 \cite{Article:Szalata}} & \makecell{1977*\\National Bureau\\ of Standards} & 0.03-0.33 & \makecell{Yes\\$^3$He/$^{12}$C\\Exp.} & \makecell{Yes \\($F_{ch}^2$)} & \makecell{Yes} & \makecell{``In the\\Standard\\Fashion"}\\

\thead{Elastic Scattering\\ from $^3$He and $^4$He at\\ High Momentum Transfer} & \makecell{Arnold} & \makecell{Phys. Rev. Letters\\ Vol. 40, No. 22 \cite{Article:Arnold}} & \makecell{1978*\\SLAC} & 18-103 & No & \makecell{Yes \\($A^{1/2}$)} & \makecell{?} & \makecell{?} \\

\thead{Magnetic Form\\ Factor of $^3$He} & \makecell{Cavedon} & \makecell{Phys. Rev. Letters\\ Vol. 49, No. 14, 986-989 \cite{Article:Cavedon}} & \makecell{1982\\Saclay} & 7-32 & \makecell{In Amroun\\ 1994} & \makecell{Yes\\ ($F_M^2$)} & \makecell{Yes\\(HADES)} & \makecell{Yes} \\

\thead{$^3$He Magnetic\\ Form Factor} & \makecell{Dunn} & \makecell{Phys. Rev. C\\ Vol. 27, No. 1, 71-82 \cite{Article:Dunn}} & \makecell{1983*\\Bates} & 0.08-11 & \makecell{Yes} & \makecell{Yes} & \makecell{Yes} & \makecell{Bergstrom +\\ Mo/Tsai} \\

\thead{Elastic Electron Scattering\\from $^3$He and $^4$He} & \makecell{Otterman} & \makecell{Nuclear Physics\\ A436 688-698 \cite{Article:Otterman}} & \makecell{1985\\Mainz} & 0.2-3.7 & \makecell{No} & \makecell{Yes} & \makecell{Yes\\(HADES)} & \makecell{Mo/Tsai} \\

\thead{Isoscalar and Isovector Form\\Factors of $^3$H and $^3$He for Q\\below 2.9 fm$^{-1}$ from Electron-\\Scattering Measurements} & \makecell{Beck} & \makecell{Phys. Rev. Letters\\ Vol. 59, No. 14, 1537-1540 \cite{Article:Beck87}} & \makecell{1987\\Bates} & 0.03-9 & \makecell{No} & \makecell{Yes\\ (Iso)} & \makecell{Yes} & \makecell{Mo/Tsai} \\

\thead{Isospin Separation of Three-\\Nucleon Form Factors} & \makecell{Amroun} & \makecell{Phys. Rev. Letters\\ Vol. 69, No. 2, 253-256 \cite{Article:Amroun92}} & \makecell{1992*\\Saclay} & 2.6-37 & \makecell{In Amroun\\ 1994} & \makecell{No} & \makecell{Yes} & \makecell{``Standard"} \\

\thead{$^3$H and $^3$He \\ Electromagnetic \\ Form Factors} & \makecell{Amroun} & \makecell{Nuc. Phys.\\A579  596-626 \cite{Article:Amroun}} & \makecell{1994*\\Saclay} & 2-48 & Yes & Yes & Yes & \makecell{Mo/Tsai, Schwinger \\ and bremsstrahlung +\\ Landau Straggling} \\

\thead{Measurement of the Elastic\\Magnetic Form Factor of 3He\\at High Momentum Transfer} & \makecell{Nakagawa} & \makecell{Phys. Rev. Letters\\ Vol. 86, No. 24, 5446-5449 \cite{Article:Nakagawa}} & \makecell{2001*\\ Bates} & 6-43 & \makecell{Yes} & \makecell{Yes\\ ($|F_M|^2$)} & \makecell{Yes} & \makecell{Mo/Tsai} \\

\thead{JLab Measurements of the\\$^3$He Form Factors at Large\\ Momentum Transfers} & \makecell{Camsonne} & \makecell{Phys. Rev. Letters\\ Vol. 119, No. 162501, 1-6 \cite{Article:Alex}} & \makecell{2016*\\ JLab} & 25-61 & \makecell{Yes} & \makecell{Yes} & \makecell{Yes\\ ($q_{eff}$)} & \makecell{Yes} \\

%\hline
%\hline
%\endhead
\hline
\hline
%\endfoot
%\end{tabular}
\label{tab:world_data_3he}
\end{longtable}

%}%End rotate box.
\end{landscape}



The datasets used in the SOG global fits in this thesis are marked with a * after the listed dates. Not all of the datasets were able to be used in this analysis for various reasons. The most common reason a dataset was not used was simply that the publication did not list its cross section data points explicitly in the publication so they could not be added to the fit. Another common reason was publications listing only the extracted form factors and not cross sections. This is not an issue when the publication also lists the beam energy and scattering angle for each data point (or Q$^2$ and one of either the energy or angle) as the cross section can be computed using these values. However, numerous publications list only the form factors without energies or angles making it impossible to calculate a cross section for each data point to be used in the global fit. Some publications like Arnold 1978 ~\cite{Article:Arnold} used different ways to parametrize form factors, and whenever possible these methods were converted to cross sections.

\section{Sum of Gaussians Parametrization}
\label{sec:sog}

The sum of Gaussians (SOG) parametrization is a powerful method for fitting nuclear cross section data developed by Ingo Sick in the early 1970s \cite{Article:SOG}. It attempts to fit elastically scattered electron cross sections by representing the electric form factor (sometimes referred to as the charge form factor, F$_{ch}$) and the magnetic form factor (F$_{m}$) as the sum of numerous Gaussians. The technique attempts to remain model independent while taking several physical requirements for the form factors and nuclear wave functions in to account. However, a model dependence of sorts does enter the fits in the form of the radii at which the various Gaussians are situated. SOG fits make the extraction of the charge and magnetic form factors easy to extract, and along with them the charge density and charge radii of the target.

Sick outlines the rules for removing a global model dependence when fitting cross section data as follows,

\begin{quote}
	\begin{enumerate}
		\item ``Accept some clearly specified limitation to generality (accept some model dependence), since data with infinite q$_{max}$ are not available (wavelengths smaller than $\lambda = \frac{2\pi}{q_{max}}$ are not determined by experiment).
		\item Choose a restriction to generality which can be justified by physical arguments.
		\item Write the density in a manner which decouples densities at different radii as much as possible." \cite{Article:SOG}
	\end{enumerate}
\end{quote} 

One of the first physical restrictions that can be applied by the SOG parametrization is on the nuclear charge densities. No structures in the nuclear charge densities are allowed to be larger than the RMS radii of the proton \cite{Article:SOG}. As this thesis is often using \cite{Article:Amroun} as a point of comparison this work employs the same minimum size allowed for structure used by Amroun \textit{et al.} of 0.8 fm, or slightly less than the proton's radius. 

A second restriction comes from the fact that the amplitudes of the high frequency components of the nuclear wave functions have limits placed on them. Paraphrasing Sick this can be seen by noting that the Schr{\"o}dinger equation strongly couples the second derivative of the radial wave functions to the proton's separation energy. This implies that the proton will become less bound for larger amplitudes and shorter wavelengths of the radial wave functions. This is yet another reason to forbid certain structures in the nuclear charge density smaller than a certain width \cite{Article:SOG}.

Gaussians are used to build the structure of the fits since they mimic the peaks of the radial wave functions. They also fall off quickly enough so as to not strongly interfere with other Gaussians not nearby them satisfying rule 2 above. Gaussians also work well with the rules and limitations imposed earlier. One can write the nuclear charge density as shown in Equation ~\ref{eq:sog_rho_no_tail}, where the charge density is represented as a sum of numerous Gaussians set at different radii R$_i$. The A$_i$ amplitudes are fitted to the cross sectional data. These Gaussians have their full width at half maximum restricted by the parameter $\gamma$ as required by the physical restrictions imposed above. The smallest width structure allowed is given by $\Gamma$ where $\Gamma = 2\gamma \sqrt{ln(2)}$ \cite{Article:SOG}. 

\begin{equation} \label{eq:sog_rho_no_tail}
	\rho(r) \propto \sum_{i=1}^N A_i e^{-\left( r-R_i \right)^2/\gamma^2}
\end{equation}

As previously mentioned, the R$_i$, representing the radii at which different Gaussians are located, form their own sort of model dependence. Since we are unable to study what happens above q$_{max}$ the R$_i$ are analogous to a model of how the charge density behaves above q$_{max}$. This issue can be resolved by choosing many different R$_i$ values randomly and fitting the A$_i$ to the data for each set of R$_i$ chosen randomly. The choice of R$_i$ may be random but it does have numerous conditions applied. More will be said on the selection of the R$_i$ later on***(maybe have an Ri selection section to link to later). 

Once a large number of fits of the data using different sets of R$_i$ have been generated the `good' fits must be distinguished from the `bad'. This is done in several ways which will be discussed in more detail in Section ***, and include finding lower $\chi^2$ fits as well as making sure the fit's form factors appear physical. Once the `good' fits are identified an error band can be built up by plotting each of the fits on top of one another. After a sufficient number of different `good' R$_i$ sets have been fitted to the data the whole of the available model space has been explored. 

The charge density is expected to have a derivative of zero at a $r=0$. This is not accounted for in Equation ~\ref{eq:sog_rho_no_tail}. To resolve this issue a tail can be added to each Gaussian that represents the Gaussian's behavior at $r<0$. This modified definition of the charge density is given in Equation ~\ref{eq:sog_rho} \cite{Article:SOG}.

\begin{equation} \label{eq:sog_rho}
	\rho(r) = \frac{Ze}{2 \pi^{3/2}\gamma^3} \sum_{i=1}^N \frac{Q_i}{1+\frac{2R_i^2}{\gamma^2}} \left( e^{-\left( r-R_i \right)^2/\gamma^2} + e^{-\left( r+R_i \right)^2/\gamma^2} \right)
\end{equation}

\noindent Equation \ref{eq:sog_rho} is normalized by Equation ~\ref{eq:normalization}. The Q$_i$ are now the parameters fitted to the data. The Q$_i$ are required to be positive as they represent the fraction of electric or magnetic charge carried by each Gaussian. $\sum Q_i=1$ is also required of the Q$_i$ terms as all of the charge fractions must sum to the total charge (one). $Z$ is the atomic number of the target and $e$ is the elementary charge \cite{Article:SOG}.

\begin{equation} \label{eq:normalization}
	4 \pi \int_0^{\infty} \rho(r) r^2 dr = Ze
\end{equation}

When using the plane wave Born approximation (PWBA) the electric and magnetic form factors can be parametrized as in Equation ~\ref{eq:sog_ffs} \cite{Article:SOG}.

\begin{equation} \label{eq:sog_ffs}
	F_{(ch,m)}(q) = exp \left(-\frac{1}{4} q^2 \gamma^2 \right) \sum^{n}_{i=1} \frac{Q_i{_{(ch,m)}}}{1+2R^2_i/\gamma^2} \left( \cos(qR_i) + \frac{2R^2_i}{\gamma^2} \frac{\sin(qR_i)}{qR_i} \right)
\end{equation}

\noindent At this point we will follow the procedure laid out in \cite{Article:Amroun} and note that there is a typo in the reference. In \cite{Article:Amroun} Equation (1) the -1/2 in the exponent should be a -1/4. Again the Q$_i$ are fitted to the data and represent the fraction of the electric or magnetic charge carried by each Gaussian. The R$_i$ are the radii at which the Gaussians are placed. $q$ is the four-momentum transferred via the virtual photon as discussed in Section ~\ref{sec:kinematics}. Lastly $\gamma$ is defined as $\gamma \sqrt{\frac{3}{2}}=0.8$ fm \cite{Article:Amroun}. 

The cross section can be represented in PWBA with the SOG parametrization as shown in Equation ~\ref{eq:xs}.

\begin{equation} \label{eq:xs}
	\frac{d\sigma}{d\Omega} = \left( \frac{d\sigma}{d\Omega} \right)_{Mott} \frac{1}{\eta} \left[ \frac{q^2}{\boldsymbol{q}^2}F_{ch}^2(q) + \frac{\mu^2q^2}{2M^2} \left( \frac{1}{2} \frac{q^2}{\boldsymbol{q}^2} + \tan^2 \left( \frac{\theta}{2} \right) \right)F_{m}^2(q) \right]
\end{equation}

\noindent Here $\eta = 1 + q^2/4M^2$, $q^2$ is the squared four-momentum transfer from ~\ref{eq:Q^2}, $\boldsymbol{q}^2$ is the three-momentum, $\mu$ is the magnetic moment of the target ($\mu_{^3He}$ = -2.1275*(3.0/2.0) and $\mu_{^3H}$ = 2.9788*(3.0/1.0)), and $M$ is the mass of the target (M$_{^3He}$ = 3.0160293 amu and M$_{^3H}$ = 3.0160492 amu) ~\cite{Article:Amroun}. 

The Mott cross section is shown in equation ~\ref{eq:mott}.

\begin{equation} \label{eq:mott}
	\left( \frac{d\sigma}{d\Omega} \right)_{Mott} = Z^2 \frac{E'}{E_0} \frac{\alpha^2 \cos^2 \left( \frac{\theta}{2} \right)^2}{4E_0^2 \sin^2 \left( \frac{\theta}{2} \right)^4} 
\end{equation}
 
\noindent $Z^2$ accounts for the charge of the target with $Z$ being the target's atomic number, $\frac{E'}{E_0}$ is the recoil factor with $E_0$ being the scattered electrons initial energy and $E'$ is the energy after scattering, $\alpha$ is the fine structure constant, and $\theta$ is the scattering angle. It is extremely important to be mindful of the units one is using when working with these equations. Be cautious of interchanging degrees and radians for the scattering angle, fm$^{-2}$ and GeV$^2$ for the squared four-momentum values, fm$^{-1}$ and GeV for the energies, and amus and GeV for the mass units. Equation ~\ref{eq:gev2fm} shows the equivalent amount of GeV$^2$ to fm$^{-2}$ in nuclear units.

\begin{equation} \label{eq:gev2fm}
	1 \; GeV^2 \approx 25.7 \; fm^{-2}
\end{equation}

The assumption that the wave functions of the electrons are plane waves is not entirely correct. The nucleus' charge distorts these wave functions due to the Coulomb interaction, and thus shifts the Q$^2$ value to Q$^2_{eff}$ given in Equation ~\ref{eq:qeff}. This leads to Q$^2_{eff}$ taking the place of Q$^2$ in the above equations in this section (i.e. Q$^2$ is taken from the literature and then Q$^2_{eff}$ is then calculated and used in the fits) \cite{Article:Alex}.

\begin{equation} \label{eq:qeff}
	Q^2_{eff} = Q^2 \left(  1+ \frac{1.5Z\alpha}{E_0*1.12*A^{\frac{1}{3}}}   \right)^2
\end{equation}

\noindent Here $A$ is the mass number and the other variables are defined above. The three-momentum, $\boldsymbol{q}^2$, is then given by Equation ~\ref{eq:three-momentum} where $\nu=E_0-E'$ as in ~\ref{eq:nu}.

\begin{equation} \label{eq:three-momentum}
	\boldsymbol{q}^2 = \nu^2 - Q^2_{eff}
\end{equation}

\section{New SOG Fits}
\label{sec:new_fits}

The world data for $^3$H and $^3$He described in Section ~\ref{sec:world_data} will be fitted with the sum of Gaussians parametrization described in Section ~\ref{sec:sog} in this section. This section will explain the choices made for each of the SOG fits such as the number of Gaussians used to fit the world data. It will also describe the choices made involving the Q$_i$ fit parameters. The placement and spacing of the R$_i$ radii at which the Gaussians are placed will also be discussed. A method used to try to optimize the fits by adjusting the R$_i$ spacing while attempting to minimize $\chi^2$ will be described. 

\subsection{Gaussian Radii Placement}
\label{ssec:radii}

As mentioned in Section ~\ref{sec:sog} the R$_i$ are the radii at which the SOG Gaussians are place. This means that they represent a sort of model dependence. To explore all of the model space many different random R$_i$ combinations must be used to fit the world data. However, selecting the R$_i$ totally at random is extremely inefficient since we are only interested in R$_i$ combinations that yield reasonable fits and physical form factors. 

To explore the R$_i$ combinations we are interested in effectively a few rules can be applied to their selection. The first of which is that there is some radii, R$_{max}$, beyond which the charge density has fallen almost to zero. Therefore, there is no reason to position and R$_i$ beyond R$_{max}$. For $^3$H and $^3$He R$_{max}$ is $\approx$ 5 fm, although it is allowed to diverge from this radii as the fits are optimized. For the majority of fits, after the optimization procedure discussed below, 4 fm $<$ R$_{max}$ $<$ 6 fm centered around 5 fm.

Once a reasonable upper limit on the radii is established the spacing of the R$_i$ from one another must be set. It has been found that the spacing of R$_i$ for R$_i$ $<$ R$_{max}$/2 should be approximately half as far apart as the R$_i$ spacing for the radii positioned above R$_{max}$/2 \cite{Article:SOG}. This is done so that the charge density region with more charge, i.e. closer to the nucleus, is described by more Gaussians. This allows the structure to be better captured by the SOG fits. Further away from the nucleus, where there is less charge, fewer Gaussians are needed to accurately describe the structure of the charge density.   

Once the R$_i$ are selected each combination of R$_i$ are fitted using the SOG parametrization. Since we are interested in the fits that best describe the data while producing form factors that conform to the physical priors we expect it is logical to search for the lowest $\chi^2$ fits and check the form factors visually for physicality. We define $\chi^2$ as in Equation ~\ref{eq:chi2}, where $N$ is the number of data points being fit, $\sigma_{exp}$ is the experimentally measured cross section at a particular Q$^2$, $\sigma_{fit}$ is the cross section given by the global fit at the same Q$^2$ as the experimental cross section, and $\Delta$ is the total uncertainty attached to the experimental cross section at a given Q$^2$. A lower $\chi^2$ value naively indicates a better fit, but numerous flaws can occur when using only $\chi^2$ \cite{doug_stats}.  

\begin{equation} \label{eq:chi2}
	\chi^2 = \sum_{n=1}^N \frac{\left( \sigma_{exp}-\sigma_{fit} \right)^2}{\Delta^2}
\end{equation}

Initial R$_i$ spacings tend to be fairly unfavorable and produce large $\chi^2$ values and strangely shaped form factors. To minimize the $\chi^2$ as much as possible the R$_i$ values need to be allowed to shift. This is accomplished by first fitting the data with an initial set of R$_i$ values. After the initial fit is done each of the R$_i$ values is then optimized. If R$_0$ was initially 0.2 fm the fit would then be redone with R$_0$ = 0.1 fm and then R$_0$ = 0.3 fm. The R$_i$ are each shifted up and down 0.1 fm until $\chi^2$ gets larger. The R$_i$ that yielded the smallest $\chi^2$ is then kept as the `optimal' R$_i$. Once this procedure is completed for each R$_i$ in ascending order the lowest, or at least close to the lowest, $\chi^2$ value for R$_i$ similar to the initial R$_i$ has been found.  

As an example, let us examine the initial R$_i$ spacings for $^3$H using eight Gaussians. While the order of the Gaussians is irrelevant it is easier to code the R$_i$ in ascending radii length. Next we choose R$_0$-R$_7$ to meet the rules defined above. We want all of the R$_i$ to sum to approximately 5 fm, and the R$_i$ spacing between consecutive Gaussians should be smaller at smaller radii. For $^3$H with eight Gaussians the initial R$_i$ spacing is produced within given ranges randomly and then optimized as previously described. The ranges for the R$_i$ spacings are divided in steps of 0.1 fm. The first Gaussian is placed near R = 0 fm and was chosen to be R$_0$ = 0.2-0.3 fm. This means that R$_0$ was randomly selected to initially be 0.2 fm or 0.3 fm. Note that an R$_0$ of 0 fm leads to poles in the parametrization. To avoid this issue a small number is used in place of 0 if R$_0$ = 0 fm is found to be the optimal radius. 

After the first Gaussian is placed at R$_0$ Gaussians R$_{1-7}$ are placed by semi-randomly choosing their distance from the radii prior to them. The spacing for R$_{1-4}$ = 0.5-0.6 fm, and the spacing for R$_{5-7}$ = 0.8-0.9 fm. Notice that the radii further from the nucleus are placed approximately twice as far apart as the inner radii in accordance with the rules previously described. We can take the average spacing of each consecutive radius and sum them to find R$_{max}$. Doing this we find 0.25 fm + 4*0.55 fm + 3*0.85 fm = 5 fm which is the target R$_{max}$. This process is then repeated for hundreds of semi-randomly generated R$_i$ sets which span the model space for $^3$H and $^3$He. 

\subsection{Number of Gaussians}
\label{ssec:ngaus}

To utilize the SOG parametrization it is necessary to select the number of Gaussians, N$_{Gaus}$, to use for each fit. This process involves balancing several competing interests. If too few Gaussians are used the structure of the form factors may not be described in enough detail, but if too many Gaussians are used the data may be overfit. Overfitting would lead to the statistical noise in the data being mistaken for signal. The goal is then to fit the data as well as possible with no more parameters than required. A commonly used tool for selecting the best model is to calculate the $\chi^2$ value for a fit, however this is insufficient and can often be deceptive and lead to issues such as overfitting \cite{doug_stats}. 

To avoid this issue numerous other tests and metrics were applied when selecting N$_{Gaus}$ for $^3$H and $^3$He. Among these are the $\chi^2$ value, the reduced $\chi^2$ value, Bayesian information criterion, Akaike information criterion, the sums of the fractions of the electric and magnetic charges held by the Gaussians, the percentage of fits that were deemed `good', and finally a visual inspection of the form factors for known physical characteristics. By combining these different tests it is possible to determine how many Gaussians provides and optimal fit while noting that it is not uncommon for two consecutive numbers of Gaussians to yield reasonably similar fits \cite{doug_stats}.

Reduced $\chi^2$, or r$\chi^2$, is similar to $\chi^2$ from Equation ~\ref{eq:chi2} except that it takes the number of data points and the number of parameters used in the fit in to account. The equation for reduced $\chi^2$ used in this analysis is given in Equation ~\ref{eq:rchi2}, where $\chi^2$ is from ~\ref{eq:chi2}, $N$ is the number of data points in the fit, and $N_{var}$ is the number of free parameters, or variables, used in the fit. Note that while $\chi^2$ must always decrease with the number of parameters added r$\chi^2$ can increase if too many parameters have been added. This makes finding the fits with the lowest r$\chi^2$ an elementary, but still useful, test that the proper number of parameters are being use to describe the data.

\begin{equation} \label{eq:rchi2}
	r\chi^2 = \frac{\chi^2}{N-N_{var}-1}
\end{equation}

The next two tests applied to determine the number of Gaussians to use in the SOG fits are Akaike information criterion (AIC) defined in Equation ~\ref{eq:AIC} \cite{Article:AIC} and Bayesian information criterion (BIC) defined in Equation ~\ref{eq:BIC} \cite{Article:BIC}. AIC and BIC are both a more advanced type of statistical test useful for selecting the proper model to use. The primary difference between the two is that BIC applies a larger penalty based on the number of model parameters used to fit the data. The way to select the correct model is to find the lowest AIC and BIC values, while remembering that these tests may choose slightly different models than the other tests and each other. To determine how much more evidence there is for one model versus another we can look at the difference between their BIC values, $\Delta$BIC. A $\Delta$BIC of $0<\Delta$BIC$<2$ indicates no real difference between models, $2<\Delta$BIC$<6$ indicates that there is positive evidence for the lower valued model, $6<\Delta$BIC$<10$ indicates strong evidence for the lower valued model, and $\Delta$BIC$>10$ indicates very strong evidence for the lower valued model \cite{Article:Delta_BIC}.

\begin{equation} \label{eq:AIC}
	AIC = N \ln\left( \frac{\chi^2}{N} \right) + 2 N_{var}
\end{equation}

\begin{equation} \label{eq:BIC}
	BIC = N \ln\left( \frac{\chi^2}{N} \right) +  \ln\left( N \right) N_{var}
\end{equation}

When selecting the number of Gaussians to use the sum of the electric and magnetic charges is also examined. The sum of the Q$_i$ charges should sum to a charge of unity, however the fits do not enforce this requirement. Instead the sum of the Q$_i$ are allowed to fluctuate with the best fit values of the individual Q$_i$. This then makes the sums another sort of test of the goodness of each fit. A `better' fit, or one that complies more with our predetermined knowledge of the form factors, will have Q$_i$ sums closer to unity. Values further from unity can indicate a worse fit, but they also help to indicate where more data is needed.

Finally a visual inspection of the form factors is applied. It is known that the form factors should have sharp minima as discussed in Section ~\ref{sec:ffs}. Often the fits will have only a dip in the form factor where a sharp minimum should exist and can thus be discarded. An example of this is seen in Figure ***. These nonphysical dip only fits tend to have higher $\chi^2$ values so cutting on $\chi^2$ can generally eliminate them. More specifically, this is done by plotting the charge form factor F$_{ch}$ and lowering the $\chi^2$ cut until all of the dip only minima fits are removed leaving only the sharp minima expected. These remaining fits are deemed to be the `good' fits. This process is done with the charge form factor as we have better data there. This procedure generally improves the corresponding magnetic form factors as well, but the lack of high Q$^2$ data for F$_m$ leads to more nonphysical or odd fits of F$_m$.

When fitting with any number of Gaussians many of the fit results do not meet the definition of a `good' fit described above. The percentage of the `good' fits to total fits is representative of the likelihood of fits of N$_{Gaus}$ to converge to physical looking fits. Assume N$_{Gaus}$ = 9 gives a good fit 40$\%$ of the time and N$_{Gaus}$ = 8 gives a good fit 5$\%$ of the time and has a slightly lower average BIC than N$_{Gaus}$ = 9. This analysis takes the low convergence rate as evidence against the slightly lower BIC results and may favor the marginally higher BIC results assuming $\Delta$BIC is small. 

Previous analyses have also done a good job locating the first minima of the form factors and can be used to check the reasonableness of this analysis' fits. For example \cite{Article:Amroun} locates the first minima of both $^3$H and $^3$He fairly well, and \cite{Article:Alex} does the same for $^3$He. If this analysis' results diverge significantly in the previously well understood regions that is taken to be a strike against the model selected. Some movement in the magnetic form factor is not unexpected since more high Q$^2$ data is being incorporated into this analysis.

Now that we have established the tools with which to select a model let us determine how many Gaussians to use when fitting $^3$H and $^3$He. $^3$He will be examined first due to there being more and often higher quality data for $^3$He than $^3$H. The method used to determine the number of Gaussians (i.e. the model) to use to fit the data was to run 100 fits of $^3$He with various numbers of Gaussians and then compute the various tests and metrics laid out above. These results were then compared and the `optimal' number of Gaussians was determined.

Table ~\ref{tab:3he_ngaus} shows the results of this model selection analysis for $^3$He. All of the values in the table are averages of the surviving `good' fits and the best values are bolded as is the final selection for the N$_{Gaus}$. $\chi^2_{max}$ is the maximum $\chi^2$ that removed all of the nonphysical dip minima in the charge form factor and $\%$ `Good' is the percentage of the 100 fits that survived this cut. Note that not all of these factors are weighted equally. The highest preference is given to BIC and AIC followed by r$\chi^2$ and visually inspecting the form factors. The other factors add more detailed information and are used more as tiebreakers and to raise red flags if something major is wrong.

\begin{table}[!h]
\centering
\begin{tabular}{|c c c c c c c c c|}
\hline
%\makecell{\textbf{Absorption}\\ \textbf{Spectrum Shape}} & \textbf{Paint QE} & \makecell{\textbf{Visual}\\ \textbf{Opacity}} \\
\textbf{N$_{Gaus}$} & \textbf{Avg. $\chi^2$} & \textbf{r$\chi^2$} & \textbf{BIC} & \textbf{AIC} & \textbf{$\sum$Q$_{i_{ch}}$} & \textbf{$\sum$Q$_{i_{m}}$} & \textbf{$\chi^2_{max}$} & \textbf{$\%$ `Good'} \\
\hline
8 & 584.902 & 2.41695 & 255.440 & 223.228 & \textbf{1.00769} & 1.11065 & 765 & 11 \\
9 & 470.435 & 1.96014 & 204.590 & 172.375 & 1.00851 & \textbf{1.02161} & 521 & 58 \\
10 & 469.177 & 1.97133 & 209.454 & 173.793 & 1.00812 & 1.08196 & 519 & 66 \\
11 & 445.136 & 1.88617 & \textbf{201.387} & 162.233 & 1.00843 & 1.04007 & 503 & 67 \\
\textbf{12} & \textbf{436.264} & \textbf{1.86438} & 201.727 & \textbf{159.045} & 1.00839 & 1.02557 & 501 & \textbf{75} \\
13 & 439.084 & 1.89260 & 208.924 & 162.685 & 1.00947 & 1.03975 & 500 & 56 \\
\hline
\end{tabular}
\caption{\bf{Determination of N$_{Gaus}$ for $^3$He}}
\label{tab:3he_ngaus}
\end{table}

Examining Table ~\ref{tab:3he_ngaus} it is clear that no model had the best value in every category so some further analysis is required to select N$_{Gaus}$ for $^3$He. N$_{Gaus}$ = 12 has the best value in both r$\chi^2$ and AIC which are both important metrics. Examining the other big metric of the BIC it can be seen that N$_{Gaus}$ = 12 and N$_{Gaus}$ = 11 have nearly identical BIC values. In fact $\Delta$BIC$<0.4$ which indicates a negligible preference between the models. All of the $\sum$Q$_{i_{ch}}$ values are fairly close offering little insight. The $\sum$Q$_{i_{m}}$ value for N$_{Gaus}$ = 12 is on the better end of the spectrum as well. N$_{Gaus}$ = 12 also had the most fits converge to be designated good fits with physical looking charge form factors. N$_{Gaus}$ = 12 also had the lowest average $\chi^2$ value, but this metric can be misleading as $\chi^2$ must always decrease as the number of parameters increases, however the different R$_i$ configurations and averaging the results makes this not necessarily true for the value in the table. Upon reviewing these metrics it is fairly clear the N$_{Gaus}$ = 12 is the best model to use for fitting the $^3$He data.

Table ~\ref{tab:3h_ngaus} mirrors Table ~\ref{tab:3he_ngaus} and shows the results of this model selection analysis for $^3$H. It is immediately obvious that the $\chi^2$ values are larger for $^3$H than they were for $^3$He. This is because there are fewer data points especially at higher Q$^2$. The $\sum$Q$_i$ values are also further from unity with the magnetic charges being especially far off. Once again this is a product of the dearth of high Q$^2$ data in the world data. If more $^3$H data could be obtained at high Q$^2$ these values would likely fall more in line with our prior expectations. The poor $\sum$Q$_{i_{m}}$ agreement with the expectation of unity also demonstrates the analysis value of not forcing the Q$_i$ to sum to unity which could hide the need for more high Q$^2$ data.

There are two entries for N$_{Gaus}$ = 8 labelled close and wide. These refer to the initial spacing of the R$_i$ values. For the close entry R$_0$ = 0.2-0.3, R$_{1-4}$ = 0.3-0.4, and R$_{5-7}$ = 0.5-0.6, and for the wide entry R$_0$ = 0.2-0.3, R$_{1-4}$ = 0.5-0.6, and R$_{5-7}$ = 0.8-0.9 as explained in Section ~\ref{ssec:radii}. This meant for the close R$_i$ the average starting R$_{max}$ = 3.3 fm and for the wide spacing R$_{max}$ = 5 fm which is what we expect from previous analyses ~\cite{Article:Amroun}. This test was done to see if the final fit results depended on the starting R$_i$ values or if the R$_i$ optimization produced consistent results with less reasonable initial R$_i$ values.

Fortunately, the results for the closer and wider R$_i$ spacings come out very similar indicating that the initial choice of R$_i$ does not significantly change the final result. This test had also previously been done for $^3$He with N$_{Gaus}$ = 10 with similar results to $^3$H with N$_{Gaus}$ = 8. The major differences were that the close less reasonable R$_i$ took longer for the R$_i$ optimization code to process, as would be expected. The close R$_i$ also had fewer fits converge to be designated `good' indicating that more of the initial R$_i$ may have been more unfavorable than the wider spacings. Thus, if the initial R$_i$ distributions are off the final fit results should generally still be reliable.  

\begin{table}[!h]
\centering
\begin{tabular}{|c c c c c c c c c|}
\hline
%\makecell{\textbf{Absorption}\\ \textbf{Spectrum Shape}} & \textbf{Paint QE} & \makecell{\textbf{Visual}\\ \textbf{Opacity}} \\
\textbf{N$_{Gaus}$} & \textbf{Avg. $\chi^2$} & \textbf{r$\chi^2$} & \textbf{BIC} & \textbf{AIC} & \textbf{$\sum$Q$_{i_{ch}}$} & \textbf{$\sum$Q$_{i_{m}}$} & \textbf{$\chi^2_{max}$} & \textbf{$\%$ `Good'} \\
\hline
7 & 611.690 & 2.79310 & \textbf{263.039} & 238.851 & \textbf{1.08373} & 1.32730 & 611.7 & 1\\
8 close & 601.836 & 2.77344 & 264.694 & 237.051 & 1.09013 & 1.32859 & 603 & 32\\
\textbf{8} wide & 601.752 & 2.79892 & 264.661 & \textbf{237.018} & 1.08970 & 1.33270 & 603 & 39\\
9 & 601.768 & 2.82579 & 270.123 & 239.025 & 1.08849 & 1.31982 & 604 & \textbf{95}\\
10 & 601.893 & 2.84416 & 275.627 & 241.074 & 1.09248 & \textbf{1.29611} & 603 & 78\\
11 & \textbf{600.750} & \textbf{2.77305} & 280.637 & 242.629 & 1.08699 & 1.34100 & 602 & 88\\
\hline
\end{tabular}
\caption{\bf{Determination of N$_{Gaus}$ for $^3$H}}
\label{tab:3h_ngaus}
\end{table}

Again, the agreement between the metrics is not unanimous, and in fact it is even less clear than for $^3$He. Let us begin by examining the lowest BIC value for N$_{Gaus}$ = 7. The other metrics also look decent until one notices that only one fit met the standards for a `good' fit. This failure for the vast majority of fits to look physical indicates that N$_{Gaus}$ = 7 is probably not a good choice for the best model. Examining the lowest AIC value for the wider R$_i$ spacings and N$_{Gaus}$ = 8 the other metrics look acceptable with a reasonable number converging to `goodness'. $\Delta$BIC for the wider R$_i$ spacings and N$_{Gaus}$ = 8 compared to N$_{Gaus}$ = 7 is only 1.6 indicating that there is little reason to prefer one model over the other. The higher Gaussian fits look reasonable as well, but the AIC, and especially BIC, grow significantly as N$_{Gaus}$ increases ruling out these fits so we select N$_{Gaus}$ = 8 with the wider initial R$_i$ spacings.

\subsection{$^3$He Fits}
\label{ssec:3he_fits}

\subsection{$^3$H Fits}
\label{ssec:3h_fits}
