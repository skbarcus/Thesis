% Introduction
\chapter{Data Analysis} % Main chapter title
\label{ch:analysis}

\section{Overview}
\label{sec:analysis_overview}

This chapter will explore the various analyses used to extract a $^3$He elastic cross section. This discussion will include all measurements required to extract a differential cross section, the corrections for efficiency losses for these values, the physics cuts applied to the data, and an estimate of the uncertainties. Also discussed will be the Monte Carlo software used to create an artificial elastic electron spectrum as well as the software used to calculate radiative corrections to this data.

\section{Experimental Cross Section}
\label{sec:exp_xs}

The theoretical origins of a differential cross section were explored in ~\ref{sec:xs}. However, this derivation is not particularly intuitive for extracting an actual experimental cross section. In practice extracting a cross section is essentially an exercise in electron counting. Let us now write the cross section as an experimentalist understands it as in Equation ~\ref{eq:exp_xs}. Each of these variables will be examined in detail later in this chapter.

\begin{equation} \label{eq:exp_xs}
	\left(\frac{d\sigma}{d\Omega}\right)_{exp} = \frac{ps*N_e}{N_{in}*\rho *LT* \epsilon_{det}} \frac{1}{\Delta\Omega\Delta P \Delta Z}
\end{equation}

Here $ps$ represents the prescale value of the given trigger being examined. $N_e$ is the number of electrons detected by the experiment that survive physics cuts, like particle identification, and acceptance cuts. $N_{in}$ is the number of electrons incident on the target, and can be calculated from the cumulative beam charge. $\rho$ is the target's density. $LT$ is the livetime correction which accounts for events missed due to electronic dead-time in the DAQ. $\epsilon_{det}$ represents the product of all of the detector efficiencies such as the GC, VDC single track, EM calorimeter, and trigger efficiencies. $\Delta\Omega$ represent the solid angle fraction covered by the spectrometer after acceptance cuts. $\Delta P$ is the momentum acceptance seen by the spectrometer after acceptance cuts. Finally $\Delta Z$ represents the length of the target seen by the spectrometer.

\subsection{Beam Charge}
\label{ssec:charge}

As the the cross section extraction is essentially an exercise in electron counting let us begin by finding the charge of the electron beam during any given experimental run. This process begins with the U and D BCMs measuring the beam current prior to the beam striking the target. These BCMs require are calibrated according to Equation ~\ref{eq:bcms}, and the constants can be found in Table ~\ref{tab:bcm_constants}. The BCM calibration for experiment E08-014 can be read about more in ~\cite{bcm_calibration} where these equations are found. The final constants in the table were updated after the analysis in ~\cite{bcm_calibration}, and were provided via private communication with Dien Nguyen.

\begin{equation} \label{eq:bcms}
	\langle I_{beam} \rangle = \frac{\frac{scalar}{time}-K'_{offset}}{C_{V-to-F}}
\end{equation}

\begin{table}[!h]
\centering
\begin{tabular}{|c c c|}
\hline
%\makecell{\textbf{Absorption}\\ \textbf{Spectrum Shape}} & \textbf{Paint QE} & \makecell{\textbf{Visual}\\ \textbf{Opacity}} \\
\textbf{BCM} & \textbf{$K'_{offset}$} & \textbf{$C_{V-to-F}$} \\
\hline
$U_1$ & 200 & 1035 \\
$D_1$ & 37 & 1263\\
\hline
\end{tabular}
\caption{\bf{BCM Calibration Constants for E08-014}}
\label{tab:bcm_constants}
\end{table}

The average current, $\langle I_{beam} \rangle$, is thus a product of the number of counts in the $scaler$ measuring beam current in a certain period of $time$ tracked by a clock scaler, and then modified by the two calibration constants $K'_{offset}$ and $C_{V-to-F}$. Now that we have a current we can calculate the charge, $Q$, of the beam during a run by Equation ~\ref{eq:charge}. We can then divide the charge by the elementary charge of the electron, $e$, to find the total number of electrons incident on our target, $N_{in}$, as in Equation ~\ref{eq:electrons}.

\begin{equation} \label{eq:charge}
	Q = \langle I_{beam} \rangle *time
\end{equation}

\begin{equation} \label{eq:electrons}
	N_{in} = \frac{Q}{e}
\end{equation}

Figure ~\ref{fig:bcms_4074} shows the current as measured by the $U_1$ and $D_1$ BCMs for experimental run 4074. During E08-014 the target beam current 120 $\mu$A. When the beam `trips' turning off and when the beam is being brought online it tends to be somewhat unstable. These events can be seen in the trailing edges and leading edges of the gaps in the BCM current measurement. Due to the beam's instability at these times measurements taken during them are discarded by placing cuts on the current spectrum represented by the red lines in ~\ref{fig:bcms_4074}. These lines are placed two scaler readouts, about 4 seconds per readout, after the BCMs register 90$\%$ or greater of the 120 $\mu$A operating current. This analysis was performed for each run of KIN 3.2 and the results are shown in Table ~\ref{tab:charges}.

\begin{figure}[!ht]
\begin{center}
\includegraphics[width=1.1\linewidth]{BCMs_4074.png}
\end{center}
\caption{
{\bf{BCM Readouts for Run 4074.}} These plots show the $U_1$ and $D_1$ BCM measurements for run 4074. The cuts are applied two scaler readouts after the current rises above or below the 90$\%$ threshold of the 120 $\mu$A target current.}
\label{fig:bcms_4074}
\end{figure}

\begin{table}[!h]
\centering
\begin{tabular}{|c c c c c c c|}
\hline
%\makecell{\textbf{Absorption}\\ \textbf{Spectrum Shape}} & \textbf{Paint QE} & \makecell{\textbf{Visual}\\ \textbf{Opacity}} \\
\textbf{Run} & \makecell{\textbf{Charge}\\ \textbf{$U_1$ ($\mu C$)}} & \makecell{\textbf{Charge}\\ \textbf{$D_1$ ($\mu C$)}} & \makecell{\textbf{Average}\\ \textbf{Charge}\\ ($\mu C$)} & \makecell{\textbf{Electrons} \\ $U_1$} & \makecell{\textbf{Electrons} \\ $D_1$} & \makecell{\textbf{Average} \\ \textbf{Electrons}}\\
\hline
3892 & 5568.68 & 6015.81 & 5792.25 & 3.47608e+16 & 3.75519e+16 & 3.61564e+16 \\
3893 & 118261 & 118016 & 118138 & 7.38207e+17 & 7.36678e+17 & 7.37443e+17 \\
3894 & 136502 & 138131 & 137316 & 8.5207e+17 & 8.62243e+17 & 8.57157e+17 \\
4073 & 7656.35 & 7654.74 & 7655.54 & 4.77924e+16 & 4.77824e+16 & 4.77874e+16 \\
4074 & 251551 & 251227 & 251389 & 1.57023e+18 & 1.56821e+18 & 1.56922e+18 \\
4075 & 280417 & 280017 & 280217 & 1.75042e+18 & 1.74792e+18 & 1.74917e+18 \\
\hline
\textbf{Totals} & \textbf{799956} & \textbf{801062} & \textbf{800509} & \textbf{4.99348e+18} & \textbf{5.00039e+18} & \textbf{4.99693e+18} \\
\hline
\end{tabular}
\caption{\bf{Charge Accumulated per Run}}
\label{tab:charges}
\end{table}

\subsection{Dead-time Correction}
\label{ssec:dead-time}

Now we have measured how many electrons total are incident on our target. However, we have not yet accounted for the electronic dead-time discussed in Section ~\ref{sec:daq} and Equation ~\ref{eq:lt}. While the electronics discard some valid trigger events because the system is busy processing the previous event there are still scalers that record every trigger created regardless of if it is recorded by the main DAQ. This means that the dead-time of the system can be calculated by taking the ratio of the total triggers recorded by the DAQ electronics to the total hardware triggers recorded by the scalers. Table ~\ref{tab:dead-time} contains the dead-time for each run as well as the weighted average of dead-times for the cumulative runs. The weighted average live-time, one minus dead-time, of 95.27$\%$ is then applied as a correction of $\frac{1}{0.9527}$ to the total number of elastic electrons detected.

\begin{table}[!h]
\centering
\begin{tabular}{|c c c c|}
\hline
%\makecell{\textbf{Absorption}\\ \textbf{Spectrum Shape}} & \textbf{Paint QE} & \makecell{\textbf{Visual}\\ \textbf{Opacity}} \\
\textbf{Run} & \makecell{\textbf{Hardware}\\ \textbf{$T_3$}} & \makecell{\textbf{Electronic}\\ \textbf{$T_3$}} & \textbf{Live-time}\\
\hline
3892 & 49802 & 42982 & 0.86306 \\
3893 & 485367 & 427476 & 0.88073 \\
3894 & - & 311724 & 0.87912* \\
4073 & 106003 & 103741 & 0.97866 \\
4074 & 1124275 & 1102321 & 0.98047 \\
4075 & 1152972 & 1129955 & 0.98004 \\  
\hline
\makecell{\textbf{Weighted}\\ \textbf{Average}} &  &  & \textbf{0.9527} \\
\hline
\end{tabular}
\caption{{\bf{Live-time per Run}} *Note: Run 3894 had no End of Run readout so the live-time is a weighted average of the two runs, 3892 and 3893, taken the same day.}
\label{tab:dead-time}
\end{table}

\subsection{Trigger Efficiency}
\label{ssec:trigger_eff}

The main trigger, $T_3$, for E08-014 was a coincidence of $S_1$, $S_{2m}$, and the $GC$ detectors. However, this trigger is not perfectly efficient. To measure $T_3$'s efficiency we use trigger $T_4$ which is the coincidence of one of either $S_1$ or $S_{2m}$ and the $GC$. The efficiency of $T_3$ can then be calculated by Equation ~\ref{eq:t3_eff} with $PS_{3,4}$ being the prescale value of $T_3$ or $T_4$ and $n_{T_{3,4}}$ being the number of triggers of either $T_3$ or $T_4$. Figure ~\ref{fig:t3_eff} shows the $T_3$ efficiencies for each of the runs, and Table ~\ref{tab:t3_eff} lists these efficiencies along with the weighted average of the runs. 

\begin{equation} \label{eq:t3_eff}
	T_{3_{eff}} = \frac{PS_{T_3}*n_{T_3}}{PS_{T_3}*n_{T_3}+PS_{T_4}*n_{T_4}}
\end{equation}

\begin{figure}[!ht]
\begin{center}
\includegraphics[width=0.9\linewidth]{T3_eff.png}
\end{center}
\caption{
{\bf{T3 Efficiency by Run.}} }
\label{fig:t3_eff}
\end{figure}

\begin{table}[!h]
\centering
\begin{tabular}{|c c|}
\hline
%\makecell{\textbf{Absorption}\\ \textbf{Spectrum Shape}} & \textbf{Paint QE} & \makecell{\textbf{Visual}\\ \textbf{Opacity}} \\
\textbf{Run} & \textbf{$T_3$ Efficiency}\\
\hline
3892 & 0.969217\\
3893 & 0.988504\\
3894 & 0.989435\\
4073 & 0.988796\\
4074 & 0.989009\\
4075 & 0.988678\\   
\hline
\makecell{\textbf{Weighted}\\ \textbf{Average}} & \textbf{0.988577} \\
\hline
\end{tabular}
\caption{{\bf{$T_3$ Efficiency by Run}} }
\label{tab:t3_eff}
\end{table}

\subsection{VDC Single Track Efficiency}
\label{ssec:vdc_eff}

In general the VDCs are very efficient, but it is possible for particles to make several tracks in the VDCs. This can cause issues with track reconstruction. For the analysis of E08-014 only events making a single track in the VDCs that also passed PID cuts were analyzed. This single track efficiency can be calculated with equation ~\ref{eq:vdc_eff}. Where $N_{track=1}$ is the number of events making only a single track and $N_{0 \leq track \leq 4}$ are the number of events producing between one and four tracks. For the LHRS during experiment E08-014 $\epsilon_{VDC}$ was found to be , and Table ~\ref{tab:vdc_eff} gives the breakdown for both detectors by number of tracks seen. For more detailed information on this calibration see ~\cite{Thesis:Ye} where these values were calculated. 

\begin{equation} \label{eq:vdc_eff}
	\epsilon_{VDC} = \frac{N_{track=1}}{N_{0 \leq track \leq 4}}
\end{equation}

\begin{table}[!h]
\centering
\begin{tabular}{|c c c c c c|}
\hline
%\makecell{\textbf{Absorption}\\ \textbf{Spectrum Shape}} & \textbf{Paint QE} & \makecell{\textbf{Visual}\\ \textbf{Opacity}} \\
\textbf{$N_{track}$} & \textbf{0} & \textbf{1} & \textbf{2} & \textbf{3} & \textbf{4}\\
\hline
HRSL & 0.030$\%$ & 99.175$\%$ & 0.743$\%$ & 0.045$\%$ & 0.005$\%$ \\  
HRSR & 0.048$\%$ & 99.360$\%$ & 0.545$\%$ & 0.039$\%$ & 0.007$\%$ \\ 
\hline
\end{tabular}
\caption{{\bf{VDC Track Efficiency for HRSs}} }
\label{tab:vdc_eff}
\end{table}

\subsection{Particle Identification}
\label{ssec:pid}

When a trigger is seen it is important to be able to identify what particle caused the trigger. For this analysis we are only interested in electrons, but even at these kinematics a pion occasionally causes a trigger. 

Now that we have demonstrated that very few pions are contaminating our electron sample we must still account for inefficiencies in the GC. During E08-014 one of the ten PMTs in the LHRS GC was slightly inefficient and this PMT happens to fall near the kinematics of this data. To study the GC efficiency we can use $T_3$ the main trigger which was a coincidence of $S_1$, $S_{2m}$, and the $GC$ detectors as well as $T_7$ which is a coincidence of $S_1$ and $S_{2m}$. Thus the ratio of $T_3$ + $T_7$ to $T_7$ will yield the GC efficiency for these E08-014 runs. 

Before we find the GC efficiency let us first place some reasonable kinematic and acceptance cuts on the combined data of our six experimental runs (see ~\ref{ssec:cuts} for details on the cut values). Figure ~\ref{fig:gc_eff} shows a plot of $T_7$ on the left and $T_3$ + $T_7$ on the right with physics cuts. These cuts are fairly strict and include a cut on $X_{Bj}$ so there are few events. Still the events form the elastic band shape we would expect. Taking the ratio we find that $\epsilon_{GC} = 196/203 = 0.966$. This is still quite efficient accounting for the slightly inefficient PMT. Knowing the GC inefficiency allows us to scale the final cross section value by 203/196 to correct for this inefficiency.

\begin{figure}[!ht]
\begin{center}
\includegraphics[width=0.9\linewidth]{GC_Eff.png}
\end{center}
\caption{
{\bf{GC Efficiency.}} The left plot is $T_7$ with physics cuts and the right is $T_3$ + $T_7$ with physics cuts.}
\label{fig:gc_eff}
\end{figure}