% Introduction
\chapter{Data Analysis} % Main chapter title
\label{ch:analysis}

\section{Overview}
\label{sec:analysis_overview}

This chapter will explore the various analyses used to extract a $^3$He elastic cross section. This discussion will include all measurements required to extract a differential cross section, the corrections for efficiency losses for these values, the physics cuts applied to the data, and an estimate of the uncertainties. Also discussed will be the Monte Carlo software used to create an artificial elastic electron spectrum as well as the software used to calculate radiative corrections to this data.

\section{Experimental Cross Section}
\label{sec:exp_xs}

The theoretical origins of a differential cross section were explored in ~\ref{sec:xs}. However, this derivation is not particularly intuitive for extracting an actual experimental cross section. In practice extracting a cross section is essentially an exercise in electron counting. Let us now write the cross section as an experimentalist understands it as in Equation ~\ref{eq:exp_xs}. Each of these variables will be examined in detail later in this chapter.

\begin{equation} \label{eq:exp_xs}
	\left(\frac{d\sigma}{d\Omega}\right)_{exp} = \frac{ps*N_e}{N_{in}*\rho *LT* \epsilon_{det}} \frac{1}{\Delta\Omega\Delta P \Delta Z}
\end{equation}

Here $ps$ represents the prescale value of the given trigger being examined. $N_e$ is the number of electrons detected by the experiment that survive physics cuts, like particle identification, and acceptance cuts. $N_{in}$ is the number of electrons incident on the target, and can be calculated from the cumulative beam charge. $\rho$ is the target's density. $LT$ is the livetime correction which accounts for events missed due to electronic dead-time in the DAQ. $\epsilon_{det}$ represents the product of all of the detector efficiencies such as the GC, VDC single track, EM calorimeter, and trigger efficiencies. $\Delta\Omega$ represent the solid angle fraction covered by the spectrometer after acceptance cuts. $\Delta P$ is the momentum acceptance seen by the spectrometer after acceptance cuts. Finally $\Delta Z$ represents the length of the target seen by the spectrometer.

\subsection{Beam Charge}
\label{ssec:charge}

As the the cross section extraction is essentially an exercise in electron counting let us begin by finding the charge of the electron beam during any given experimental run. This process begins with the U and D BCMs measuring the beam current prior to the beam striking the target. These BCMs require are calibrated according to Equation ~\ref{eq:bcms}, and the constants can be found in Table ~\ref{tab:bcm_constants}. The BCM calibration for experiment E08-014 can be read about more in ~\cite{bcm_calibration} where these equations are found. The final constants in the table were updated after the analysis in ~\cite{bcm_calibration}, and were provided via private communication with Dien Nguyen.

\begin{equation} \label{eq:bcms}
	\langle I_{beam} \rangle = \frac{\frac{scalar}{time}-K'_{offset}}{C_{V-to-F}}
\end{equation}

\begin{table}[!h]
\centering
\begin{tabular}{|c c c|}
\hline
%\makecell{\textbf{Absorption}\\ \textbf{Spectrum Shape}} & \textbf{Paint QE} & \makecell{\textbf{Visual}\\ \textbf{Opacity}} \\
\textbf{BCM} & \textbf{$K'_{offset}$} & \textbf{$C_{V-to-F}$} \\
\hline
$U_1$ & 200 & 1035 \\
$D_1$ & 37 & 1263\\
\hline
\end{tabular}
\caption{\bf{BCM Calibration Constants for E08-014}}
\label{tab:bcm_constants}
\end{table}

The average current, $\langle I_{beam} \rangle$, is thus a product of the number of counts in the $scaler$ measuring beam current in a certain period of $time$ tracked by a clock scaler, and then modified by the two calibration constants $K'_{offset}$ and $C_{V-to-F}$. Now that we have a current we can calculate the charge, $Q$, of the beam during a run by Equation ~\ref{eq:charge}. We can then divide the charge by the elementary charge of the electron, $e$, to find the total number of electrons incident on our target, $N_{in}$, as in Equation ~\ref{eq:electrons}.

\begin{equation} \label{eq:charge}
	Q = \langle I_{beam} \rangle *time
\end{equation}

\begin{equation} \label{eq:electrons}
	N_{in} = \frac{Q}{e}
\end{equation}

Figure ~\ref{fig:bcms_4074} shows the current as measured by the $U_1$ and $D_1$ BCMs for experimental run 4074. During E08-014 the target beam current 120 $\mu$A. When the beam `trips' turning off and when the beam is being brought online it tends to be somewhat unstable. These events can be seen in the trailing edges and leading edges of the gaps in the BCM current measurement. Due to the beam's instability at these times measurements taken during them are discarded by placing cuts on the current spectrum represented by the red lines in ~\ref{fig:bcms_4074}. These lines are placed two scaler readouts, about 4 seconds per readout, after the BCMs register 90$\%$ or greater of the 120 $\mu$A operating current. This analysis was performed for each run of KIN 3.2 and the results are shown in Table ~\ref{tab:charges}.

\begin{figure}[!ht]
\begin{center}
\includegraphics[width=1.1\linewidth]{BCMs_4074.png}
\end{center}
\caption{
{\bf{BCM Readouts for Run 4074.}} These plots show the $U_1$ and $D_1$ BCM measurements for run 4074. The cuts are applied two scaler readouts after the current rises above or below the 90$\%$ threshold of the 120 $\mu$A target current.}
\label{fig:bcms_4074}
\end{figure}

\begin{table}[!h]
\centering
\begin{tabular}{|c c c c c c c|}
\hline
%\makecell{\textbf{Absorption}\\ \textbf{Spectrum Shape}} & \textbf{Paint QE} & \makecell{\textbf{Visual}\\ \textbf{Opacity}} \\
\textbf{Run} & \makecell{\textbf{Charge}\\ \textbf{$U_1$ ($\mu C$)}} & \makecell{\textbf{Charge}\\ \textbf{$D_1$ ($\mu C$)}} & \makecell{\textbf{Average}\\ \textbf{Charge}\\ ($\mu C$)} & \makecell{\textbf{Electrons} \\ $U_1$} & \makecell{\textbf{Electrons} \\ $D_1$} & \makecell{\textbf{Average} \\ \textbf{Electrons}}\\
\hline
3892 & 5568.68 & 6015.81 & 5792.25 & 3.47608e+16 & 3.75519e+16 & 3.61564e+16 \\
3893 & 118261 & 118016 & 118138 & 7.38207e+17 & 7.36678e+17 & 7.37443e+17 \\
3894 & 136502 & 138131 & 137316 & 8.5207e+17 & 8.62243e+17 & 8.57157e+17 \\
4073 & 7656.35 & 7654.74 & 7655.54 & 4.77924e+16 & 4.77824e+16 & 4.77874e+16 \\
4074 & 251551 & 251227 & 251389 & 1.57023e+18 & 1.56821e+18 & 1.56922e+18 \\
4075 & 280417 & 280017 & 280217 & 1.75042e+18 & 1.74792e+18 & 1.74917e+18 \\
\hline
\textbf{Totals} & \textbf{799956} & \textbf{801062} & \textbf{800509} & \textbf{4.99348e+18} & \textbf{5.00039e+18} & \textbf{4.99693e+18} \\
\hline
\end{tabular}
\caption{\bf{Charge Accumulated per Run}}
\label{tab:charges}
\end{table}

\subsection{Dead-time Correction}
\label{ssec:dead-time}

Now we have measured how many electrons total are incident on our target. However, we have not yet accounted for the electronic dead-time discussed in Section ~\ref{sec:daq} and Equation ~\ref{eq:lt}. While the electronics discard some valid trigger events because the system is busy processing the previous event there are still scalers that record every trigger created regardless of if it is recorded by the main DAQ. This means that the dead-time of the system can be calculated by taking the ratio of the total triggers recorded by the DAQ electronics to the total hardware triggers recorded by the scalers. Table ~\ref{tab:dead-time} contains the dead-time for each run as well as the weighted average of dead-times for the cumulative runs. The weighted average live-time, one minus dead-time, of 95.27$\%$ is then applied as a correction of $\frac{1}{0.9527}$ to the total number of elastic electrons detected.

\begin{table}[!h]
\centering
\begin{tabular}{|c c c c|}
\hline
%\makecell{\textbf{Absorption}\\ \textbf{Spectrum Shape}} & \textbf{Paint QE} & \makecell{\textbf{Visual}\\ \textbf{Opacity}} \\
\textbf{Run} & \makecell{\textbf{Hardware}\\ \textbf{$T_3$}} & \makecell{\textbf{Electronic}\\ \textbf{$T_3$}} & \textbf{Live-time}\\
\hline
3892 & 49802 & 42982 & 0.86306 \\
3893 & 485367 & 427476 & 0.88073 \\
3894 & - & 311724 & 0.87912* \\
4073 & 106003 & 103741 & 0.97866 \\
4074 & 1124275 & 1102321 & 0.98047 \\
4075 & 1152972 & 1129955 & 0.98004 \\  
\hline
\makecell{\textbf{Weighted}\\ \textbf{Average}} &  &  & \textbf{0.9527} \\
\hline
\end{tabular}
\caption{{\bf{Live-time per Run}} *Note: Run 3894 had no End of Run readout so the live-time is a weighted average of the two runs, 3892 and 3893, taken the same day.}
\label{tab:dead-time}
\end{table}

\subsection{Trigger Efficiency}
\label{ssec:trigger_eff}

The main trigger, $T_3$, for E08-014 was a coincidence of $S_1$, $S_{2m}$, and the $GC$ detectors. However, this trigger is not perfectly efficient. To measure $T_3$'s efficiency we use trigger $T_4$ which is the coincidence of one of either $S_1$ or $S_{2m}$ and the $GC$. The efficiency of $T_3$ can then be calculated by Equation ~\ref{eq:t3_eff} with $PS_{3,4}$ being the prescale value of $T_3$ or $T_4$ and $n_{T_{3,4}}$ being the number of triggers of either $T_3$ or $T_4$. Figure ~\ref{fig:t3_eff} shows the $T_3$ efficiencies for each of the runs, and Table ~\ref{tab:t3_eff} lists these efficiencies along with the weighted average of the runs. 

\begin{equation} \label{eq:t3_eff}
	T_{3_{eff}} = \frac{PS_{T_3}*n_{T_3}}{PS_{T_3}*n_{T_3}+PS_{T_4}*n_{T_4}}
\end{equation}

\begin{figure}[!ht]
\begin{center}
\includegraphics[width=0.9\linewidth]{T3_eff.png}
\end{center}
\caption{
{\bf{T3 Efficiency by Run.}} }
\label{fig:t3_eff}
\end{figure}

\begin{table}[!h]
\centering
\begin{tabular}{|c c|}
\hline
%\makecell{\textbf{Absorption}\\ \textbf{Spectrum Shape}} & \textbf{Paint QE} & \makecell{\textbf{Visual}\\ \textbf{Opacity}} \\
\textbf{Run} & \textbf{$T_3$ Efficiency}\\
\hline
3892 & 0.969217\\
3893 & 0.988504\\
3894 & 0.989435\\
4073 & 0.988796\\
4074 & 0.989009\\
4075 & 0.988678\\   
\hline
\makecell{\textbf{Weighted}\\ \textbf{Average}} & \textbf{0.988577} \\
\hline
\end{tabular}
\caption{{\bf{$T_3$ Efficiency by Run}} }
\label{tab:t3_eff}
\end{table}

\subsection{VDC Single Track Efficiency}
\label{ssec:vdc_eff}

In general the VDCs are very efficient, but it is possible for particles to make several tracks in the VDCs. This can cause issues with track reconstruction. For the analysis of E08-014 only events making a single track in the VDCs that also passed PID cuts were analyzed. This single track efficiency can be calculated with equation ~\ref{eq:vdc_eff}. Where $N_{track=1}$ is the number of events making only a single track and $N_{0 \leq track \leq 4}$ are the number of events producing between one and four tracks. For the LHRS during experiment E08-014 $\epsilon_{VDC}$ was found to be , and Table ~\ref{tab:vdc_eff} gives the breakdown for both detectors by number of tracks seen. For more detailed information on this calibration see ~\cite{Thesis:Ye} where these values were calculated. 

\begin{equation} \label{eq:vdc_eff}
	\epsilon_{VDC} = \frac{N_{track=1}}{N_{0 \leq track \leq 4}}
\end{equation}

\begin{table}[!h]
\centering
\begin{tabular}{|c c c c c c|}
\hline
%\makecell{\textbf{Absorption}\\ \textbf{Spectrum Shape}} & \textbf{Paint QE} & \makecell{\textbf{Visual}\\ \textbf{Opacity}} \\
\textbf{$N_{track}$} & \textbf{0} & \textbf{1} & \textbf{2} & \textbf{3} & \textbf{4}\\
\hline
HRSL & 0.030$\%$ & 99.175$\%$ & 0.743$\%$ & 0.045$\%$ & 0.005$\%$ \\  
HRSR & 0.048$\%$ & 99.360$\%$ & 0.545$\%$ & 0.039$\%$ & 0.007$\%$ \\ 
\hline
\end{tabular}
\caption{{\bf{VDC Track Efficiency for HRSs}} }
\label{tab:vdc_eff}
\end{table}

\subsection{Particle Identification}
\label{ssec:pid}

When a trigger is seen it is important to be able to identify what particle caused the trigger. For this analysis we are only interested in electrons, but pions occasionally cause triggers as well. Figure ~\ref{fig:pid_pr_ye} shows a plot of the two pion rejector calorimeters used for PID in E08-014. As discussed in Section ~\ref{ssec:em_cal} electrons will leave large signals in the PRs, and pions much smaller signals. Using this property of the PRs we can show the area almost certain to be electrons shaded blue, and the area we expect to be pions is shaded red. Note that the blue (red) region is not all of the electrons (pions), but triggers in the blue (red) region are almost certainly electrons (pions). The other low energy events not identified as pions are generally delta, or knock-on, electrons which are created when particles strike the metals of the detector frame and spectrometer apertures etc. We can see for this one particular run, that was not part of this analysis, that there is a non-zero pion contamination in this data. 

\begin{figure}[!ht]
\begin{center}
\includegraphics[width=0.9\linewidth]{PID_PR_Ye.png}
\end{center}
\caption{
{\bf{PID with the Pion Rejectors.}} Image from ~\cite{Thesis:Ye}}.
\label{fig:pid_pr_ye}
\end{figure}

Alternatively, we can look at the same data in the GC to identify the particles as in Figure ~\ref{fig:pid_gc_ye}. We discussed how the gas in the GC is chosen for electrons to emit Cherenkov radiation in Section ~\ref{ssec:gc}. Due to this gas choice pions rarely will cause Cherenkov radiation. We can see this difference in ~\ref{fig:pid_gc_ye} by identifying the bulk of the signal, blue, as electrons, and the lowest signals, red, as pions. Again, the blue (red) region is not all of the electrons (pions), but triggers in the blue (red) region are almost certainly electrons (pions). In this data we see the non-zero pion contamination again. Fortunately, as with the PR plots, the pions and electrons are easy to distinguish from one another.

\begin{figure}[!ht]
\begin{center}
\includegraphics[width=1.1\linewidth]{PID_GC_Ye.png}
\end{center}
\caption{
{\bf{PID with the Gas Cherenkov.}} Image from ~\cite{Thesis:Ye}}.
\label{fig:pid_gc_ye}
\end{figure}

Now let us examine the combined six experimental runs used in this analysis using the PRs and GC. Figure ~\ref{fig:pid_pr} shows the PRs for our data with reasonable physics cuts applied to the data. We see the same `cloud' of electrons in the higher PR ADC channels indicating that electron energy is being well measured by the PRs. The red box in this clod shows a similar region to before where we are almost certain to have `good' electrons. However, when we search for the pion cloud in the lower energy channels, represented by the small red box in the lower left, we see very few pions at all. This is a result of KIN 3.2 producing few pions that can be measured by the detectors so our pion contamination is extremely low. The small number of pions can be eliminated by placing a diagonal cut on the data represented by the green line.

\begin{figure}[!ht]
\begin{center}
\includegraphics[width=1.\linewidth]{PID_PR.png}
\end{center}
\caption{
{\bf{PID with the Pion Rejectors.}} }.
\label{fig:pid_pr}
\end{figure}

We can also examine the GC, Figure ~\ref{fig:pid_gc}, as we did previously. Once again we have applied reasonable physics cuts to the data. We see a large bulk signal at its strongest between channels 300 and 500, shown between two red lines, as expected. This shows that we are getting good detecting electrons well with the GC. Looking in the lower ADC channels for pions we again see that there seem to be very few in our data. In fact, below ADC channel 80 we see only 15 events showing that there are very few pions in our sample. As a result when making the final cuts to the data only the diagonal cut on the PRs was used to eliminate possible pions and more likely junk electrons which are mostly likely delta electrons.

\begin{figure}[!ht]
\begin{center}
\includegraphics[width=1.\linewidth]{PID_GC.png}
\end{center}
\caption{
{\bf{PID with the Gas Cherenkov.}} }.
\label{fig:pid_gc}
\end{figure}

Now that we have demonstrated that very few pions are contaminating our electron sample we must still account for inefficiencies in the GC. During E08-014 one of the ten PMTs in the LHRS GC was slightly inefficient and this PMT happens to fall near the kinematics of this data. To study the GC efficiency we can use $T_3$ the main trigger which was a coincidence of $S_1$, $S_{2m}$, and the $GC$ detectors as well as $T_7$ which is a coincidence of $S_1$ and $S_{2m}$. Thus the ratio of $T_3$ + $T_7$ to $T_7$ will yield the GC efficiency for these E08-014 runs as shown by Equation ~\ref{eq:gc_eff}. 

\begin{equation} \label{eq:gc_eff}
	\epsilon_{GC} = \frac{T_3+T_7}{T_7}
\end{equation}

Before we find the GC efficiency let us first place some reasonable kinematic and acceptance cuts on the combined data of our six experimental runs (see ~\ref{ssec:cuts} for details on the cut values). Figure ~\ref{fig:gc_eff} shows a plot of $T_7$ on the left and $T_3$ + $T_7$ on the right with physics cuts. These cuts are fairly strict and include a cut on $X_{Bj}$ so there are few events. Still the events form the elastic band shape we would expect. Taking the ratio we find that $\epsilon_{GC} = 196/203 = 0.966$. This is still quite efficient accounting for the slightly inefficient PMT. Knowing the GC inefficiency allows us to scale the final cross section value by 203/196 to correct for this inefficiency.

\begin{figure}[!ht]
\begin{center}
\includegraphics[width=1.1\linewidth]{GC_Eff.png}
\end{center}
\caption{
{\bf{GC Efficiency.}} The left plot is $T_7$ with physics cuts and the right is $T_3$ + $T_7$ with physics cuts.}
\label{fig:gc_eff}
\end{figure}

\subsection{Target Density}
\label{ssec:density}

As the electron beam passes through the gaseous $^3$He target it causes the target to heat considerably. In the ideal case the target would heat uniformly, but studies of the target boiling effect during E08-014 indicate that the target boiled preferentially based on the target's long axis, $Z_{react}$. Figure ~\ref{fig:boiling_effect} shows events detected along $Z_{react}$ for various beam currents. The two peaks at either end of the plot are the Aluminium endcaps of the target. It is clear from this plot that the density of the gas across the cell is not constant otherwise the event rate would be approximately constant. For a detailed discussion of the target boiling studies see ~\cite{Thesis:Ye} Section 5.4.1 and Appendix D. 

\begin{figure}[!ht]
\begin{center}
\includegraphics[width=1.\linewidth]{Boiling_Effect.png}
\end{center}
\caption[\bf{$^3$He Boiling Effect.}]{
{\bf{$^3$He Boiling Effect.}} Image from ~\cite{Thesis:Ye}.}
\label{fig:boiling_effect}
\end{figure}

The target boiling effects were the result of using a significantly higher beam current than the $^3$He cells were designed to cool. (This higher current did provide the silver lining of helping to create enough elastic electrons for this analysis to be possible.) To better understand the density of the target when the beam was on Silviu Covrig created a computer simulation of the target and the experimental conditions. The computational fluid dynamics (CFD) simulation studied the behavior of $^4$He based on its density, specific heat, thermal conductivity, and viscosity. The simulation also factored in the conditions of the cell such as pressure and temperature which were measured during the experiment ~\cite{density}. The mechanics of the CFD simulation are best described in Covrig's own words, 

\begin{quote}
This method of calculation is called finite volume element, which means that the volume of the target cell is broken into smaller volumes, a process controlled by the size of the mesh. Fluid dynamics equations, transport equations or equations for any scalar/vector of interest are solved on these elements of volume or computational cells and predictions are made for the fields of temperature, velocity, density etc. in the whole volume of the geometry. At any given z-location along the beam line the raster area is, say, 9 mm 2 or 3 mm by 3 mm. If the mesh size was, say, 0.25 mm, then you could expect about 12x12 or 144 volumetric cells at that z-location. For each of these cells the program predicts the velocity of the fluid, its density, temperature etc. at the center of the cell. If you then make a 2D plot of density vs. beam line z-location, the program plots a vertical dotted line that represents the spread of density among the volumetric cells at that z-location. The spread in value is given by what happens at that z-locations over the area of the raster. The fluid may not have constant velocity in the raster area at that z-location, so heating from the beam will decrease more or less its density in volumetric cells that move slower or faster respectively ~\cite{density}."
\end{quote}

This target density study was performed on $^4$He so the results of the study need to be related to $^3$He. This can be done by using the isotopic nature of $^3$He and $^4$He as well as the ideal gas law. We can begin by using the relation between the fractional change in density between $^3$He and $^4$He as given in Equation ~\ref{eq:fractional_density}, where $d\rho_{\left(^3He\right)}$ ($d\rho_{\left(^4He\right)}$) is the fractional change in density for $^3$He ($^4$He), $A_3$ ($A_4$) is the number of nucleons for $^3$He ($^4$He), $P_3$ ($P_4$) is the pressure for $^3$He ($^4$He), and $I_3$ ($I_4$) is the beam current for $^3$He ($^4$He) ~\cite{density}. $d\rho$ is defined as $\frac{\rho-\rho_0}{\rho_0}$, where $\rho$ is the density with the beam on and $\rho_0$ is the density with the beam off ~\cite{density}. 

\begin{equation} \label{eq:fractional_density}
	d\rho_{\left(^3He\right)} = \frac{A_4}{A_3} * \frac{P_4}{P_3} * \frac{I_3}{I_4} * d\rho_{\left(^4He\right)}
\end{equation}

Using the experimental and simulation values we can rewrite the product in Equation ~\ref{eq:fractional_density} as Equation ~\ref{eq:r1}.

\begin{equation} \label{eq:r1}
	R_1 = \frac{A_4}{A_3} * \frac{P_4}{P_3} * \frac{I_3}{I_4} = \frac{4}{3}
\end{equation}

\noindent We can then use the ideal gas law to relate the densities of $^3$He and $^4$He without beam as in Equation ~\ref{eq:density}.

\begin{equation} \label{eq:density}
	\rho_{0\left(^3He\right)} = \frac{A_3}{A_4} * \frac{P_3}{P_4} * \frac{T_4}{T_3} * \rho_{0\left(^4He\right)}
\end{equation}

\noindent Once again we can rewrite the product in Equation ~\ref{eq:density} using the experimental and simulation conditions as Equation ~\ref{eq:r2}.

\begin{equation} \label{eq:r2}
	R_2 = \frac{A_3}{A_4} * \frac{P_3}{P_4} * \frac{T_4}{T_3} = 0.745
\end{equation}

Finally we can solve Equation ~\ref{eq:fractional_density} for $\rho_{\left(^3He\right)}$ by plugging in Equation ~\ref{eq:density}. This gives us the equation describing the $^3$He density with the beam on as shown in Equation ~\ref{eq:3he_density}.

\begin{equation} \label{eq:3he_density}
	\rho_{\left(^3He\right)} = R_1 * R_2 * \rho_{\left(^4He\right)} + \rho_{0\left(^3He\right)} * (1-R_2)
\end{equation}

\noindent We can then get the $^3$He density along the Z axis of the target from the $^4$He CFD simulation results. By integrating over Z we then have the average absolute density of the target. This density was found to be $0.013 \frac{g}{cm^3}$ $\pm 0.0004 \frac{g}{cm^3}$. For a more in depth discussion of this density extraction see ~\cite{density}.

%Due to this unusual density profile our Monte Carlo results need to account for this boiling effect otherwise we will have a less accurate simulation of the elastic events. Figure ~\ref{fig:density_profile} shows the step function, blue line, chosen to approximate the target's measured boiling effect which is shown by the red dots. This was implemented by running the MC with one density and then running the MC again with a second density. These MC data runs were then joined by cutting on each dataset where the step in the density profile is located. The cross section results did not vary significantly when this split density profile was implemented. For a detailed discussion of the target boiling studies see ~\cite{Thesis:Ye} Section 5.4.1 and Appendix D.  

%\begin{figure}[!ht]
%\begin{center}
%\includegraphics[width=1.\linewidth]{Density_Profile.png}
%\end{center}
%\caption{
%{\bf{$^3$He Density Profile.}} }
%\label{fig:density_profile}
%\end{figure}

\subsection{Cuts}
\label{ssec:cuts}

Not all of the electrons we detect necessarily come from regions we are interested in studying. For example, any electrons that originated outside of the target or that have a momentum far away from the momentum setting of the spectrometer are not wanted in our electron sample. To ensure we are studying the electrons scattering from the target we impose a series of physics cuts for the spectrometer's acceptance as well as some of the kinematics. In Section ~\ref{ssec:pid} we already discussed a cut to remove the very small number of pions as well as the delta electrons from our sample. In this section will discuss some of the other major cuts we have imposed on the data which are summarized in Table ~\ref{tab:cuts}. 

\begin{table}[!h]
\centering
\begin{tabular}{|r | l l|}
\hline
%\makecell{\textbf{Absorption}\\ \textbf{Spectrum Shape}} & \textbf{Paint QE} & \makecell{\textbf{Visual}\\ \textbf{Opacity}} \\
\textbf{Cut Type} & \textbf{Minimum} & \textbf{Maximum}\\
\hline
Y Target & -0.03 m & 0.028 m\\ 
$\theta$ & -0.049 rad & 0.042 rad\\ 
$\phi$ & -0.03 rad & 0.03 rad\\ 
dP & -0.02 & 0.03\\ 
\hline
\end{tabular}
\caption{{\bf{Summary of Acceptance Cuts}} }
\label{tab:cuts}
\end{table}

Each of these cuts was made using several techniques. Let us take the Y target ($Y_{tar}=sin\left(\theta_{HRS}\right)Z_{vertex}$) cut as our example. The first technique is a visual assessment of the range of data that we wish to accept. Figure ~\ref{fig:acceptance_y} shows the Y target plot of the experimental data, and the red region shows the range of data we chose to cut on. Notice the two peaks at either end of the data. These peaks are created by the Aluminium end caps and as such are not of interest in the this analysis so they were cut out. 

\begin{figure}[!ht]
\begin{center}
\includegraphics[width=1.1\linewidth]{Acceptance_Y.png}
\end{center}
\caption[\bf{Y Target Acceptance}]{
{\bf{Y Target Acceptance.}} The red acceptance region runs from -0.03 m to 0.028 m.}
\label{fig:acceptance_y}
\end{figure}

The second technique involves taking initially too wide cuts and continually making the cuts smaller until the cross section result remains stable. When cuts are too large they accept irrelevant electrons which will change the cross section, but once the cuts are strict enough to capture only good electrons the cross section remains stable. Cuts were made such that they accept the maximum number of electrons in the region where the cross section is stable independent of the cuts. The acceptance region chosen for Y target runs from -0.03 m to 0.028 m.

Now let us look at the remaining cuts for $\theta$, $\phi$, and dP. Figure ~\ref{fig:acceptance_th} shows a plot of $\theta$ for experiment E08-014. This cut was primarily made visually and runs from -0.042 to 0.049 radians. Figure ~\ref{fig:acceptance_ph} shows a plot of $\phi$ for experiment E08-014. This cut was also primarily made visually, but was also tested with the second method. The cut for $\phi$ runs from -0.03 to 0.03 radians. Finally, Figure ~\ref{fig:acceptance_dp} shows a plot of dP for experiment E08-014. This cut was made mostly using the second method of checking for cross section stability. The dP cut runs from -0.02 to 0.03 where $dP = \frac{P-P_0}{P_0}$ and $P$ is the electron momentum and $P_0$ is the spectrometer's momentum setting.

\begin{figure}[!ht]
\begin{center}
\includegraphics[width=1.1\linewidth]{Acceptance_Theta.png}
\end{center}
\caption[\bf{$\theta$ Acceptance}]{
{\bf{$\theta$ Acceptance.}} The red acceptance region runs from -0.042 radians to 0.049 radians.}
\label{fig:acceptance_th}
\end{figure}

\begin{figure}[!ht]
\begin{center}
\includegraphics[width=1.1\linewidth]{Acceptance_Phi.png}
\end{center}
\caption[\bf{$\phi$ Acceptance}]{
{\bf{$\phi$ Acceptance.}} The red acceptance region runs from -0.03 radians to 0.03 radians.}
\label{fig:acceptance_ph}
\end{figure}

\begin{figure}[!ht]
\begin{center}
\includegraphics[width=1.1\linewidth]{Acceptance_dP.png}
\end{center}
\caption[\bf{dP Acceptance}]{
{\bf{dP Acceptance.}} The red acceptance region runs from -0.02 to 0.03.}
\label{fig:acceptance_dp}
\end{figure}

\subsection{$^3$He Elastic Cross Section Monte Carlo}
\label{ssec:simc}

To extract the final $^3$He elastic cross section value for this analysis a physics simulation Monte Carlo program called SIMC was used. This program is primarily used by JLab's halls A and C to simulate electron scattering experiments. SIMC contains the geometry of the Hall A spectrometers including their various apertures and the materials that comprise them. SIMC uses an event generator to create the physics events resulting from electrons scattering from a given target. These events are transported through the spectrometer based on an optics matrix which allows for the transformation of detector coordinates to target coordinates. 

SIMC requires numerous inputs before generating the initial events. The program must be given ranges in momentum, in-plane angle ($\phi$), out-of-plane angle ($\theta$), beam energy, and spectrometer angle. These acceptances are chosen to match the conditions of the real world Hall A experimental setup. SIMC then randomly and uniformly creates particles in the detector in the provided acceptance ranges. SIMC then calculates the energy of each electron produced and weights each event by the cross section and finally applies radiative corrections (see ~\ref{ssec:rc}) to that event. 

SIMC then transforms these events back through the spectrometer to their reaction vertex in the target. Along the way it applies energy losses due to multiple scattering as well as ionization and Bremsstrahlung. The spectrometer magnets are modelled using the COSY INFINITY program ~\cite{cosy}. This program contains the various transformation matrices required to transport particles between the various coordinate systems and through the spectrometer. While being transported through the spectrometer SIMC checks to be sure that all recorded electrons do not strike the walls of the spectrometer or the non-sensitive elements of the detectors in the stack, and that the electrons pass through all apertures in the spectrometer. SIMC also applies a smearing function to the electrons' VDC positions to match the real world VDCs ~\cite{Thesis:Wang}. 

Let us now compare the results of SIMC to the real experimental data. For all of these plots reasonable physics and acceptance cuts have been applied. Figure ~\ref{fig:simc_phi} shows the SIMC result for $\phi$ (in-plane angle) in red and the blue shows the experimental data. The agreement between SIMC and data seems to be fairly good, although the SIMC data may fall off a little more quickly. Figure ~\ref{fig:simc_theta} shows the SIMC result for $\theta$ (out-of-plane angle) in red and the blue shows the experimental data. Once again the agreement looks decent, but there seems to be some drop off in the SIMC data in the negative $\theta$ direction.  Figure ~\ref{fig:simc_dp} shows the SIMC result for dP in red and the blue shows the experimental data. Here we see the correct shape, but the SIMC data seems to be shifted a bit. This is probably the result of the optics calibration, and is a known issue with SIMC currently.

\begin{figure}[!ht]
\begin{center}
\includegraphics[width=1.\linewidth]{SIMC_vs_Data_Phi.png}
\end{center}
\caption[\bf{SIMC $\phi$ vs. Experimental Data $\phi$}]{
{\bf{SIMC $\phi$ vs. Experimental Data $\phi$.}} The red histogram is the SIMC data and the blue histogram is the experimental data.}
\label{fig:simc_phi}
\end{figure}

\begin{figure}[!ht]
\begin{center}
\includegraphics[width=1.\linewidth]{SIMC_vs_Data_Theta.png}
\end{center}
\caption[\bf{SIMC $\theta$ vs. Experimental Data $\theta$}]{
{\bf{SIMC $\theta$ vs. Experimental Data $\theta$.}} The red histogram is the SIMC data and the blue histogram is the experimental data.}
\label{fig:simc_theta}
\end{figure}

\begin{figure}[!ht]
\begin{center}
\includegraphics[width=1.\linewidth]{SIMC_vs_Data_dP.png}
\end{center}
\caption[\bf{SIMC dP vs. Experimental Data dP}]{
{\bf{SIMC dP vs. Experimental Data dP.}} The red histogram is the SIMC data and the blue histogram is the experimental data.}
\label{fig:simc_dp}
\end{figure}

\subsection{Radiative Corrections}
\label{ssec:rc}

In Chapter ~\ref{ch:elastic} we discussed the lowest order (Born term) Feynman diagram for elastic electron scattering. While this is a good first approximation other diagrams contribute significantly to the cross section and must be accounted for as well. These diagrams can be categorized as external or internal radiative corrections. External corrections are characterized by interactions with other particles that are not the primary scattering source for the electron. These corrections come in the form of Bremsstrahlung radiation and ionization. Bremsstrahlung radiation is the radiation released as photons when electrons are slowed down by the Coulomb fields of particles in the materials the electron is passing through. 

Internal corrections are characterized by the electron interacting with the primary scattering source via the exchange of real or virtual photons. Figure ~\ref{fig:rc} shows the additional first order diagrams for internal corrections along with the Born term diagram. Once more we can break these diagrams in to the two categories of elastic and inelastic corrections. The elastic corrections exchange only virtual photons shown in Figure ~\ref{fig:rc} b), c), and d). The inelastic corrections emit real photons as seen in e) ~\cite{Thesis:Wang}.

\begin{figure}[!ht]
\begin{center}
\includegraphics[width=1.\linewidth]{Radiative_Corrections.png}
\end{center}
\caption[\bf{Born and Lowest Order Radiative Correction Diagrams}]{
{\bf{Born and Lowest Order Radiative Correction Diagrams.}} Image from ~\cite{Thesis:Wang}.}
\label{fig:rc}
\end{figure}

The radiative corrections for E08-014 were calculated by the program XEMC. For a detailed discussion of the radiative corrections applied by XEMC see ~\cite{Article:RC} and ~\cite{Article:RC2}. This program is derived from a previous program called RadCor which is detailed in ~\cite{Thesis:Yao} and ~\cite{Thesis:Slifer}. XEMC uses the built in cross section model to calculate the Born cross section, without radiative corrections, and the radiative cross section. XEMC also uses the Peaking approximation discussed in ~\cite{Article:RC2}. 

These radiative corrections now allow us to compare our Monte Carlo elastic data to our experimental data by means of Equation ~\ref{eq:rc}. Where $\sigma^{Exp}_{Rad}$ is the experimental cross section, $\sigma^{Exp}_{Born}$ is the experimental Born cross section, $\sigma^{Model}_{Rad}$ is the model radiative cross section, and $\sigma^{Model}_{Born}$ is the model Born cross section ~\cite{Thesis:Ye}. Figure ~\ref{fig:simc_elastics} shows the $^3$He elastically scatter electron spectrum in $X_{Bj}$. The tail below the elastic peak is due to the radiative corrections, and demonstrates that the corrections are working as intended.

\begin{equation} \label{eq:rc}
	\sigma^{Exp}_{Born} = \sigma^{Exp}_{Rad} \frac{\sigma^{Model}_{Rad}}{\sigma^{Model}_{Born}}
\end{equation}

\subsection{Aluminium Background Subtraction}
\label{ssec:al}

Many of the events detected by the experiment are derived from electrons scattering off of the Aluminium walls of the target cell. These are not events we want in our cross section measurement so steps are taken to subtract this background out of the sample. This is done by means of a `dummy' Al cell which is an empty Al cell. This dummy cell is placed in the beam and the electrons scattered from it are measured. Since the cell is empty the dummy shows how the Al background of the actual target cell will look. Finally after a few corrections the Al dummy cell data can be subtracted from the production data to give an Al background subtracted result which removes the Al contamination.

The Al background must be scaled to account for the different amount of beam charge that the dummy data collected as compared to the production data. For this analysis the dummy cell result must be multiplied by 21.2708 to match the charge of the experimental runs. The dummy cell also had different thickness walls Al thicknesses as compared to the target cell. To compensate for the larger dummy Al thickness the dummy data was scaled by another 0.1979 times.

Finally when subtracting the Al background found with the dummy cell from the $^3$He cell one must take into account the different radiative corrections due to the Al for the two cells due to the Al thickness differences. The Al dummy cell gives a radiative correction ratio of $\sigma^{Model}_{Born}/\sigma^{Model}_{Rad} = 1.823$ and the $^3$He target cell gives a ratio of $\sigma^{Model}_{Born}/\sigma^{Model}_{Rad} = 1.467$. Lastly, the Al dummy background is scaled by the ratio of the dummy radiative corrections, 1.823, to the target cell corrections, 1.467. The resultant scaled Al background is shown in red in Figure ~\ref{fig:al}. This red histogram is then subtracted from the blue histogram that represents the $^3$He production data that is still contaminated by Aluminium. 

\begin{figure}[!ht]
\begin{center}
\includegraphics[width=1.1\linewidth]{Al_Background.png}
\end{center}
\caption[\bf{Scaled Aluminium Background}]{
{\bf{Scaled Aluminium Background.}} The red histogram shows the scaled Al background, and the blue histogram shows the $^3$He production data before the Al background is subtracted.}
\label{fig:al}
\end{figure}

Since the production data was taken on $^3$He we do not anticipate seeing events at $X_{Bj} > 3$ since that is the elastic peak. However, there are clearly events in this region above the elastic peak. These events must be coming from something other than the $^3$He gas. Notice that above the $^3$He elastic peak the scaled Al background closely matches the production background. This is strong evidence that these events are coming from scattering off of the Al of the cell. Once the Al background is subtracted out most of these events disappear leaving us with only the desired electrons that were scattered from the $^3$He target. This can be seen in Figure ~\ref{fig:combined}.

\subsection{Electron Yields to Cross Sections}
\label{ssec:yield}

At this point we have made numerous cuts and corrections to our data with the goal of getting a clean electron sample scattered from the $^3$He target. Now we must use this data to obtain a cross section. We showed in Figure ~\ref{fig:elastic_xbj} in Section ~\ref{sec:x>2} that experiment E08-014 definitely captured elastically scattered electrons. We can determine how many electrons we detected after we apply all of the physics cuts and corrections to the $X_{Bj}$ plot the production runs.

To find the number of elastically scattered electrons we need to be able to count how many electrons are in the elastic peak. To accomplish this we need a method to fit the data such that the area of the elastic peak can be measured. The $X_{Bj}$ plot can be broken down in to two areas, the quasielastic region and the elastic peak. We want to use one function to describe each region and then combine these two functions and fit the relevant region of $X_{Bj}$ with this combined fit. 

Two functions immediately spring to mind as good candidates for the fit. Recall that the $X_{Bj}$ plot is logarithmic, and notice that the quasielastic region sloping down to the elastic peak appears roughly linear on this log plot. This indicates that an exponential function will likely yield reasonable fit results for the quasielastic region. As we are only interested in elastic electrons in the end we should only need to fit the plot in the region of the elastic peak. To fit the quasielastic region near the elastic peak a two parameter exponential function as in Equation ~\ref{eq:exp} was chosen with $P_0$ and $P_1$ being the free parameters.

\begin{equation} \label{eq:exp}
	F_{exp} = e^{\left( P_0+P_1*x \right)}
\end{equation}

Next we want to fit the elastic peak around $X_{Bj}=3$. While a Poisson distribution may technically better describe the distribution of the electrons in the elastic peak a simple Gaussian describes the peak equally well. As such, Equation ~\ref{eq:gaus} was chosen to fit the elastic peak with free parameters $P_0$, $P_1$, and $P_2$. Finally by summing Equations ~\ref{eq:exp} and ~\ref{eq:gaus} we have a combined fit that can be used to fit the region of $X_{Bj}$ around the elastic peak. This combined fit is given in Equation ~\ref{eq:combined} with free parameters $P_0$ through $P_4$.

\begin{equation} \label{eq:gaus}
	F_{Gaus} = P_0 e^{\left( \frac{-1}{2} \left( \frac{x-P_1}{P_2} \right)^2 \right)}
\end{equation}

\begin{equation} \label{eq:combined}
	F_{combined} =e^{\left( P_0+P_1*x \right)} + P_2 e^{\left( \frac{-1}{2} \left( \frac{x-P_3}{P_4} \right)^2 \right)}
\end{equation}

Applying the combined fit to the $X_{Bj}$ plot after applying physics cuts, corrections, and the Aluminium background subtraction we get the result seen in Figure ~\ref{fig:combined}, where the solid blue line is the combined fit. The combined fit seems to be doing a good job of capturing the quasielastic region before the elastic peak while also locating the elastic peak well. The fit is allowed to extend slightly beyond the peak for the purpose of creating some analysis histograms, but this extended region does not influence the number of electrons found in the elastic peak. We can also see that as we discussed in Section ~\ref{ssec:al} once the Al background is subtracted out there are almost no events above the $^3$He elastic peak as we expect. 

\begin{figure}[!ht]
\begin{center}
\includegraphics[width=1.1\linewidth]{Combined_Fit_Xbj.png}
\end{center}
\caption[\bf{Combined Fit of $X_{Bj}$ for E08-014}]{
{\bf{Combined Fit of $X_{Bj}$ for E08-014.}} The histogram is the plot of all of the E08-014 $^3$He production runs containing elastic data after applying physics cuts, corrections, and the Aluminium background subtraction. The solid blue line is the combined fit of an exponential, for the quasielastic region, and a Gaussian, for the elastic region.}
\label{fig:combined}
\end{figure}

The number of electrons under the Gaussian part of the combined fit is then the number of elastic electrons detected by the experiment prior to some corrections. After physics and acceptance cuts, but before live-time, GC efficiency, Trigger efficiency, and VDC efficiency corrections, experiment E08-014 detected ***565 electrons elastically scattered from $^3$He. Table ~\ref{tab:corrections} shows a summary of the correction values applied to the number of electrons detected. Multiplying the uncorrected yield by all of these corrections factors we find the experimental yield of $^3$He elastically scattered electrons to be ***627 electrons.

\begin{table}[!h]
\centering
\begin{tabular}{|r | l l|}
\hline
\textbf{Correction Type} & \textbf{Efficiency ($\%$)} & \textbf{Correction Factor}\\
\hline
Live-time & 95.27 & 1.050 \\ 
GC Efficiency & 96.55 & 1.036 \\ 
Trigger Efficiency & 98.858 & 1.0155 \\ 
VDC Efficiency & 99.175 &  1.0083\\ 
\hline
\end{tabular}
\caption{{\bf{Summary of Correction Factors}} }
\label{tab:corrections}
\end{table}

Now we need to determine how this electron yield from the production data corresponds to a cross section value. To do this we will use the Monte Carlo simulation program SIMC discussed in ~/ref{ssec:simc}. SIMC has a built in model of the $^3$He cross section that has the correct shape of the form factors and cross section derived from older fits of the $^3$He world data ~\cite{Article:Amroun}. We use SIMC to simulate elastically scattered electron data off of $^3$He corresponding to the same energy, angle, charge, and acceptance cuts as our experiment. This purely elastic data is shown in Figure ~\ref{fig:simc_elastics}. The tail below the elastic peak is due to radiative corrections.

\begin{figure}[!ht]
\begin{center}
\includegraphics[width=1.1\linewidth]{SIMC_Elastics.png}
\end{center}
\caption[\bf{SIMC Elastically Scattered Electrons}]{
{\bf{SIMC Elastically Scattered Electrons.}} These electrons were generated with the same kinematics, acceptances, and charge of experiment E08-014.}
\label{fig:simc_elastics}
\end{figure}

The goal at this point is to use the same combined fit we used for the production data to fit the SIMC data for a direct comparison. Clearly, having the elastic events from SIMC is not yet enough to match our experimental data since the SIMC data has no quasielastic events which make up the bulk of our datasets. To make the SIMC data comparable to the production data we need to add in the equivalent quasielastic events. This was done by taking the same type exponential fit from Equation ~\ref{eq:exp} and fitting the quasielastic region below the elastic peak of the production data. This exponential fit was done in the region where SIMC predicts there to be fewer than ten elastic electrons so as to only fit quasielastic data. 

A histogram was then binned to this fit of the quasielastic data and can be seen in ~\ref{fig:QE_background} as the black histogram. Note that the fit is allowed to extend slightly beyond the elastic peak for the purpose of obtaining a good total fit, but this extended region does not influence the number of electrons found in the elastic peak. The new histogram representing only quasielastic data was then summed with the SIMC elastics only histogram in the region before and up to the $^3$He elastic peak. This new combined SIMC and quasielastic histogram is then of the same shape as the production data in the region of interest allowing it to be fit with Equation ~\ref{eq:combined}. The area under the Guassian portions of the combined fits is then directly proportional to the cross section values for SIMC and the production data. 

\begin{figure}[!ht]
\begin{center}
\includegraphics[width=1.1\linewidth]{Quasielastic_Background.png}
\end{center}
\caption[\bf{Histogram Binned to Fit of Quasielastic Background}]{
{\bf{Histogram Binned to Fit of Quasielastic Background.}} The black histogram is binned to the fit of the quasielastic background of $X_{Bj}$ without the elastic events. The blue histogram shows the production data including elastics with physics cuts for comparison.}
\label{fig:QE_background}
\end{figure}

While the shape of the form factors and cross section built into SIMC is correct the magnitude at Q$^2 \approx$ 34.2 fm$^{-2}$ is likely off. This is why the SIMC elastic electron yield doesn't perfectly match the experimental data electron yield. So to find the cross section value of the production data we scale the SIMC elastic data by a constant magnitude up or down until the area of the Gaussian of the combined SIMC fit, the elastic electron yield, matches the area of the Gaussian portion of the combined fit of the production data. When the two Gaussian areas of the combined fit match the electron yields of SIMC and the experimental data then match meaning the cross sections are equivalent. This matching of the SIMC yield to the Production yield leaves us with the scale factor, $C_{SIMC}$, we applied to the SIMC data to match the real data. 

Since the Gaussian areas of the combined fits and the cross sections are directly proportional, and we have matched the Gaussian areas of SIMC and experimental data, we can multiply the cross section value in SIMC by $C_{SIMC}$ to find the cross section value of the production data. Figure ~\ref{fig:final_xs} shows the $X_{Bj}$ plot for the production data and it's combined exponential and Gaussian fit in blue as well as the SIMC elastics summed with the fitted quasielastic background histogram with its combined fit in red. 

\begin{figure}[!ht]
\begin{center}
\includegraphics[width=1.1\linewidth]{Peak_Matched_Xbj_Fits.png}
\end{center}
\caption[\bf{Elastic Peak Fits of $^{3}$He Production Runs and SIMC Elastics Summed with QE Background Fit}]{
{\bf{Elastic Peak Fits of $^{3}$He Production Runs and SIMC Elastics Summed with QE Background Fit.}} The blue histogram shows the production data for E08-014 with physics cuts, and it's elastic peak is fit by the blue line. The red histogram is the sum of the SIMC elastics histogram ~\ref{fig:simc_elastics} and the histogram binned to the fit of the quasielastic background \ref{fig:QE_background}, and it's elastic peak is fit by the red line.}
\label{fig:final_xs}
\end{figure}

Table ~\ref{tab:peak_pars} lists the parameters of the total fits to the production data and yield matched SIMC data. The SIMC elastic peak is slightly wider than the production peak indicating that the smearing function in SIMC may be tuned too high. The slight offset between the two elastic peaks is likely due to a small offset in $Z_{target}$, however this should only minimally influence the area under the elastic peak and thus not significantly change the cross section. 

For this analysis the scale factor needed to match the yields, $C_{SIMC}$, was found to be 1.01984. This means that the model cross section needed to be increased by 1.984$\%$ to match the experimental data. When this adjustment is made the model built into SIMC will then yield the cross section for elastic scattering off of $^3$He at our kinematics. This cross section is found to be 1.335 * 10$^{-10}$ fm$^2$/sr or 1.335 * 10$^{-6}$ $\mu$b/sr.

\begin{table}[!h]
\centering
\begin{tabular}{|c | l l l l l|}
\hline
\makecell{\textbf{Combined Fit Par:}\\ \textbf{Individual Fit Par:}} & \makecell{\textbf{P$_0$}\\ \textbf{Exp. P$_0$}} & \makecell{\textbf{P$_1$}\\ \textbf{Exp. P$_1$}} & \makecell{\textbf{P$_2$}\\ \textbf{Gaus. P$_0$}} & \makecell{\textbf{P$_3$}\\ \textbf{Gaus. P$_1$}} & \makecell{\textbf{P$_4$}\\ \textbf{Gaus. P$_2$}}\\
\hline
Production Data: & 16.9593 & -4.38377 & 297.297 & 3.01946 & -0.0151706 \\ 
\makecell{SIMC Elastics Plus\\ QE Background Fit}: & 16.3564 & -4.11888 & 282.465 & 3.04182 & -0.0177163 \\ 
\hline
\end{tabular}
\caption{{\bf{Combined Exponential and Gaussian Fit Parameters}} }
\label{tab:peak_pars}
\end{table}

As a test of the scale factor ($C_{SIMC}$) applied to the SIMC model cross section we can remake the SIMC elastic histogram with the SIMC model multiplied by $C_{SIMC}$. Once a new set of elastic events are generated we can rerun the yield matching code with the new Monte Carlo results. If our scale factor applied to the cross section is correct we would expect the yields of the production data and the new SIMC elastics summed with the QE background to exactly match without any scaling needed (i.e. $C_{SIMC}=1$). This test was performed and the electron yields were found to match with $C_{SIMC}=1$ as desired.

\subsection{Bin Centering Corrections}
\label{ssec:bin_cor}

Now that we have found the magnitude of our $^3$He elastic cross section we must ask, ``Where should we place this data point in Q$^2$?" The immediately intuitive answer to this question is that we should place this point at the center of our bin in Q$^2$. However, this is not the correct place for our data point because the shape of the cross section across the range of our Q$^2$ bin is not linear.
 
To find the correct location for our data point in Q$^2$ we will follow the procedure laid out in ~\cite{Article:data_placement}. To see the issue with placing our point at the center of our bin let us look at a plot of our cross section at 3.356 GeV seen in Figure ~\ref{fig:xs_bin}. Note that the specific form factors chosen for this cross section are chosen as a representative fit for the group of `good' sum of Gaussian fits discussed in Chapter ***~\ref{Ch:SOG}. A good fit here was determined by having a low $\chi^2$ value and more importantly the physical characteristics we would expect from a form factor. 

\begin{figure}[!ht]
\begin{center}
\includegraphics[width=1.1\linewidth]{XS_3356MeV_Representative.png}
\end{center}
\caption[\bf{$^3$He Elastic Cross Section at 3.356 GeV}]{
{\bf{$^3$He Elastic Cross Section at 3.356 GeV}} This cross section was produced using a representative sit from the sum of Gaussians fits discussed in Chapter ***~\ref{Ch:SOG}}
\label{fig:xs_bin}
\end{figure}

We know our point should be located somewhere in the neighborhood of Q$^2 \approx$35 simply from the kinematics of the experiment. Looking around Q$^2 \approx$35 in Figure ~\ref{fig:xs_bin} the plot appears linear. However, the plot is in a log scale indicating that the actual shape is an exponential. To see the true shape of the cross section let us first define our bin in Q$^2$ and then zoom in while removing the log on the Y-axis. 

To find our bin size in Q$^2$ we can use our cuts on the in plane angle $\phi$. Since we know the beam energy we can calculate Q$^2$ using Equations ~\ref{eq:Q^2} and ~\ref{eq:E'} knowing that $\phi$ is the analyzer variable for the arctangent of the deviation from the set spectrometer scattering angle. Table ~\ref{tab:bin} shows the size of our analysis bin in various units. Figure ~\ref{fig:xs_bin_zoom} shows the cross section in the region of the analysis bin with the log plot removed to show the true shape.

\begin{table}[!h]
\centering
\begin{tabular}{|c | c c |}
\hline
\textbf{Variable} & \textbf{Bin Minimum} & \textbf{Bin Maximum}\\
\hline
Analyzer $\phi$ & -0.03 & 0.03\\ 
\makecell{Deviation from \\ Spectrometer Angle} & -0.29991 & 0.29991\\
Angle (Radians) & 0.3372 &  0.3972\\
Angle (Degrees) & 19.32 & 22.75\\
Q$^2$ (GeV$^2$) & 1.188 & 1.604\\
Q$^2$ (fm$^{-2}$) & 30.55 & 41.24\\
\hline
\end{tabular}
\caption{{\bf{Bin Width at 3.356 GeV and Spectrometer Setting of 21.04$^\circ$}} }
\label{tab:bin}
\end{table}

\begin{figure}[!ht]
\begin{center}
\includegraphics[width=1.1\linewidth]{XS_3356MeV_Representative_Zoom.png}
\end{center}
\caption[\bf{$^3$He Elastic Cross Section at 3.356 GeV Analysis Bin No Log}]{
{\bf{$^3$He Elastic Cross Section at 3.356 GeV Analysis Bin No Log}} This plot shows the same cross section as Figure ~\ref{fig:xs_bin} except zoomed in on the analysis bin region with the log on the Y-axis removed. The two vertical black lines represent the minimum and maximum Q$^2$ of the analysis bin.}
\label{fig:xs_bin_zoom}
\end{figure}

Figure ~\ref{fig:xs_bin_zoom} makes it clear that the cross section across our bin is not linear and thus taking the average of the bin to set our cross section's Q$^2$ value is incorrect. Instead we must account for the fact that the shape of the cross section biases our data towards lower Q$^2$. This can be done by taking a weighted average of the Q$^2$ values in our bin where the weights are the cross section values at each Q$^2$. Performing this calculation we find that the weight average Q$^2$ for our analysis bin is 34.19 fm$^{-2}$. This is a significant deviation from the bin center of 35.90 fm$^{-2}$. We now see that as expected the shape of the cross section requires us to place our data point at a lower Q$^2$ than the bin center.

\subsection{Error Approximation}
\label{ssec:error}