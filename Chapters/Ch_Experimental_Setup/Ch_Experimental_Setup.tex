% Introduction
\chapter{Experimental Setup} % Main chapter title
\label{ch:experiment} % For referencing the chapter elsewhere, use 

\section{Overview}
\label{sec:overview}

The Thomas Jefferson National Accelerator Facility (Jefferson Lab or JLab) located in Newport News Virginia uses a continuous electron beam accelerator facility (CEBAF) to perform electron scattering experiments as a means of studying nuclear structure. Jefferson lab consists of four experimental halls designated Hall A, B, C, and the newly commissioned Hall D as in figure ~\ref{fig:jlab}. The facility is capable of creating electron beams of energies as high as 12 GeV and supplying those electrons to the four halls simultaneously. The 12 GeV capability is a recently completed upgrade and during the time the experiment discussed in this thesis, experiment E08-014, ran the facility was limited to 6 GeV beam energies.  

\begin{figure}[!ht]
\begin{center}
\includegraphics[width=0.7\linewidth]{JLab_Layout.png}
\end{center}
\caption{
{\bf{Thomas Jefferson National Accelerator Facility.}} CEBAF is the ring connecting to the experimental halls. Hall D is now located at the top right of the image. Image from ~\cite{Article:HallA}.}
\label{fig:jlab}
\end{figure}

\section{Experiment E08-014}
\label{sec:x>2}
Experiment E08-014 ran in Jefferson Lab's Hall A in 2011. The experiment used electron scattering to measure the inclusive cross sections of various targets using both of Hall A's high resolution spectrometers (HRSs). E08-014 aimed to compare heavy targets to two and three-nucleon targets to study the short range correlations (SRC) between these two and three-nucleon clusters. To this end inclusive cross sections for $^2$H, $^3$He, $^4$He, $^12$C, $^{40}$Ca, and $^{48}$Ca were measured in the region of 1.1 GeV/c $<$ Q$^2$ $<$ 2.5 GeV/c. This experiment covered the $x_{Bj}$ range encompassing the quasielastic (QE) region up to $x_{Bj}$ greater than 3 ~\cite{Thesis:Ye} ~\cite{src_website}. 

While experiment E08-014 focused on the QE region of electron scattering, one kinematic region, KIN 3.2 in figure ~\ref{fig:kin3.2}, also included elastically scattered electrons. The elastic events can be seen by plotting the scattering angle of the electron versus the scattered electron's energy, $E'$, as shown in figure ~\ref{fig:elastic_band}. The resulting curve (red) gives the elastic scattering band for $^3$He. When this band is compared with the spectrometer's upper and lower acceptance in energy and angle, represented by the black lines, it becomes apparent that the elastic band passes through KIN 3.2, and thus we expect to find $^3$He elastic data in KIN 3.2.  

\begin{figure}[!ht]
\begin{center}
\includegraphics[width=0.7\linewidth]{SRC_Kinematics.png}
\end{center}
\caption{
{\bf{Kinematic Coverage of Experiment E08-014.}} The elastic $^3$He data is located in KIN 3.2. Image from ~\cite{Thesis:Ye}.}
\label{fig:kin3.2}
\end{figure}

\begin{figure}[!ht]
\begin{center}
\includegraphics[width=0.7\linewidth]{Elastic_Band.png}
\end{center}
\caption{
{\bf{Elastic Band for $^3$He.}} It is clear that the red elastic band passes through the box made by the intersecting black lines representing the maximum and minimum spectrometer acceptances in energy and angle for KIN 3.2.}
\label{fig:elastic_band}
\end{figure}

These elastic events were scattered from a gaseous $^3$He target allowing for the extraction of an elastic $^3$He cross section. This new measurement is located in the little studied region of Q$^2 = $ 35 fm$^{-2}$ as seen in figure \ref{fig:jlab_3he} which shows the $^3$He charge form factor, $F_{ch}$. This understudied region is interesting because it has the potential to constrain and improve previous fits of the $^3$He form factors. In particular, high $Q^2$ data like this helps to pin down the magnetic form factor. Equation ~\ref{eq:rosenbluth_long} makes clear that to measure the magnetic form factor's contribution to the cross section large $Q^2$ values and large angles are required. Unfortunately, there are few measurements in the world data of high enough $Q^2$ to understand the magnetic form factor's behavior after its first minima. 

\begin{figure}[!ht]
\begin{center}
\includegraphics[width=0.7\linewidth]{JLab_3He_High_Q2_Charge_FF_Clean_Arrow.png}
\end{center}
\caption{
{\bf{Location of New Elastic Measurement.}} This plot shows the $Q^2$ region of the $^3$He electric form factor, $F_{ch}$, where the new elastic $^3$He measurement is located. Image from ~\cite{Article:Alex}.}
\label{fig:jlab_3he}
\end{figure}

\section{CEBAF}
\label{sec:CEBAF}

Jefferson Lab's Continuous Electron Beam Accelerator Facility (CEBAF) uses superconducting radio frequency (srf) cavities to accelerate electrons to energies up to 12 GeV after a recent upgrade. However, this upgrade was completed after this experiment and as such this section will discuss the 6 GeV era beam before the upgrade and Hall D was built. The accelerated electrons form polarizable continuous wave (cw) beams that can be delivered to up to four scientific halls simultaneously for use in nuclear physics experiments. These beams have a maximum energy of 5.7 GeV and a maximum current of 200 $\mu$A. The beam can be split this current among the three experimental halls in any combination totalling less than the maximum current while providing the maximum energy to each hall. ~\cite{Article:CEBAF}

CEBAF begins creating an electron beam using either a thermionic or polarized gun to inject electrons into the accelerator. The polarized gun produces electrons by illuminating a GaAs cathode crystal with a 1497 MHz diode laser tuned to 780 nm. These electrons then enter the first (North) of two linacs each of which contain 20 cryomodules that accelerate the electrons with a maximum gradient exceeding 7 MeV/m. At the end of the North linac the electrons are bent around a 180$\degree$ bend and enter the South linac passing through 20 more cryomodules. Upon reaching the end of the South linac the beam can be directed into any of the three halls by means of RF separators and septa. If higher energies are desired the beam can be recirculated through the linacs up to four additional times for a maximum of five passes through the accelerator resulting in the maximum energy of ~5.7 GeV  ~\cite{Article:HallA}.

\section{Hall A Beamline}
\label{sec:HallA_beamline}

The distinguishing feature of Jefferson Lab's experimental Hall A are the two High Resolution Spectrometers (HRSs) and their associated detector packages. At a central momentum setting of 4 GeV these two spectrometers provide a momentum resolution better than $\frac{\delta p}{p} = 2*10^{-4}$ as well as a horizontal angular resolution of more than 2 mrad. The spectrometer magnets bend the particles upward into the detector stack using a series of quadrupole and dipole magnets in a QQDQ arrangement ~\cite{Article:HallA}. A side view of Hall A is given in figure ~\ref{fig:halla_side} and a top view is given in ~\ref{fig:halla_top}. Each of the components listed in ~\ref{fig:halla_top} will be discussed individually in the following sections.

\begin{figure}[!ht]
\begin{center}
\includegraphics[width=0.7\linewidth]{Hall_A_Side_View.png}
\end{center}
\caption{
{\bf{Hall A Side View.}} Image from ~\cite{Article:HallA}.}
\label{fig:halla_side}
\end{figure}

\begin{figure}[!ht]
\begin{center}
\includegraphics[width=0.7\linewidth]{Hall_A_Top_View.png}
\end{center}
\caption{
{\bf{Hall A Top View.}} Image from ~\cite{Thesis:Wang}.}
\label{fig:halla_top}
\end{figure}

\subsection{Beam Energy}
\label{ssec:beam_energy}

An accurate measure of the electron beam's energy is necessary to obtain accurate experimental results. The energy of the electron beam was measured using the Arc method laid out in ~\cite{Article:HallA}. This method works by passing the electron beam through a series of quadrupole magnets in the arc section of the beam line and measuring its deflection as shown in figure ~\ref{fig:arc}. The beam's momentum, $p$, is then given by the field integral of the eight quadrupole magnets, $\int \overrightarrow{\rm \textbf{B}} \cdot \overrightarrow{\rm \textbf{d}}l$, divided by the arc bend angle, $\theta$, multiplied by a constant $k=0.299792$ GeV rad T$^{-1}$ m$^{-1}$/c as given in equation ~\ref{eq:arc}. To perform this calculation two measurements are required. The first measurement is of the magnetic field integral of the eight quadrupoles is made based on a ninth reference quadrupole. The second measurement is of the bend angle of the arc which is measured by a set of wire scanners.   

\begin{figure}[!ht]
\begin{center}
\includegraphics[width=0.7\linewidth]{Arc_Method.png}
\end{center}
\caption{
{\bf{Arc Energy Measurement Diagram.}} Image from ~\cite{Thesis:Wang}.}
\label{fig:arc}
\end{figure}

\begin{equation} \label{eq:arc}
	p = k \frac{\int \overrightarrow{\rm \textbf{B}} \cdot \overrightarrow{\rm \textbf{d}}l}{\theta}
\end{equation}

\subsection{Beam Position Monitors}
\label{ssec:bpms}

Along with the beam's energy it's position must also be well known. To assess the beam's position two beam position monitors (BPMs) are located 7.524 m and 1.286 m upstream of the Hall A target ~\cite{Article:HallA}. Figure ~\ref{fig:beamline} shows the layout of the beamline components. Each of these BPMs is made up of four orthogonally oriented antennae place perpendicular to the beam. For beam currents above 1 $\mu$A these antennae produce a signal that is inversely proportional to the beam's distance. The position of the beam is determined by the difference-over-sum technique laid out in ~\cite{bpm1} and ~\cite{bpm2}.

\begin{figure}[!ht]
\begin{center}
\includegraphics[width=0.7\linewidth]{Beamline_Layout.png}
\end{center}
\caption{
{\bf{Layout of Hall A Beamline Components.}} Image from ~\cite{Thesis:Wang}.}
\label{fig:beamline}
\end{figure}

The BPMs are calibrated by wire scanners adjacent to each of them. These wire scanners are independently calibrated periodically with respect to the Hall A coordinates and are accurate to within 200 $\mu$m. The data from the BPMs is recorded in the EPICS database every second and in the data stream every three to four seconds. Each of the eight antennae signals is also recorded in the CODA data stream on an event by event basis. 

\subsection{Raster}
\label{ssec:raster}

The electron beam is generally quite narrow at $<$ 0.3 mm ~\cite{Thesis:Wang}. If this small beam falls on a single part of the target the target can overheat locally causing parts of the target to behave differently than those not struck by the beam. There is also the risk of burning a hole through the target. To avoid these problems the electron beam is rastered meaning that the beam is widened and spread out over a larger area of the target. The fast raster consists of two steering magnets and operates with a frequency of 17-24 kHz. It is located 23 m upstream from the target ~\cite{Article:HallA} as seen in figure ~\ref{fig:beamline}. This system has the capability to spread the beam over several mm in all directions allowing for the target to be illuminated uniformly. 

\subsection{Beam Current Monitors}
\label{ssec:bcms}

The beam current is measured by two beam current monitors (BCMs) which are RF cavity monitors and an Unser located 25 m upstream of the target ~\cite{Article:HallA} as shown in figure ~\ref{fig:beamline}. The Unser measures beam current by, ``a system combining a second harmonic magnetic modulator with an active current transformer in an operational feedback loop, to obtain wide band response down to dc~\cite{Article:Unser}." The two BCM RF cavities are located on either side of the Unser and are tuned to the electron beam frequency of 1.497 GHz. 

These RF cavity monitors then produce a signal that is proportional to the beam current which is recorded by a data acquisition system. Before being recorded the output signals of the two BCMs are each split into three. One of those three signals is amplified by a factor of three and another by a factor of ten. This results in six signals from the two BCMs designated U$_1$, U$_3$, U$_{10}$, D$_1$, D$_3$, and D$_{10}$. The procedure and results for the BCM calibration for experiment E08-014 can be found in ~\cite{bcm_calibration}. 

\section{Target}
\label{sec:target}

Hall A has a cryogenic distribution system (CDS) capable of cooling targets to temperatures of around 5 K to 15 K ~\cite{Article:HallA}. The targets are kept in a vacuum scattering chamber where they are allowed to interact with the electron beam. Inside this chamber is a ladder containing various experimental targets as shown in figure ~\ref{fig:ladder}. The target ladder is divided into three loops which are cryogenically cooled while having their pressure and temperature monitored continuously. Loops one and two contain one each of a 10 cm and 20 cm target cell while loop 3 contains two 20 cm target cells. The target ladder can be moved up and down to place different targets in the path of the electron beam from the counting house ~\cite{Thesis:Ye}.  

\begin{figure}[!ht]
\begin{center}
\includegraphics[width=0.4\linewidth]{Ladder.png}
\end{center}
\caption{
{\bf{Target Ladder for Experiment E08-014.}} Only loops one and two are shown. Image from ~\cite{Thesis:Ye}.}
\label{fig:ladder}
\end{figure}

Experiment E08-014 used a liquid deuterium target as well as a gaseous $^3$He target, examined in this thesis, and a gaseous $^4$He target. The $^3$He target was used in the second run period of the experiment taking place from April 21$^{st}$ to May 15$^{th}$ of 2011. The $^3$He target was located in loop 1 and was cooled to 17 K with a pressure of 211 psia ~\cite{Thesis:Ye}. The entire target layout over the two runs can be seen in figure ~\ref{fig:targets} and the monitoring and control system is shown in figure ~\ref{fig:cryo_controls}.

\begin{figure}[!ht]
\begin{center}
\includegraphics[width=0.7\linewidth]{Targets_Table.png}
\end{center}
\caption{
{\bf{Table of Target Information for Experiment E08-014.}} Image from ~\cite{Thesis:Ye}.}
\label{fig:targets}
\end{figure}

\begin{figure}[!ht]
\begin{center}
\includegraphics[width=0.7\linewidth]{Cryo_Controls.png}
\end{center}
\caption{
{\bf{Cryogentic Monitoring and Control Screen.}} Image from ~\cite{Thesis:Ye}.}
\label{fig:cryo_controls}
\end{figure}

A 30 cm long carbon foil optics target was installed below loop 3. This target contains seven carbon foils spaced 5 cm apart ranging from -15 cm to +15 cm. Electrons scattering off of these foils can be used to calibrate the position of the target from data in collected by the detector package as will be discussed in ~\ref{sec:optics}. Along with these target a 10 cm and a 20 cm dummy target were also installed below the optics target. These dummy targets each contained two thick aluminium foils separated by 10 cm and 20 cm respectively. The dummy targets were used to study the end cap contributions of the target cells. There were also a BeO, $^{12}$C, and an empty target installed below ~\cite{Thesis:Ye}.  

\section{High Resolution Spectrometers}
\label{sec:HRSs}

The high resolution spectrometers (HRSs) are the workhorses of Hall A. They are designated the left HRS (LHRS) and right HRS (RHRS) for the spectrometers on the left and right side of the beam direction respectively. While they were designed to be identical they each have unique features ased on their construction as well as the general wear and tear of age on their components. They have a momentum resolution of 1*10$^{-4}$ from 0.8 Gev to 4 GeV. They use a combination of superconducting quadrupole and dipole magnets in a QQDQ combination to bend the scattered particles through a 45$\degree$ angle up into the detector stack as shown in figure ~\ref{fig:hrs_side}. The HRSs were designed to provide, ``a large acceptance in both angle and momentum, good position and angular resolution in the scattering plane, an extended target acceptance, and a large angular range.~\cite{Article:HallA}"

\begin{figure}[!ht]
\begin{center}
\includegraphics[width=0.7\linewidth]{HRS_Diagram.png}
\end{center}
\caption{
{\bf{Side View of Single HRS.}} Image from ~\cite{Thesis:Wang}.}
\label{fig:hrs_side}
\end{figure}

The superconducting quadrupoles are labelled Q1, Q2, and Q3 as shown in ~\ref{fig:hrs_side}. The Q1 magnet provides focussing of the particles in the vertical plane, and the identical Q2 and Q3 provide focussing in the transverse plane. The 6.6 m long superconducting dipole magnet bends the particles through a 45$\degree$ angle  while providing additional focussing while also allowing for the use of extended targets ~\cite{Article:HallA}. The primary characteristics of the HRSs are given in figure ~\ref{fig:hrs_specs}. Unfortunately, during the run period the Q3 power supply of the RHRS was malfunctioning and could not reach the central momentum setting of 3.055 GeV required for the experiment. As such this analysis considers only data from the LHRS ~\cite{Thesis:Ye}. 

\begin{figure}[!ht]
\begin{center}
\includegraphics[width=0.6\linewidth]{HRS_Specs.png}
\end{center}
\caption{
{\bf{Table of Important HRS Values.}} Image from ~\cite{Article:HallA}.}
\label{fig:hrs_specs}
\end{figure}

\section{Detector Package}
\label{sec:detectors}

The standard detector package for Hall A can be seen in figures ~\ref{fig:halla_top} and ~\ref{fig:hrs_side}. These compnents include VDCs, scintillator planes, gas Cherenkovs, and electromagnetic calorimeters as well as the corresponding data acquisition systems. Each of these components will be described in the following sections. 

\subsection{Vertical Drift Chambers}
\label{ssec:vdcs}

After passing through the HRSs the scattered particles first pass through two vertical drift chambers (VDCs). These chambers each contain two planes of 368 wires each that are designated U$_1$ and V$_1$ for the bottom VDC and U$_2$ and V$_2$ for the top VDC. These pairs of planes are oriented at a 45$\degree$ angle to one another and separated by 0.335 m as shown in figure ~\ref{fig:vdcs_exterior}. The lower VDC lies in spectrometer focal plane and the upper VDC allows for angular reconstruction of particle trajectories. The VDCs' interior, figure ~\ref{fig:vdcs_interior}, are filled with a 62-38 mixture of argon and ethane gasses and flowed through at a rate of 10 liter/hour. Two gold-plated mylar planes sit above and below the sense wires and a large electric field is created between them on the order of -4 kV. As charged particles pass through the electric field attracts the charged particles to the sense wires which then pass the signal from those particles through amplifier cards and on to the data acquisition system ~\cite{Article:VDCs} ~\cite{Thesis:Ye}.

\begin{figure}[!ht]
\begin{center}
\includegraphics[width=0.6\linewidth]{VDCs_External.png}
\end{center}
\caption{
{\bf{External VDC Diagram.}} Image from ~\cite{Article:VDCs}.}
\label{fig:vdcs_exterior}
\end{figure}

\begin{figure}[!ht]
\begin{center}
\includegraphics[width=0.6\linewidth]{VDCs_Internal.png}
\end{center}
\caption{
{\bf{Internal VDC Diagram.}} Image from ~\cite{Article:VDCs}.}
\label{fig:vdcs_interior}
\end{figure}

When operating at a high enough voltage the sense wires are efficient to above 99$\%$ in the central region of the wire planes. Towards the edge of the wire planes the efficiency falls off. In general a charged particle will be detected by four to six wires. A wire is considered to be efficient if that wire fired at the same time as its two adjacent wires also fired. Thus the efficiency of a wire is given by equation ~\ref{eq:wire_efficiency}, where $\kappa$ is the number of times a wire was considered to be efficient for an event and $\lambda$ is the number of time that wire was inefficient for an event ~\cite{Article:VDCs}. Figure ~\ref{fig:vdc_efficiency} shows the wire efficiency spectrum for a good run on the left and a bad run on the right. The right spectrum indicates that the operating voltage may be too low or unsteady or that some of the amplifier cards have become disconnected.

\begin{equation} \label{eq:wire_efficiency}
	\epsilon = \frac{\kappa}{\kappa + \lambda}
\end{equation}

\begin{figure}[!ht]
\begin{center}
\includegraphics[width=1.0\linewidth]{Wire_Efficiencies.png}
\end{center}
\caption{
{\bf{Example Wire Efficiency Spectra.}} The left spectrum represents a good run and the right represents a run that has some type of problem.}
\label{fig:vdc_efficiency}
\end{figure}