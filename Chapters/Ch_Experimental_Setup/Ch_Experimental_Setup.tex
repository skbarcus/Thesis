% Introduction
\chapter{Experimental Setup} % Main chapter title
\label{ch:experiment} % For referencing the chapter elsewhere, use 

\section{Overview}
\label{sec:overview}

The Thomas Jefferson National Accelerator Facility (Jefferson Lab or JLab) located in Newport News Virginia uses a continuous electron beam accelerator facility (CEBAF) to perform electron scattering experiments as a means of studying nuclear structure. Jefferson lab consists of four experimental halls designated Hall A, B, C, and the newly commissioned Hall D as in figure ~\ref{fig:jlab}. The facility is capable of creating electron beams of energies as high as 12 GeV and supplying those electrons to the four halls simultaneously. The 12 GeV capability is a recently completed upgrade and during the time the experiment discussed in this thesis, experiment E08-014, ran the facility was limited to 6 GeV beam energies.  

\begin{figure}[!ht]
\begin{center}
\includegraphics[width=0.7\linewidth]{JLab_Layout.png}
\end{center}
\caption{
{\bf{Thomas Jefferson National Accelerator Facility.}} CEBAF is the ring connecting to the experimental halls. Hall D is now located at the top right of the image. Image from ~\cite{Article:HallA}.}
\label{fig:jlab}
\end{figure}

\section{Experiment E08-014}
\label{sec:x>2}
Experiment E08-014 ran in Jefferson Lab's Hall A in 2011. The experiment used electron scattering to measure the inclusive cross sections of various targets using both of Hall A's high resolution spectrometers (HRSs). E08-014 aimed to compare heavy targets to two and three-nucleon targets to study the short range correlations (SRC) between these two and three-nucleon clusters. To this end inclusive cross sections for $^2$H, $^3$He, $^4$He, $^12$C, $^{40}$Ca, and $^{48}$Ca were measured in the region of 1.1 GeV/c $<$ Q$^2$ $<$ 2.5 GeV/c. This experiment covered the $x_{Bj}$ range encompassing the quasielastic (QE) region up to $x_{Bj}$ greater than 3 ~\cite{Thesis:Ye} ~\cite{src_website}. 

While experiment E08-014 focused on the QE region of electron scattering, one kinematic region, KIN 3.2 in figure ~\ref{fig:kin3.2}, also included elastically scattered electrons. The elastic events can be seen by plotting the scattering angle of the electron versus the scattered electron's energy, $E'$, as shown in figure ~\ref{fig:elastic_band}. The resulting curve (red) gives the elastic scattering band for $^3$He. When this band is compared with the spectrometer's upper and lower acceptance in energy and angle, represented by the black lines, it becomes apparent that the elastic band passes through KIN 3.2, and thus we expect to find $^3$He elastic data in KIN 3.2.  

\begin{figure}[!ht]
\begin{center}
\includegraphics[width=0.7\linewidth]{SRC_Kinematics.png}
\end{center}
\caption{
{\bf{Kinematic Coverage of Experiment E08-014.}} The elastic $^3$He data is located in KIN 3.2. Image from ~\cite{Thesis:Ye}.}
\label{fig:kin3.2}
\end{figure}

\begin{figure}[!ht]
\begin{center}
\includegraphics[width=0.7\linewidth]{Elastic_Band.png}
\end{center}
\caption{
{\bf{Elastic Band for $^3$He.}} It is clear that the red elastic band passes through the box made by the intersecting black lines representing the maximum and minimum spectrometer acceptances in energy and angle for KIN 3.2.}
\label{fig:elastic_band}
\end{figure}

These elastic events were scattered from a gaseous $^3$He target allowing for the extraction of an elastic $^3$He cross section. This new measurement is located in the little studied region of Q$^2 = $ 35 fm$^{-2}$ as seen in figure \ref{fig:jlab_3he} which shows the $^3$He charge form factor, $F_{ch}$. This understudied region is interesting because it has the potential to constrain and improve previous fits of the $^3$He form factors. In particular, high $Q^2$ data like this helps to pin down the magnetic form factor. Equation ~\ref{eq:rosenbluth_long} makes clear that to measure the magnetic form factor's contribution to the cross section large $Q^2$ values and large angles are required. Unfortunately, there are few measurements in the world data of high enough $Q^2$ to understand the magnetic form factor's behavior after its first minima. 

\begin{figure}[!ht]
\begin{center}
\includegraphics[width=0.7\linewidth]{JLab_3He_High_Q2_Charge_FF_Clean_Arrow.png}
\end{center}
\caption{
{\bf{Location of New Elastic Measurement.}} This plot shows the $Q^2$ region of the $^3$He electric form factor, $F_{ch}$, where the new elastic $^3$He measurement is located. Image from ~\cite{Article:Alex}.}
\label{fig:jlab_3he}
\end{figure}

\section{CEBAF}
\label{sec:CEBAF}

Jefferson Lab's Continuous Electron Beam Accelerator Facility (CEBAF) uses superconducting radio frequency (srf) cavities to accelerate electrons to energies up to 12 GeV after a recent upgrade. However, this upgrade was completed after this experiment and as such this section will discuss the 6 GeV era beam before the upgrade and Hall D was built. The accelerated electrons form polarizable continuous wave (cw) beams that can be delivered to up to four scientific halls simultaneously for use in nuclear physics experiments. These beams have a maximum energy of 5.7 GeV and a maximum current of 200 $\mu$A. The beam can be split this current among the three experimental halls in any combination totalling less than the maximum current while providing the maximum energy to each hall. ~\cite{Article:CEBAF}

CEBAF begins creating an electron beam using either a thermionic or polarized gun to inject electrons into the accelerator. The polarized gun produces electrons by illuminating a GaAs cathode crystal with a 1497 MHz diode laser tuned to 780 nm. These electrons then enter the first (North) of two linacs each of which contain 20 cryomodules that accelerate the electrons with a maximum gradient exceeding 7 MeV/m. At the end of the North linac the electrons are bent around a 180$\degree$ bend and enter the South linac passing through 20 more cryomodules. Upon reaching the end of the South linac the beam can be directed into any of the three halls by means of RF separators and septa. If higher energies are desired the beam can be recirculated through the linacs up to four additional times for a maximum of five passes through the accelerator resulting in the maximum energy of ~5.7 GeV  ~\cite{Article:HallA}.

\section{Hall A}
\label{sec:HallA}

The distinguishing feature of Jefferson Lab's experimental Hall A are the two High Resolution Spectrometers (HRSs) and their associated detector packages. At a central momentum setting of 4 GeV these two spectrometers provide a momentum resolution better than $\frac{\delta p}{p} = 2*10^{-4}$ as well as a horizontal angular resolution of more than 2 mrad. The spectrometer magnets bend the particles upward into the detector stack using a series of quadrupole and dipole magnets in a QQDQ arrangement ~\cite{Article:HallA}. A side view of Hall A is given in figure ~\ref{fig:halla_side} and a top view is given in ~\ref{fig:halla_top}. Each of the components listed in ~\ref{fig:halla_top} will be discussed individually in the following sections.

\begin{figure}[!ht]
\begin{center}
\includegraphics[width=0.7\linewidth]{Hall_A_Side_View.png}
\end{center}
\caption{
{\bf{Hall A Side View.}} Image from ~\cite{Article:HallA}.}
\label{fig:halla_side}
\end{figure}

\begin{figure}[!ht]
\begin{center}
\includegraphics[width=0.7\linewidth]{Hall_A_Top_View.png}
\end{center}
\caption{
{\bf{Hall A Top View.}} Image from ~\cite{Thesis:Wang}.}
\label{fig:halla_top}
\end{figure}