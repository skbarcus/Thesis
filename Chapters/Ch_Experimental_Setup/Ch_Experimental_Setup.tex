% Introduction
\chapter{Experimental Setup} % Main chapter title
\label{ch:experiment} % For referencing the chapter elsewhere, use 

\section{Overview}
\label{sec:overview}

The Thomas Jefferson National Accelerator Facility (Jefferson Lab or JLab) located in Newport News, Virginia uses the Continuous Electron Beam Accelerator Facility (CEBAF) to perform electron scattering experiments to study nuclear structure. Jefferson lab consists of four experimental halls designated Halls A, B, C, and the newly commissioned Hall D as shown in Figure \ref{fig:jlab}. The facility is capable of creating electron beams of energies as high as 12 GeV and supplying those electrons to the four halls simultaneously (Hall A is limited to 11 GeV). The 12 GeV capability is a recently completed upgrade. At the time of the experiment discussed in this thesis, experiment E08-014, the facility was limited to 6 GeV beam energy.  

\begin{figure}[!ht]
\begin{center}
\includegraphics[width=0.7\linewidth]{JLab_Layout.png}
\end{center}
\caption[Thomas Jefferson National Accelerator Facility]{
{\bf{Thomas Jefferson National Accelerator Facility.}} CEBAF is the ring connecting to the experimental halls. This image is of the previous 6 GeV configuration. Now Hall D is now located at the top right of the image. Image from \cite{Article:HallA}.}
\label{fig:jlab}
\end{figure}

\section{Experiment E08-014}
\label{sec:x>2}
Experiment E08-014 ran in Jefferson Lab's Hall A in 2011. The experiment used electron scattering to measure the inclusive cross sections, N(e,e'), of various targets using both of Hall A's high resolution spectrometers (HRSs). E08-014 compared heavy targets to two and three-nucleon targets to study the short range correlations (SRC). To this end, inclusive cross sections for $^2$H, $^3$He, $^4$He, $^{12}$C, $^{40}$Ca, and $^{48}$Ca were measured in the region of 1.1 GeV$^2$ $<$ $Q^2$ $<$ 2.5 GeV$^2$. This experiment studied the region of 1.3 $<$ $x_{Bj}$ $<$ 3 \cite{Thesis:Ye} \cite{src_website}. 

While experiment E08-014 focused on the QE region of electron scattering, one kinematic region, Kin 3.2 in Figure \ref{fig:kin3.2}, also included elastically scattered electrons. The elastic events can be seen by plotting the scattering angle of the electron versus the scattered electron's energy, $E'$, as shown in Figure \ref{fig:elastic_band}. The resulting curve (red) gives the elastic scattering band for $^3$He. When this band is compared with the spectrometer's upper and lower acceptance in energy and angle, represented by the black lines, it becomes apparent that the elastic band passes through Kin 3.2. Thus we expect to find $^3$He elastic data in Kin 3.2. 

\begin{figure}[!ht]
\begin{center}
\includegraphics[width=0.7\linewidth]{SRC_Kinematics.png}
\end{center}
\caption[Kinematic Coverage of Experiment E08-014]{
{\bf{Kinematic Coverage of Experiment E08-014.}} Kinematic coverage of experiment E08-014. The elastic $^3$He data used in this analysis is located in Kin 3.2. Image from \cite{Thesis:Ye}.}
\label{fig:kin3.2}
\end{figure}

\begin{figure}[!ht]
\begin{center}
\includegraphics[width=0.7\linewidth]{Elastic_Band.png}
\end{center}
\caption[Elastic Band for $^3$He]{
{\bf{Elastic Band for $^3$He.}} The box made by the intersecting black lines represents the maximum and minimum spectrometer acceptances in energy and angle for Kin 3.2. The red line is the elastic scattering band. Clearly the red elastic band passes through the spectrometer's acceptance so we expect to find elastic events in Kin 3.2.}
\label{fig:elastic_band}
\end{figure}

Now that we believe there should be elastically scattered electrons in our data let us see if we can detect them. To find these electrons we will search for the elastic peak in $x_{Bj}$ that we discussed in Section \ref{sec:kinematics}. Figure \ref{fig:elastic_xbj} shows a plot of $x_{Bj}$ for the summed production runs of Kin 3.2. The large initial distribution is the quasielastic peak of $^3$He. Looking at $x_{Bj}=3$ it is clear that there is another smaller peak indicating the presence of elastically scattered electrons. The smaller peak contains around a maximum of 1000 electrons, which while enough for a cross section extraction, places a limit on the uncertainty of our measurement. This electron yield will also decrease as we begin to impose cuts on the data.

\begin{figure}[!ht]
\begin{center}
\includegraphics[width=1.\linewidth]{Elastic_Xbj.png}
\end{center}
\caption[Elastic Peak in $x_{Bj}$]{
{\bf{Elastic Peak in $x_{Bj}$.}} The elastic peak is located at approximately $x_{Bj}=3$ as expected. Notice that there is a large quasielastic background in this electron sample in addition to the elastic electrons in the elastic peak.}
\label{fig:elastic_xbj}
\end{figure}

These elastic events were scattered from a gaseous $^3$He target and used to extract an elastic $^3$He cross section. This new measurement is located in the little studied region of $Q^2$ = 35 fm$^{-2}$ as seen in Figure \ref{fig:jlab_3he} which shows the $^3$He charge form factor, $F_{ch}$. This understudied region is interesting because it has the potential to constrain and improve previous fits of the $^3$He form factors. In particular, high $Q^2$ data points like this help to pin down the magnetic form factor. Equation \ref{eq:rosenbluth_long} makes clear that to measure the magnetic form factor's contribution to the cross section large $Q^2$ values and large back angles are required. Unfortunately, there are few measurements in the world data of high enough $Q^2$ to understand the magnetic form factor's behavior after its first minima. 

\begin{figure}[!ht]
\begin{center}
\includegraphics[width=0.7\linewidth]{JLab_3He_High_Q2_Charge_FF_Clean_Arrow.png}
\end{center}
\caption[Location of New Elastic Measurement]{
{\bf{Location of New Elastic Measurement.}} This plot shows the $Q^2$ region of the $^3$He electric form factor, $F_{ch}$, where the new elastic $^3$He measurement is located. The $Y$-axis is $\lvert {F_{ch}(Q^2)} \rvert$. Image from \cite{Article:Alex}.}
\label{fig:jlab_3he}
\end{figure}

\section{CEBAF}
\label{sec:CEBAF}

Jefferson Lab's Continuous Electron Beam Accelerator Facility (CEBAF) uses superconducting radio frequency (SRF) cavities to accelerate electrons to energies up to 12 GeV after a recent upgrade. However, this upgrade was completed after this experiment and as such this section will discuss the 6 GeV era beam before the upgrade and Hall D was built. The accelerated electrons form polarizable continuous wave (CW) beams that can be delivered to up to four scientific halls simultaneously for use in nuclear physics experiments. These beams have a maximum energy of 5.7 GeV and a maximum current of 200 $\mu$A. The accelerator can split this current among the three experimental halls \cite{Article:CEBAF}.

CEBAF begins creating an electron beam using either a thermionic or polarized gun to inject electrons into the accelerator. The polarized gun produces electrons by illuminating a GaAs cathode crystal with a diode laser pulsed at 1497 MHz. (While CEBAF operates with a `continuous wave' beam, if one looks at small enough time lengths the beam is still pulsed. This is due to needing to inject the electrons into the RF cavities at the proper time to have the standing waves in the RF cavities accelerate them.) These electrons then enter the first (North) of two linacs each of which contain 20 cryomodules that accelerate the electrons with a maximum gradient exceeding 7 MeV/m. At the end of the North linac the electrons are bent around a 180$\degree$ bend and enter the South linac passing through 20 more cryomodules. Upon reaching the end of the South linac the beam can be directed into any of the three halls by means of RF separators and septa. If higher energies are desired the beam can be recirculated through the linacs up to four additional times for a maximum of five passes through the accelerator resulting in the maximum energy of 5.7 GeV  \cite{Article:HallA}.%which operates at 780 nm.

\section{Hall A Beamline}
\label{sec:HallA_beamline}

The distinguishing feature of Jefferson Lab's experimental Hall A are the two High Resolution Spectrometers (HRSs) and their associated detector packages. At a central momentum setting of 4 GeV these two spectrometers provide a momentum resolution better than $\frac{\delta p}{p} = 2\times10^{-4}$ as well as a horizontal angular resolution of more than 2 mrad. The spectrometer magnets bend the particles upward into the detector stack using a series of quadrupole and dipole magnets in a QQDQ arrangement \cite{Article:HallA}. A side view of Hall A is given in Figure \ref{fig:halla_side} and a top view is given in \ref{fig:halla_top}. Each of the components listed in \ref{fig:halla_top} will be discussed individually in the following sections.

\begin{figure}[!ht]
\begin{center}
\includegraphics[width=\linewidth]{Hall_A_Side_View.png}\end{center}
\caption[Hall A Side View]{
{\bf{Hall A Side View.}} The electron beam enters the hall from the left side of the image. It then interacts with the target at the hall's center. The scattered particles then pass through the High Resolution Spectrometers which bend the particles upward 45$^\circ$ where they enter the detector stacks. Image from \cite{Article:HallA}.}
\label{fig:halla_side}
\end{figure}

\begin{figure}[!ht]
\begin{center}
\includegraphics[width=1.\linewidth]{Hall_A_Top_View.png}
\end{center}
\caption[Hall A Top View]{
{\bf{Hall A Top View.}} The electron beam enters the hall from the left side of the image. It then passes through the beam current monitors which measure the current. The beam then goes through a raster to avoid overheating one area of the target. Next the beam passes through the beam position monitors which measure its position. It then interacts with the target at the hall's center. The scattered particles then pass through the High Resolution Spectrometers which bend the particles upward 45$^\circ$ where they enter the detector stacks. In the stacks the particles pass through the vertical drift chambers, used for trajectory reconstruction, followed by the straw chambers. They then pass through the first set of scintillator paddles, S$_0$, followed by the gas Cherenkov, before passing through the second set of scintillator paddles, S$_{2m}$. A coincidence of S$_0$, S$_{2m}$, and the GC creates the main production trigger. The GC also performs particle identification. Finally the particles enter the electromagnetic shower calorimeters which help further aid particle identification. Image from \cite{Thesis:Wang}.}
\label{fig:halla_top}
\end{figure}

\subsection{Beam Energy}
\label{ssec:beam_energy}

An accurate measure of the electron beam's energy is necessary to obtain accurate experimental results. The energy of the electron beam was measured using the Arc method laid out in \cite{Article:HallA}. This method works by passing the electron beam through a series of dipole magnets in the arc section of the beam line and measuring its deflection as shown in Figure \ref{fig:arc}. The beam's momentum, $|\overrightarrow{p}|$, is then given by the field integral of the eight dipole magnets, $\int \overrightarrow{\rm \textbf{B}} \cdot \overrightarrow{\rm \textbf{d}l}$, divided by the arc bend angle, $\phi_{arc}$, multiplied by a constant $C_{arc}=0.299792$ GeV rad T$^{-1}$ m$^{-1}$/c as given in Equation \ref{eq:arc}. To perform this calculation two measurements are required. The first measurement is of the magnetic field integral of the eight dipoles and is made based on a ninth reference dipole. The second measurement is of the bend angle of the arc which is measured by a set of wire scanners (Super Harps).   

\begin{figure}[!ht]
\begin{center}
\includegraphics[width=0.7\linewidth]{Arc_Method.png}
\end{center}
\caption[Arc Energy Measurement Diagram]{
{\bf{Arc Energy Measurement Diagram.}} The electron beam is bent through an angle $\phi_{arc}$ by a series of eight dipole magnets. Image from \cite{Thesis:Wang}.}
\label{fig:arc}
\end{figure}

\begin{equation} \label{eq:arc}
	|\overrightarrow{p}| = C_{arc} \frac{\int \overrightarrow{\rm \textbf{B}} \cdot \overrightarrow{\rm \textbf{d}l}}{\phi_{arc}}
	\text{\equationlabels{Beam Momentum (Arc Method)}}
\end{equation}

\subsection{Beam Position Monitors}
\label{ssec:bpms}

Along with the beam's energy its position must also be well known. To assess the beam's position two beam position monitors (BPMs) are located 7.524 m and 1.286 m upstream of the Hall A target \cite{Article:HallA}. Figure \ref{fig:beamline} shows the layout of the beamline components. Each of these BPMs is made up of four orthogonally oriented antennae placed perpendicular to the beam. For beam currents above 1 $\mu$A these antennae produce a signal that is inversely proportional to the beam's distance from the antennae. The position of the beam is determined by the difference-over-sum technique laid out in \cite{bpm1} and \cite{bpm2}.

\begin{figure}[!ht]
\begin{center}
\includegraphics[width=1.\linewidth]{Beamline_Layout.png}
\end{center}
\caption[Layout of Hall A Beamline Components]{
{\bf{Layout of Hall A Beamline Components.}} Prior to striking the target the electron beam passes through the two beam current monitors, BCM1 and BCM2, and the Unser which measure the beam current. The beam then passes through the raster which uses steering magnets to spread the beam out over the surface of the target. Finally the beam passes through two beam position monitors, BPMA and BPMB, which measure the beam's position. Image from \cite{Thesis:Wang}.}
\label{fig:beamline}
\end{figure}

The BPMs are calibrated by wire scanners adjacent to each of them. These wire scanners are independently calibrated periodically with respect to the Hall A coordinates and are accurate to within 200 $\mu$m. The data from the BPMs is recorded in the EPICS database every second and in the data stream every three to four seconds. Each of the eight antennae signals is also recorded in the CODA data stream on an event by event basis \cite{Thesis:Ye}. 

\subsection{Raster}
\label{ssec:raster}

The electron beam is generally quite narrow at $<$ 0.3 mm \cite{Thesis:Wang}. If this small beam falls on a single part of the target the target can overheat locally causing parts of the target to behave differently than those not struck by the beam. There is also the risk of burning a hole through the target. To avoid these problems the electron beam is rastered, meaning that the beam is spread out over a larger area of the target. The fast raster consists of two steering magnets and operates with a frequency of 17-24 kHz. It is located 23 m upstream from the target \cite{Article:HallA} as seen in Figure \ref{fig:beamline}. This system has the capability to spread the beam over several mm in all directions illuminating the target uniformly. 

\subsection{Beam Current Monitors}
\label{ssec:bcms}

The beam current is measured by two beam current monitors (BCMs), which are RF cavity monitors, and an Unser located 25 m upstream of the target \cite{Article:HallA} as shown in Figure \ref{fig:beamline}. The Unser measures beam current by, ``a system combining a second harmonic magnetic modulator with an active current transformer in an operational feedback loop, to obtain wide band response down to dc" \cite{Article:Unser}. The two BCM RF cavities are located on either side of the Unser and are tuned to the electron beam frequency of 1497 MHz (see Section \ref{sec:CEBAF}). 

These RF cavity monitors then produce a signal that is proportional to the beam current which is recorded by a data acquisition system. Before being recorded the output signals of the two BCMs are each split into three. One of those three signals is amplified by a factor of three and another by a factor of ten. This results in six signals from the two BCMs designated U$_1$, U$_3$, U$_{10}$, D$_1$, D$_3$, and D$_{10}$. The procedure and results for the BCM calibration for experiment E08-014 can be found in \cite{bcm_calibration}. 

\section{Target}
\label{sec:target}

Hall A has a cryogenic distribution system (CDS) capable of cooling targets to temperatures of around 5 K to 15 K \cite{Article:HallA}. The targets are kept in a vacuum scattering chamber where they are allowed to interact with the electron beam. Inside this chamber is a ladder containing various experimental targets as shown in Figure \ref{fig:ladder}. The target ladder is divided into three loops which are cryogenically cooled while having their pressure and temperature monitored continuously. During experiment E08-014 loops one and two contained one each of a 10 cm and 20 cm target cell while loop 3 contained two 20 cm target cells. The target ladder can be moved up and down to place different targets in the path of the electron beam from the counting house \cite{Thesis:Ye}.  

\begin{figure}[!ht]
\begin{center}
\includegraphics[width=0.5\linewidth]{Ladder.png}
\end{center}
\caption[Target Ladder for Experiment E08-014]{
{\bf{Target Ladder for Experiment E08-014.}} Coolant is flowed through the target ladder to maintain the targets' temperatures. The cylinders on the right of the image contain the target gasses. Only loops one and two are shown in this image. Image from \cite{Thesis:Ye}.}
\label{fig:ladder}
\end{figure}

Experiment E08-014 used a liquid deuterium target as well as a gaseous $^3$He target, examined in this thesis, and a gaseous $^4$He target. The $^3$He target was used in the second run period of the experiment taking place from April 21\textsuperscript{st} to May 15\textsuperscript{th} of 2011. The $^3$He target was located in loop 1 and was cooled to 17 K with a pressure of 211 psia \cite{Thesis:Ye}. The entire target layout over the two runs can be seen in Figure \ref{fig:targets} and the monitoring and control system is shown in Figure \ref{fig:cryo_controls}.

\begin{figure}[!ht]
\begin{center}
\includegraphics[width=0.8\linewidth]{Targets_Table.png}
\end{center}
\caption[Table of Target Information for Experiment E08-014]{
{\bf{Table of Target Information for Experiment E08-014.}} Image from \cite{Thesis:Ye}.}
\label{fig:targets}
\end{figure}

\begin{figure}[!ht]
\begin{center}
\includegraphics[width=1.1\linewidth]{Cryo_Controls.png}
\end{center}
\caption[Cryogentic Monitoring and Control Screen]{
{\bf{Cryogentic Monitoring and Control Screen.}} This control panel monitors and controls the temperatures of the individual targets. This panel is also used to move the target ladder up and down so that different targets can be placed in the electron beam. Image from \cite{Thesis:Ye}.}
\label{fig:cryo_controls}
\end{figure}

A 30 cm long carbon foil optics target was installed below loop 3. This target contains seven carbon foils spaced 5 cm apart ranging from -15 cm to +15 cm. Electrons scattering off of these foils can be used to calibrate the position of the target from data collected by the detector package as will be discussed in \ref{sec:optics}. Along with these targets a 10 cm and a 20 cm dummy target were also installed below the optics target. These dummy targets each contained two thick aluminium foils separated by 10 cm and 20 cm respectively. The dummy targets were used to study the end cap contributions of the target cells. There were also a BeO, a $^{12}$C, and an empty target installed below \cite{Thesis:Ye}.  

\section{High Resolution Spectrometers}
\label{sec:HRSs}

The high resolution spectrometers (HRSs) are the workhorses of Hall A. They are designated the left HRS (LHRS) and right HRS (RHRS) for the spectrometers on the left and right side of the beam direction respectively. While they were designed to be identical they each have unique features based on their construction as well as the general wear and tear of age on their components. They have a momentum resolution of 1$\times$10$^{-4}$ from 0.8 GeV to 4 GeV. They use a combination of superconducting quadrupole and dipole magnets in a QQDQ combination to bend the scattered particles through a 45$\degree$ angle up into the detector stack as shown in Figure \ref{fig:hrs_side}. The HRSs were designed to provide, ``a large acceptance in both angle and momentum, good position and angular resolution in the scattering plane, an extended target acceptance, and a large angular range.\cite{Article:HallA}"

\begin{figure}[!ht]
\begin{center}
\includegraphics[width=1.\linewidth]{HRS_Diagram.png}
\end{center}
\caption[Side View of Single HRS]{
{\bf{Side View of Single HRS.}} After interacting with the target the scattered particles pass through the High Resolution Spectrometers. These spectrometers have three quadrapole focusing magnets and a dipole magnet which bends the particles upward 45$^\circ$ to the detector stacks. In the stacks the particles pass through the vertical drift chambers, used for trajectory reconstruction, followed by the straw chambers. They then pass through the first set of scintillator paddles, S$_0$, followed by the gas Cherenkov, before passing through the second set of scintillator paddles, S$_{2m}$. A coincidence of S$_0$, S$_{2m}$, and the GC creates the main production trigger. The GC also performs particle identification. Finally the particles enter the electromagnetic shower calorimeters which help further aid particle identification. Image from \cite{Thesis:Wang}.}
\label{fig:hrs_side}
\end{figure}

The superconducting quadrupoles are labelled Q1, Q2, and Q3 as shown in \ref{fig:hrs_side}. The Q1 magnet provides focussing of the particles in the vertical plane, and the identical Q2 and Q3 provide focussing in the transverse plane. The 6.6 m long superconducting dipole magnet bends the particles through a 45$\degree$ angle  while providing additional focussing and the ability to use extended targets \cite{Article:HallA}. The primary characteristics of the HRSs are given in Figure \ref{fig:hrs_specs}. Unfortunately, during the run period the Q3 power supply of the RHRS was malfunctioning and could not reach the central momentum setting of 3.055 GeV required for the experiment. As such this analysis considers only data from the LHRS \cite{Thesis:Ye}. 

\begin{figure}[!ht]
\begin{center}
\includegraphics[width=0.6\linewidth]{HRS_Specs.png}
\end{center}
\caption[Table of Important HRS Values]{
{\bf{Table of Important HRS Values.}} Image from \cite{Article:HallA}.}
\label{fig:hrs_specs}
\end{figure}

\section{Detector Package}
\label{sec:detectors}

The standard detector package for Hall A can be seen in Figures \ref{fig:halla_top} and \ref{fig:hrs_side}. These components include VDCs, scintillator planes, gas Cherenkovs, and electromagnetic calorimeters as well as the corresponding data acquisition systems. Each of these components will be described in the following sections. 

\subsection{Vertical Drift Chambers}
\label{ssec:vdcs}

After passing through the HRSs the scattered charged particles first pass through two vertical drift chambers (VDCs). These chambers each contain two planes of 368 wires each that are designated U$_1$ and V$_1$ for the bottom VDC and U$_2$ and V$_2$ for the top VDC. These pairs of planes are oriented at a 45$\degree$ angle to one another and are separated by 0.335 m as shown in Figure \ref{fig:vdcs_exterior}. The lower VDC lies in the spectrometer focal plane and the upper VDC allows angular reconstruction of particle trajectories \cite{Article:VDCs}. 

The VDCs' interiors, Figure \ref{fig:vdcs_interior}, were filled with a 62$\%$-38$\%$ mixture of argon and ethane gasses flowed through at a rate of 10 liter/hour. Two gold-plated mylar planes sit above and below the sense wires and a large electric field is created between them on the order of -4 kV. As charged particles pass through the gas the gas molecules become ionized and release electrons. The electric field then attracts these electrons to the sense wires which then pass the signal from those electrons through amplifier cards and on to the data acquisition system. The position of the particle can be resolved to around 100 $\mu$m and the angle can be resolved to about 100 mrad \cite{Thesis:Ye} \cite{Article:VDCs}.

\begin{figure}[!ht]
\begin{center}
\includegraphics[width=0.6\linewidth]{VDCs_External.png}
\end{center}
\caption[External VDC Diagram]{
{\bf{External VDC Diagram.}} Each of the two VDCs, called the upper VDC and lower VDC, contains two planes of 368 sense wires oriented at 45$^{\circ}$ angles to each other. The interior of the VDCs is filled with a 62$\%$/32$\%$ argon/ethane gas mixture. A large electric field is created perpendicular to the wire planes which attracts electrons to the sense wires. Image from \cite{Article:VDCs}.}
\label{fig:vdcs_exterior}
\end{figure}

\begin{figure}[!ht]
\begin{center}
\includegraphics[width=0.7\linewidth]{VDCs_Internal.png}
\end{center}
\caption[Internal VDC Side View Diagram]{
{\bf{Internal VDC Side View Diagram.}} Two gold-plated mylar planes (three bold central lines) are situated above and below each sense wire plane (dashed lines). These mylar planes create a large a electric field perpendicular to the wire planes which attracts electrons to the sense wires. A 62$\%$/32$\%$ argon/ethane gas mixture is flowed through the VDCs from the gas boxes seen on either side of the figure. Image from \cite{Article:VDCs}.}
\label{fig:vdcs_interior}
\end{figure}

When operating at a high enough voltage the sense wires are efficient to above 99$\%$ in the central region of the wire planes. Towards the edge of the wire planes the efficiency falls off. In general a charged particle will be detected by four to six wires. A wire is considered to be efficient if that wire fired at the same time as its two adjacent wires. Thus the efficiency of a wire is given by Equation \ref{eq:wire_efficiency}, where $\kappa$ is the number of times a wire was considered to be efficient for an event and $\lambda$ is the number of times that the wire was inefficient for an event \cite{Article:VDCs}. Figure \ref{fig:vdc_efficiency} shows the wire efficiency spectrum for a good run on the left and a bad run on the right. The right spectrum indicates that the operating voltage may be too low or unsteady, and/or that some of the amplifier cards have become disconnected.

\begin{equation} \label{eq:wire_efficiency}
	\epsilon = \frac{\kappa}{\kappa + \lambda}
	\text{\equationlabels{VDC Wire Efficiency}}
\end{equation}

\begin{figure}[!ht]
\begin{center}
\includegraphics[width=1.0\linewidth]{Wire_Efficiencies.png}
\end{center}
\caption[Example Wire Efficiency Spectra]{
{\bf{Example Wire Efficiency Spectra.}} The $X$-axis represents the wire number and the $Y$-axis represents the wire's efficiency. The left image indicates a good run, and the right image shows a bad run. In the right image the operating voltage may be too low or unsteady, and/or some of the amplifier cards have become disconnected.}
\label{fig:vdc_efficiency}
\end{figure}

Where and at what angle a charged particle passes through the VDCs determines the time it takes for the electrons produced by its passing to drift to the sense wires. Figure \ref{fig:drift} shows a typical drift time spectrum for a single wire on the left. On the right it shows a single sense wire drift cell with the sense wire at the center of the circle. Timing is measured by a common stop TDC which triggers on the sense wire signals and is stopped by the main event trigger. Higher TDC channels indicate shorter drift times.

By examining each of the four trajectory regions A, B, C, and D on the right side of Figure \ref{fig:drift} we can explain the structure of the drift time spectrum. Region A on the right side of the figure corresponds to TDC channels 1020-1080 on the left side of the figure. These drift times are created by electrons from particles with large trajectory angles that intersect less of the cell leading to a lower detection probability. Region B corresponds to channels 1080-1460. This region represents the bulk of the cell where the electric field lines are parallel giving the electrons a constant drift velocity and track density (i.e. the number of tracks that pass through a given region). Region C corresponds to channels 1460-1540. Here the field lines become quasiradial as we approach the sense wire. Electron drift velocity is roughly constant here, but the track density increases. Finally region D corresponds to channels 1540-1620. Now the trajectory is close to the sense wires so the electrons reach a maximal track density and a dramatic increase in drift velocity occurs because the electric field lines converge to the wire \cite{Article:VDCs}.

\begin{figure}[!ht]
\begin{center}
\includegraphics[width=1.\linewidth]{Wire_Timing_Spectrum.png}
\end{center}
\caption[Wire Timing Spectrum]{
{\bf{Wire Timing Spectrum.}} The left plot represents a single wire's timing spectrum. The $X$-axis is in units of TDC channel. The right image shows a single wire VDC drift cell and several trajectories a charged particle may take when passing through the cell. Image from \cite{Article:VDCs}.}
\label{fig:drift}
\end{figure}

The electron's drift time can be used to calculate the distance the particle was from the sense wire to begin pinning down its spatial location. Before this is done all of the drift spectra must be calibrated to one another. This is done by defining a reference time, $t_0$, which is defined as the channel where the maximal slope is obtained in the short drift time region around channel 1600. The drift distance can then be calculated by integrating over the number of events, $dN$, per time bin, $dt$, and scaled by a calibration constant, $k$, determined by the size of the drift cell as in Equation \ref{eq:distance} from \cite{Article:VDCs}.

\begin{equation} \label{eq:distance}
	x(t') = \frac{1}{k} \int_{t_0}^{t'} \frac{dN}{dt} dt
	\text{\equationlabels{VDC Drift Distance}}
\end{equation}

The ultimate job of the VDCs is to reconstruct the charged particles' trajectories to provide $\theta$, $\phi$, and $\textbf{k}'$. A typical particle trajectory passes through all four wire planes in the two VDCs as shown in the left side of Figure \ref{fig:trajectory}. Because there are two wire planes per VDC the two locations where the particle was determined to cross each of the VDC's wire planes can be used to create a trajectory as seen in the right side of Figure \ref{fig:trajectory}. In this case the trajectory can be described by four variables where one has defined a $U-V$ plane starting from (0,0) in the right side of the Figure \cite{Article:VDCs}. 

\begin{figure}[!ht]
\begin{center}
\includegraphics[width=1.\linewidth]{Trajectory_Reconstruction.png}
\end{center}
\caption[Trajectory Reconstruction]{
{\bf{Trajectory Reconstruction.}} The left plot shows a charged particle passing through both VDCs, and the right plot shows the coordinates needed to reconstruct that particle's trajectory. Image from \cite{Article:VDCs}.}
\label{fig:trajectory}
\end{figure}

The first two variables are the two $U-V$ coordinates of the crossing point of the particle through the first VDC wire plane called ($U$,$V$). The second two variables are the two angles, $\theta_U$ and $\theta_V$, which are the angles in the $U$ and $V$ directions respectively between the two wire plane crossing points of the first VDC. Equations \ref{eq:u}, \ref{eq:v}, and \ref{eq:theta_uv} give these four variables with $\Delta U$ and $\Delta V$ being the differences between the crossing point locations in the $U$ and $V$ directions respectively. $d$ is the distance between the $U_1$ wire plane of the bottom VDC and the $U_2$ wire plane of the top VDC, and $l_0$ is the distance between the two wire planes in a single VDC\cite{Article:VDCs} .

\begin{equation} \label{eq:u}
	U = U_1
	\text{\equationlabels{VDC U}}
\end{equation}

\begin{equation} \label{eq:v}
	V = V_1-\Delta V = V_1 - l_0 \tan\left( \theta_V \right)
	\text{\equationlabels{VDC V}}
\end{equation}

\begin{equation} \label{eq:theta_uv}
	\tan\left( \theta_{U(V)} \right) = \frac{U_2-U_1}{d}
	\text{\equationlabels{VDC $\theta_{U(V)}$}}
\end{equation}

\subsection{Scintillator Counters}
\label{ssec:scintillators}

After the VDCs are two planes of scintillator paddles located two meters apart. Figure \ref{fig:paddle} shows a schematic of a single scintillator paddle. The first plane, S$_1$, consists of six paddles of plastic scintillators which provide overlapping coverage. The second plane, S$_{2m}$, consists of 16 plastic scintillator paddles smaller than those in S$_1$ \cite{Thesis:Ye}. When a charged particle strikes one of these paddles it produces photons which travel along the length of the paddle to either end where the photons encounter a photomultiplier tube (PMT). 

\begin{figure}[!ht]
\begin{center}
\includegraphics[width=0.35\linewidth]{Paddle.png}
\end{center}
\caption[Single Plastic Scintillator Paddle]{
{\bf{Single Plastic Scintillator Paddle.}} The S$_0$ and S$_{2m}$ scintillator plane paddles are composed of an active plastic scintillator region for particle detection. The light released from the passage of particles then travels the length of the paddle to PMTs on either side. Image from \cite{Thesis:Wang}.}
\label{fig:paddle}
\end{figure}

When the photons enter the PMT they create electrons via the photoelectric effect, and an electric field accelerates these electrons down the PMT creating a larger cascade of electrons as they pass through. This produces an analog electric signal. The PMT then passes this signal on to the data acquisition system. These scintillators are very efficient and have timing resolutions of around 0.30 ns \cite{Article:HallA} \cite{Thesis:Ye}. Due to these excellent timing responses most of the primary triggers in Hall A use a coincidence of S$_1$ and S$_{2m}$ for the main production triggers of experiments. The triggers used in E08-014 are discussed in \ref{sec:triggers}. 

\subsection{Gas Cherenkov}
\label{ssec:gc}

Located in the detector stack between the S$_1$ and S$_{2m}$ scintillator planes is a gas Cherenkov (GC) detector. The GC operates by detecting Cherenkov radiation created by particles passing through a gas at velocities greater than the speed of light in that gas medium. When light passes through a transparent medium its velocity is reduced by the medium's index of refraction, $n$, as seen in Equation \ref{eq:threshold}. Because the speed of light is reduced, particles can travel faster than the speed of light in that medium. When a particle exceeds the speed of light in a medium (i.e. its velocity threshold), $v_{th}$, it creates an electromagnetic shock wave in the same manner as a jet plane creating a sonic boom. The shock wave formed has the conical shape seen in Figure \ref{fig:em_shock_wave} where the angle $\theta$ is given in Equation \ref{eq:shock_wave_angle} with $\beta = \frac{v}{c}$, where $v$ is the velocity and $c$ is the speed of light \cite{Book:Leo}.

\begin{equation} \label{eq:threshold}
	v_{th} = \frac{c}{n}
	\text{\equationlabels{Gas Cherenkov Velocity Threshold}}
\end{equation}

\begin{figure}[!ht]
\begin{center}
\includegraphics[width=0.35\linewidth]{EM_Shock_Wave.png}
\end{center}
\caption[EM Shock Wave]{
{\bf{EM Shock Wave.}} The EM shock wave is produced by a particle exceeding the speed of light in the medium it is passing through. This shock wave then produces Cherenkov radiation. Image from \cite{Thesis:Cummings}.}
\label{fig:em_shock_wave}
\end{figure}

\begin{equation} \label{eq:shock_wave_angle}
	\cos(\theta) = \frac{1}{\beta n}
	\text{\equationlabels{Cherenkov Shock Wave Angle}}
\end{equation}

GC detectors are especially useful for particle discrimination, in particular, for distinguishing between electron and pion events. GCs are able to discriminate between particles of different mass because of the velocity threshold required to create Cherenkov radiation shown in Equation \ref{eq:momentum_threshold}. The Hall A GC is generally filled with CO$_2$ which has an index of refraction of $n=1.00041$. Therefore, the momentum threshold for an electron is $p_{th}=0.0178$ GeV and for a pion it is $p_{th}=4.87$ GeV. The HRSs accept momenta up to about 4.3 GeV. Thus electrons generally create a signal passing through the GC while the pions generally do not. It is still possible for pions to interact with the gas or the detector structures and create low energy $\delta$-electrons that can be seen by the GC, but this is a low probability process. When in normal operation the GC can generally detect electrons with an efficiency of over 99$\%$ \cite{Thesis:Ye}.

\begin{equation} \label{eq:momentum_threshold}
	p_{th} = \frac{mc}{\sqrt{n^2-1}}
	\text{\equationlabels{Gas Cherenkov Momentum Threshold}}
\end{equation}

The Hall A GCs each contain ten spherical mirrors which focus the Cherenkov light onto ten PMTs as seen in Figure \ref{fig:gc}. The LHRS GC has a detector path length of 80 cm and produces an average of 7 photoelectrons per event, and the RHRS GC has a detector path length of 130 cm and yields on average 12 photoelectrons per event. \cite{Article:HallA} \cite{Article:GC}. Each PMT signal passes through a 10X amplifier before being split. One of these two signals is sent to an ADC and put in the data stream for offline analysis. The other signal is again split and one of these two new signals is converted to digital and sent to a TDC. The remaining PMT signal is summed with the PMT signals of the nine other PMTs. This summed signal is converted to a digital signal and then is used as part of the efficiency triggers. The combined GC signal was also added to the main production trigger during experiment E08-014 to prevent pions from firing the main trigger \cite{Thesis:Ye}.

\begin{figure}[!ht]
\begin{center}
\includegraphics[width=0.5\linewidth]{GC.png}
\end{center}
\caption[Hall A GC Interior]{
{\bf{Hall A GC Interior.}} Ten spherical mirrors inside the gas Cherenkov reflect the Cherenkov light onto the ten PMTs. Image from \cite{Thesis:Ye}.}
\label{fig:gc}
\end{figure}

The ten photomultiplier tubes in each GC must be calibrated such that they place the same charge detected in the same ADC channel. The charge detected is then proportional to the number of incident photons. This means each of the PMTs have the same response to a photon and thus can be compared to one another. This procedure has two parts, a hardware calibration and a software calibration. First the hardware is calibrated by gain matching the PMTs. This is done by first locating the single photoelectron (SPE) peak of each of the PMTs in the ADC data. Then by increasing or decreasing the voltage of the PMTs the SPE peak can be shifted up or down in ADC channels respectively. A target ADC channel is then chosen and the voltages of the individual PMTs are then adjusted until all of the SPE peaks are located in the same ADC channel. After the PMTs are gain matched they will each still have slightly different responses than the other tubes. The final software calibration is done by selecting the same target ADC channel for the SPE peaks. Each PMT's ADC spectrum is then given an constant offset to place its SPE peak in the desired channel. For the specific procedure and results from experiment E08-014's GC calibration see Section 4.2.1 of \cite{Thesis:Ye}.

\subsection{Electromagnetic Calorimeters}
\label{ssec:em_cal}

In the detector stack of each HRS behind the GC and the S$_{2m}$ scintillator plane sit a series of lead-glass blocks with PMTs attached making up an electromagnetic calorimeter. The calorimeters in the LHRS and RHRS have their lead-glass blocks arranged differently as shown in Figure \ref{fig:em_cal}. The LHRS blocks are set in two columns of 17 blocks each with the first layer called pion-rejector 1 (PRL1) and the second layer called pion-rejector 2 (PRL2). The RHRS blocks are arranged in two layers. The first layer is referred to as the `preshower' and is made up of two columns of 24 PMTs each. The second layer, called the `shower', is comprised of five columns of 16 blocks each \cite{Article:HallA} \cite{Thesis:Ye}. 

\begin{figure}[!ht]
\begin{center}
\includegraphics[width=0.7\linewidth]{EM_Cal.png}
\end{center}
\caption[Hall A Electromagnetic Calorimeters]{
{\bf{Hall A Electromagnetic Calorimeters.}} The LHRS blocks are set in two columns of 17 blocks each with the first layer called pion-rejector 1 (PRL1) and the second layer called pion-rejector 2 (PRL2). The RHRS blocks are arranged in two layers. The first layer is referred to as the `preshower' and is made up of two columns of 24 PMTs each. The second layer, called the `shower', is comprised of five columns of 16 blocks each \cite{Article:HallA} \cite{Thesis:Ye}. Image from \cite{Article:HallA}.}
\label{fig:em_cal}
\end{figure}

The calorimeters provide a second level of particle identification for the experiments. When charged particles pass through the lead-glass blocks they are slowed down by their interactions with the nuclei of the particles in those blocks (lead-glass is chosen for its heavy nuclei). The energy lost by this deceleration from nuclei is emitted as photons via Bremsstrahlung radiation. These released photons continue on and produce electron-positron pairs through pair production which again create more Bremsstrahlung radiation. This alternating process creates a `shower' of photons, electrons, and positrons in the calorimeter. The PMTs then detect the Cherenkov light from these electrons and positrons in the lead-glass. 

Heavier particles compared to electrons, like pions, require a much greater path length in the lead-glass blocks to release a shower of particles because pions mostly interact through ionization not Bremsstrahlung. This means that only electrons should create significant showers in the calorimeters. Although they still provide accurate particle identification the PRL1 and PRL2 in the LHRS do not totally absorb all of the electrons' energy, whereas the preshower and shower in the RHRS are total absorbers \cite{Thesis:Ye}. The PMTs attached to the lead-glass blocks must also be calibrated in the same manner as the GC discussed in \ref{ssec:gc}. For a full accounting of the calorimeter calibration procedure for experiment E08-014 see section 4.2.2 of \cite{Thesis:Ye}.

\section{Data Acquisition System}
\label{sec:daq}

A schematic of the data acquisition system (DAQ) is shown in Figure \ref{fig:daq}. To understand the Hall A DAQ let us follow the data signal chronologically through the various components. This journey begins with the analog signals produced by each of the detector components in the detector stack described in section \ref{sec:detectors}. Some signals are digitized as they pass through the system. Detector signals enter the system through one of two ways. The first is by being sent into a majority logic unit (MLU) which then uses various combinations of signals to create triggers and coincidences between the detectors. For example, it may create a new trigger output that is a coincidence of the S$_1$, S$_{2m}$, and GC signals which was used as the main production trigger in experiment E08-014 \cite{DAQ}. 

\begin{figure}[!ht]
\begin{center}
\includegraphics[width=0.7\linewidth]{DAQ.png}
\end{center}
\caption[Schematic of Hall A DAQ and Trigger System]{
{\bf{Schematic of Hall A DAQ and Trigger System.}} Image from \cite{DAQ}.}
\label{fig:daq}
\end{figure}

These MLU output signals are then transported via NIM-ECL translator to a trigger supervisor (TS) module. The TS module is the central control point for the DAQ. This module decides whether or not to accept a trigger from the detectors. If a trigger is accepted the TS then creates a level one accept (L1A) signal. These L1A signals provide timing and gating information for the electronics such as ADCs and TDCs. The TS can also prescale the triggers if the rate is higher than desired. The L1A signals are then sent to the retiming module where they await the trigger signals from the detectors that enter the DAQ via the second method \cite{DAQ}.

The second way into the DAQ is by taking the signals from the detectors directly into NIM cards. Here, using the NIM cards such as amplifiers and coincidence modules, the detector signals are combined to make triggers matching those produced by the MLU. These signals are called the retiming signals and are also sent to the retiming module where they meet up with the L1A signals. The L1A signals are then retimed to match the detector signals. Now the gates and stops determined by the detector signals are in coincidence with the L1A signal \cite{DAQ}.

This coincidence signal is sent to the transition module (TM) which acts as an interface between the TS and the electronics. The TM then copies the signals for the triggers, gates, and stops and sends them to the VME and FASTBUS crates containing the ADCs, TDCs, and scalers which are controlled by read out controllers (ROCs). These front end electronics then begin recording data based on the trigger information from the detectors and a unique event number for this trigger is created that contains all of the front end electronics information for that event \cite{DAQ}.

The Hall A DAQ also contains two timing scalers (clocks), a fast (104 kHz) and a slow clock (1024 Hz), which are used to time the experimental run and are used to normalize experimental data. The timing from the clocks allows the missing data from computer dead-time to be understood. When high data rates are passing through the DAQ the computer processing time of that data becomes a limiting factor. While one event is being processed the next event that arrives before the first event is processed is missed by the system and not recorded. This problem can be alleviated in several ways. 

The first method is to `prescale' the data meaning that the system doesn't try to read all of the data. For example if the system is set to a prescale factor of five the system will only record one fifth of the data in the stream. This allows the electronics to keep up and the data can be normalized back by knowing the prescale factor. The proportion of time the DAQ was able to accept signals is known as the `live time' given in Equation \ref{eq:lt}. Here $ps_i$ is the prescale value of trigger $i$, $T_i^{acc}$ is the number of trigger $i$ accepted and recorded by the DAQ, and $T_i$ is the total number of trigger $i$ created by the detectors. The `dead-time' is then equal to one minus the live time. \cite{DAQ}. 

\begin{equation} \label{eq:lt}
	LT = \frac{ps_i \times T_i^{acc}}{T_i}
	\text{\equationlabels{Live Time}}
\end{equation}

Another method of preventing data from being lost to electronic dead-time is by placing the data in a buffer before analyzing it. The buffer is essentially an electronic storage bin where the raw data is kept in computer memory until the system has processed all the events preceding the buffered event. As long as the buffer is large enough all the events can be processed eventually without loss. This allows the experiment to run as fast as the front end electronics are capable of running \cite{DAQ}. 

The dead-time can be monitored by several methods. One is by the electronic dead-time monitor (EDTM) signal in the DAQ. The EDTM sends constant pulser signals into the S$_1$, S$_{2m}$, and GC data streams. If the DAQ is free these signals will be accepted, but if the DAQ is busy they will be rejected. The number of accepted signals in the final data can be compared to the known EDTM pulser rate to calculate the dead-time. The TS module also sends out an electronic busy signal when it is processing data and unable to accept new triggers. This busy signal is essentially an internal constant pulser signal that is gated by the TS entering the busy state, and provides a second method of calculating the dead-time \cite{DAQ}. %Because the pulser rate is known the number of accepted signals in the final data can be compared to the known pulser signals sent to calculate the dead-time.

Now we have seen how the data is gathered from the detectors, but how is this system controlled and how does the data come to be in a datafile useful for analysis? The CEBAF Online Data Acquisition (CODA) program, seen schematically in Figure \ref{fig:coda}, allows the DAQ to be controlled remotely and builds the experiment's run datafiles. The main CODA GUI is used to begin and end each experimental run. The GUI communicates these commands to a readout controller (RC) server that ties all of CODA's components together. Once the RC server gets the command to begin a run it informs the electronics described above to begin the process of taking data \cite{DAQ}. 

\begin{figure}[!ht]
\begin{center}
\includegraphics[width=0.7\linewidth]{CODA.png}
\end{center}
\caption[Schematic of CODA]{
{\bf{Schematic of CODA.}} Image from \cite{DAQ}.}
\label{fig:coda}
\end{figure}

Once a trigger is accepted by the DAQ the crates containing ADCs and TDCs are read out by the ROCs. The ROCs pass this information on to CODA's event builder (EB). The EB then takes these disparate pieces of data from the various components and organizes them all into one file using the CODA formatting structure. The EB then sends this file in the form of a single event to the event transfer (ET) system. The ET gathers these events and then sends them on to the event recorder (ER) where they are finally written to permanent storage such as tape file. The resulting CODA files can then be decoded to create ROOT files which are then used in the offline analysis of the experimental runs. These data files contain scaler readouts every one to four seconds. They also log the Experimental Physics and Industrial Control System (EPICS) data, which contains information from the hall like target position, spectrometer angle, BCM readings, BPM readings, beam energy, and spectrometer magnet information, periodically every few seconds \cite{Thesis:Ye} \cite{DAQ}. 

\section{Triggers}
\label{sec:triggers}

Experiment E08-014 used seven different triggers. Each of these triggers is made up of a combination of the signals from the S$_1$ and/or S$_{2m}$ scintillator planes and the sum of the ten GC signals. The main production trigger is a coincidence of S$_1$, S$_{2m}$, and the GC sum signals called T1 (T3) for the RHRS (LHRS), and is denoted by (S$_1$ \& S$_{2m}$ \& GC). Two efficiency triggers designed to measure the efficiency of the main production triggers T1 (T3) are T2 (T4) for the RHRS (LHRS). These are made up of a coincidence of the GC signal and only one of either S$_1$ or S$_{2m}$. The triggers T6 (T7) for the RHRS (LHRS) are the coincidences of the S$_1$ and S$_{2m}$ scintillator planes. Since these triggers do not involve the GC they recorded pion events as well and were thus useful for particle identification purposes. Finally trigger T5, the coincidence of T1 and T3, was disabled for this experiment \cite{Thesis:Ye}. 

\section{High Resolution Spectrometer Optics}
\label{sec:optics}

Once the charged particles are bent through a spectrometer and into the detector stack they first pass through the two VDCs set in the HRS's focal plane as described in \ref{ssec:vdcs}. The VDCs give the particle's location in the focal plane which can then be used to reconstruct where the electron interacted with the target, the `reaction vertex', as well as its trajectory at the target. This reconstruction is done by applying an optics matrix determined by the characteristics of the HRS. This matrix needs to be calibrated for each experiment as there are always slight changes in the relative positions of the target, spectrometer, and detectors as well as changes in spectrometer magnet behavior. This section describes the coordinate systems used in Hall A and the standard procedure for optimizing the optics matrix. 

\subsection{Hall A Coordinate Systems}
\label{ssec:coordinates}

Hall A has five different coordinate systems which can all be related to one another. 

\noindent $\bullet$ \textbf{Hall Coordinate System (HCS):}

The HCS can be seen in Figure \ref{fig:hcs}. The origin of the HCS begins at the center of Hall A which is located at the intersection of the electron beam and the target's vertical axis of symmetry, $\hat{x}$. The $\hat{z}$ direction is defined in the direction of the electron beam's travel. The $\hat{y}$ direction is defined to be vertically up \cite{optics}.

\begin{figure}[!ht]
\begin{center}
\includegraphics[width=0.5\linewidth]{Hall_Coordinate_System.png}
\end{center}
\caption[Hall Coordinate System]{
{\bf{Hall Coordinate System.}} Image from \cite{optics}.}
\label{fig:hcs}
\end{figure}

\noindent $\bullet$ \textbf{Target Coordinate System (TCS):}

Each spectrometer has their own target coordinate system as seen in Figure \ref{fig:tcs}. The $z$ axis is defined by drawing a line perpendicular from the sieve slit surface of the spectrometer and the midpoint of the central sieve slit hole. The $\hat{z}_{tg}$ direction is defined to be pointing away from the target. When optimally aligned the spectrometer is pointing at the hall center with the sieve slit being perfectly centered causing the $\hat{z}_{tg}$ vector to pass through the hall center \cite{optics}. 

\begin{figure}[!ht]
\begin{center}
\includegraphics[width=1.0\linewidth]{Target_Coordinate_System_Better.png}
\end{center}
\caption[Target Coordinate System]{
{\bf{Target Coordinate System.}} Image from \cite{Thesis:Ye}.}
\label{fig:tcs}
\end{figure}

With this optimal alignment the distance separating the hall center and the central midpoint of the sieve slit hole is defined as the spectrometer constant, $Z_0$. By definition the TCS origin is located a distance $Z_0$ from the sieve surface on the $\hat{z}_{tg}$ axis, and in the optimal case is the same as the hall center. The  $\hat{x}_{tg}$ direction is defined to be parallel to the sieve plate surface pointing downwards. Finally the in-plane and out-of-plane angles, $\phi_{tg}$ and $\theta_{tg}$, are defined as $\frac{dy_{tg}}{Z_0}$ and $\frac{dx_{tg}}{Z_0}$ respectively \cite{optics}.

\noindent $\bullet$ \textbf{Detector Coordinate System (DCS):}

The origin of the detector coordinate system is located inside of the first VDC on either spectrometer as shown in \ref{fig:dcs}. This origin is defined by the intersection of wire 184 at the center of the U1 wire plane of the bottom VDC with the projection of wire 184 at the center of the V1 wire plane of the bottom VDC. $\hat{y}$ is defined in the direction parallel to the short axis of the VDC pointing to the left of the direction of the particles entering the VDC. $\hat{x}$ is defined along the longer VDC axis pointing away from the hall center. The $\hat{z}$ direction is defined as vertically up. For a more detailed description of the coordinates and how to calculate the detector vertex see \cite{optics}.

\begin{figure}[!ht]
\begin{center}
\includegraphics[width=0.5\linewidth]{Detector_Coordinate_System.png}
\end{center}
\caption[Detector Coordinate System]{
{\bf{Detector Coordinate System.}} Image from \cite{optics}.}
\label{fig:dcs}
\end{figure}

\noindent $\bullet$ \textbf{Transport Coordinate System (TRCS):}

The transport coordinate system is defined by rotating the DCS 45$\degree$ about its $\hat{y}_{det}$ axis as shown in Figure \ref{fig:trcs} \cite{optics}.

\begin{figure}[!ht]
\begin{center}
\includegraphics[width=0.5\linewidth]{Transport_Coordinate_System.png}
\end{center}
\caption[Transport Coordinate System]{
{\bf{Transport Coordinate System.}} Image from \cite{optics}.}
\label{fig:trcs}
\end{figure}

\noindent $\bullet$ \textbf{Focal Plane Coordinate System (FCS):}

The FCS is another rotated coordinate system as shown in Figure \ref{fig:fcs}. It is created by rotating the DCS about its $\hat{y}_{det}$ axis by and angle $\rho$. $\rho$ is defined as the angle between the $\hat{z}_{det}$ axis and the central ray passing through the target, i.e. $\phi_{tg}$ = $\theta_{tg}$ = 0, for the corresponding relative momentum given in Equation \ref{eq:rel_mom} \cite{Thesis:Ye} \cite{optics}.

\begin{figure}[!ht]
\begin{center}
\includegraphics[width=0.7\linewidth]{Focal_Plane_Coordinate_System.png}
\end{center}
\caption[Focal Plane Coordinate System]{
{\bf{Focal Plane Coordinate System.}} Image from \cite{optics}.}
\label{fig:fcs}
\end{figure}

\begin{equation} \label{eq:rel_mom}
	\delta p = \frac{p-p_0}{p_0}
	\text{\equationlabels{Momentum Fraction (dP)}}
\end{equation}

\subsection{Spectrometer Optics Optimization Procedure}
\label{ssec:optics_optimization} 

Now that we understand the coordinate systems in Hall A we must calibrate the optics of the HRS so that we can reconstruct what happens at the target when the electrons scatter. To do this we create an optics matrix that links the focal plane coordinates to the target coordinates. To first order this matrix can be written as Equation \ref{eq:matrix}. A set of polynomial tensors in $x_{fp}$ can then describe the target variables in terms of the focal plane variables as shown in Equations \ref{eq:delta_tg}, \ref{eq:theta_tg}, \ref{eq:y_tg}, and \ref{eq:phi_tg}. These tensors can all be written similar to the one written in Equation \ref{eq:tensor} \cite{optics}.

\begin{equation} \label{eq:matrix}
	\left[ \begin{matrix}
		\delta \\
		\theta \\
		y	   \\
		\phi
	\end{matrix} \right]_{tg}
	=
	\left[ \begin{matrix}
		\langle \delta | x \rangle & \langle \delta | \theta \rangle & 0 & 0 \\
		\langle \theta | x \rangle & \langle \theta | \theta \rangle & 0 & 0 \\
		0 & 0 & \langle y | y \rangle & \langle y | \phi \rangle \\
		0 & 0 & \langle \phi | y \rangle & \langle \phi | \phi \rangle 
	\end{matrix} \right]
		\left[ \begin{matrix}
		x \\
		\theta \\
		y	   \\
		\phi
	\end{matrix} \right]_{fp}
	\text{\equationlabels{Optics Matrix}}
\end{equation}

\begin{equation} \label{eq:delta_tg}
	\delta = \sum_{j,k,l} D_{j,k,l} \theta_{fp}^j y_{fp}^k \phi_{fp}^l
	\text{\equationlabels{Momentum Fraction ($\delta$)}}
\end{equation}

\begin{equation} \label{eq:theta_tg}
	\theta_{tg} = \sum_{j,k,l} T_{j,k,l} \theta_{fp}^j y_{fp}^k \phi_{fp}^l
	\text{\equationlabels{$\theta_{tg}$}}
\end{equation}

\begin{equation} \label{eq:y_tg}
	y_{tg} = \sum_{j,k,l} Y_{j,k,l} \theta_{fp}^j y_{fp}^k \phi_{fp}^l
	\text{\equationlabels{$y_{tg}$}}
\end{equation}

\begin{equation} \label{eq:phi_tg}
	\phi_{tg} = \sum_{j,k,l} P_{j,k,l} \theta_{fp}^j y_{fp}^k \phi_{fp}^l
	\text{\equationlabels{$\phi_{tg}$}}
\end{equation}

\begin{equation} \label{eq:tensor}
	D_{j,k,l} = \sum_{i=1}^m C_i^{D_{j,k,l}} x_{fp}^i
	\text{\equationlabels{Optics Tensors ($D_{j,k,l}$)}}
\end{equation}

This matrix is calibrated by placing a sieve slit, shown in Figure \ref{fig:sieve}, over the spectrometer entrance. The sieve has a series of holes with a well known pattern. This pattern then shows up in the focal plane data and by knowing the hole locations well the focal plane data can be correlated with the sieve holes.  The variables in Equations \ref{eq:theta_tg}, \ref{eq:y_tg}, and \ref{eq:phi_tg} described above turn out to be impractical to work with so an additional three variables are defined. %A set of three more user friendly variables are defined. 

The first of these variables is $Z_{react}$ which describes the point of interaction between the beam and the target given in Equation \ref{eq:zreact}. The second and third are $x_{sieve}$, Equation \ref{eq:xsieve}, and $y_{sieve}$, Equation \ref{eq:ysieve}, which describe the horizontal and vertical positions of the sieve plate respectively. In these equations $L$ and $D_y$ are defined as they were in the TCS above, $\theta_0$ is the spectrometer angle, and $x_{beam}$ is the horizontal beam position \cite{optics}. For the specific results of the optics calibration procedure for experiment E08-014 see section 4.3.2 of \cite{Thesis:Ye}.

\begin{figure}[!ht]
\begin{center}
\includegraphics[width=0.3\linewidth]{Sieve.png}
\end{center}
\caption[Optics Sieve Plate for E08-014]{
{\bf{Optics Sieve Plate for E08-014.}} The two larger holes make it possible to determine the plate's orientation when performing the optics calibration. Image from \cite{Thesis:Ye}.}
\label{fig:sieve}
\end{figure}

\begin{equation} \label{eq:zreact}
	Z_{react} = \frac{-\left( y_{tg} + D_y \right) + x_{beam}\left( \cos\left( \theta_0 \right) - \phi_{tg} \sin\left( \theta_0 \right) \right)}{\phi_{tg} \cos\left( \theta_0 \right) + \sin\left( \theta_0 \right)}
	\text{\equationlabels{Interaction Vertex ($Z_{react}$)}}
\end{equation}

\begin{equation} \label{eq:xsieve}
	x_{sieve} = x_{tg} + L \theta_{tg}
	\text{\equationlabels{Horizontal Sieve Position ($x_{sieve}$)}}
\end{equation}

\begin{equation} \label{eq:ysieve}
	y_{sieve} = y_{tg} + L \phi_{tg}
	\text{\equationlabels{Vertical Sieve Position ($y_{sieve}$)}}
\end{equation}