% Introduction
\chapter{Experimental Setup} % Main chapter title
\label{ch:experiment} % For referencing the chapter elsewhere, use 

\section{Overview}
\label{sec:overview}

The Thomas Jefferson National Accelerator Facility (Jefferson Lab or JLab) located in Newport News Virginia uses a continuous electron beam accelerator facility (CEBAF) to perform electron scattering experiments as a means of studying nuclear structure. Jefferson lab consists of four experimental halls designated Hall A, B, C, and the newly commissioned Hall D as in figure ~\ref{fig:jlab}. The facility is capable of creating electron beams of energies as high as 12 GeV and supplying those electrons to the four halls simultaneously. The 12 GeV capability is a recently completed upgrade and during the time the experiment discussed in this thesis, experiment E08-014, ran the facility was limited to 6 GeV beam energies.  

\begin{figure}[!ht]
\begin{center}
\includegraphics[width=0.7\linewidth]{JLab_Layout.png}
\end{center}
\caption{
{\bf{Thomas Jefferson National Accelerator Facility.}} CEBAF is the ring connecting to the experimental halls. Hall D is now located at the top right of the image. Image from ~\cite{Article:HallA}.}
\label{fig:jlab}
\end{figure}

\section{Experiment E08-014}
\label{sec:x>2}
Experiment E08-014 ran in Jefferson Lab's Hall A in 2011. The experiment used electron scattering to measure the inclusive cross sections of various targets using both of Hall A's high resolution spectrometers (HRSs). E08-014 aimed to compare heavy targets to two and three-nucleon targets to study the short range correlations (SRC) between these two and three-nucleon clusters. To this end inclusive cross sections for $^2$H, $^3$He, $^4$He, $^12$C, $^{40}$Ca, and $^{48}$Ca were measured in the region of 1.1 GeV/c $<$ Q$^2$ $<$ 2.5 GeV/c. This experiment covered the $x_{Bj}$ range encompassing the quasielastic (QE) region up to $x_{Bj}$ greater than 3 ~\cite{Thesis:Ye} ~\cite{src_website}. 

While experiment E08-014 focused on the QE region of electron scattering, one kinematic region, KIN 3.2 in figure ~\ref{fig:kin3.2}, also included elastically scattered electrons. The elastic events can be seen by plotting the scattering angle of the electron versus the scattered electron's energy, $E'$, as shown in figure ~\ref{fig:elastic_band}. The resulting curve (red) gives the elastic scattering band for $^3$He. When this band is compared with the spectrometer's upper and lower acceptance in energy and angle, represented by the black lines, it becomes apparent that the elastic band passes through KIN 3.2, and thus we expect to find $^3$He elastic data in KIN 3.2.  

\begin{figure}[!ht]
\begin{center}
\includegraphics[width=0.7\linewidth]{SRC_Kinematics.png}
\end{center}
\caption{
{\bf{Kinematic Coverage of Experiment E08-014.}} The elastic $^3$He data is located in KIN 3.2. Image from ~\cite{Thesis:Ye}.}
\label{fig:kin3.2}
\end{figure}

\begin{figure}[!ht]
\begin{center}
\includegraphics[width=0.7\linewidth]{Elastic_Band.png}
\end{center}
\caption{
{\bf{Elastic Band for $^3$He.}} It is clear that the red elastic band passes through the box made by the intersecting black lines representing the maximum and minimum spectrometer acceptances in energy and angle for KIN 3.2.}
\label{fig:elastic_band}
\end{figure}

These elastic events were scattered from a gaseous $^3$He target allowing for the extraction of an elastic cross section. This cross section is located in the little studied region of Q$^2 = $ 35 fm$^{-2}$ as seen in figure \ref{fig:jlab_3he}. This understudied region is interesting because it has the potential to constrain and improve previous fits of the $^3$He form factors. 

\begin{figure}[!ht]
\begin{center}
\includegraphics[width=0.7\linewidth]{JLab_3He_High_Q2_Charge_FF_Clean_Arrow.png}
\end{center}
\caption{
{\bf{Location of New Elastic Measurement.}} This plot shows the $Q^2$ region of the $^3$He electric form factor, $F_{ch}$, where the new elastic $^3$He measurement is located. Image from ~\cite{Article:Alex}.}
\label{fig:jlab_3he}
\end{figure}

