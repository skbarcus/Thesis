% Introduction
\chapter{Elastic Electron Scattering} % Main chapter title
\label{ch:elastic} % For referencing the chapter elsewhere, use 

Electron scattering is one of the most powerful tools available to physicists to study the nature of nuclear matter. When electrons are accelerated to high energies by means of a particle accelerator, like Jefferson Lab's CEBAF, and fired at a nuclear target the electrons scatter according to the nuclear structure of the target. This scattering is well described by quantum electrodynamics (QED). Thus, by measuring the scattered electrons (and occasionally other particles), called `semi-inclusive electron scattering', the nuclear structure of the target is revealed. 

\section{Kinematics of Elastic Electron Scattering}
\label{sec:kinematics}

The process of elastic electron scattering via the electromagnetic process is shown in Figure \ref{fig:elastic_scattering}. An incident electron, with four-momentum $k = (E_0,\textit{\textbf{k}})$, exchanges a virtual photon, $q = (\nu,\textit{\textbf{q}})$, with a target in the target's rest frame, $p = (M,0)$. The virtual photon exchanges energy and momentum causing the electron to scatter with a scattering angle $\theta$ and four-momentum $k' = (E',\textit{\textbf{k}}')$. The proton is also scattered with four-momentum $p'$, but the proton is not measured in inclusive electron scattering. When the kinetic energy of this scattering process is conserved the process is called `elastic'. 

\begin{figure}[!ht]
\begin{center}
\includegraphics[width=0.7\linewidth]{Elastic_Electron_Scattering_Clean.png}
\end{center}
\caption[Elastic Electron Scattering]{
{\bf{Elastic Electron Scattering.}} An incident electron interacts with a target by exchanging a virtual photon causing the electron to scatter.}
\label{fig:elastic_scattering}
\end{figure}

When the scattering process is elastic the entire process can be described by two variables. These variables are the scattering angle, $\theta$, and the initial energy, $E_0$. By using conservation of energy and momentum as well as applying the Einstein relation the scattered electron's final energy, $E'$, is found to be given by Equation \ref{eq:E'}. The energy lost by the incident electron during scattering, $\nu$, is given by Equation \ref{eq:nu}. The strength of the interaction, how much four-momentum is transferred to the target by the electron, is generally described as in Equation \ref{eq:Q^2}. $Q^2$ is given in units of (GeV/c)$^2$ which can be converted to fm$^{-2}$ by multiplying the (GeV/c)$^2$ result by a value of $\approx$ 25.7.

\begin{equation} \label{eq:E'}
	E' = \frac{E_0}{1+\frac{2E_0}{M}\sin^2\left(\frac{\theta}{2}\right)}
	\text{\equationlabels{Elastically Scattered Electron Energy (E')}}
\end{equation}

\begin{equation} \label{eq:nu}
	\nu = E_0-E'
	\text{\equationlabels{Energy Lost by Incident Electron ($\nu$)}}
\end{equation}

\begin{equation} \label{eq:Q^2}
	Q^2 = -q^2 = 4E_0E'\sin^2\left(\frac{\theta}{2}\right)
	\text{\equationlabels{Four-Momentum Transfer ($Q^2$)}}
\end{equation}

Another useful quantity to define is Bjorken $x$, denoted $x_{Bj}$, given by Equation \ref{eq:xbj}. This variable is interpretable as the fraction of the nucleon's three-momentum carried by the quark struck by the electron in the Breit frame. For a single nucleon $0 \leq x_{Bj} \leq 1$. However, for a nucleus $0 \leq x_{Bj} \leq A$, where $A$ is the atomic mass number of the target. The elastic peak can then be found at $x_{Bj} \approx A$. Taking $^3$He as an example one would then expect to find the elastic peak at $x_{Bj} = 3$.

\begin{equation} \label{eq:xbj}
	x_{Bj} = \frac{Q^2}{2M(E_0-E')}
	 \text{\equationlabels{Bjorken x ($x_{Bj}$)}}
\end{equation}

\section{Differential Cross Section}
\label{sec:xs}

The differential cross section is proportional to the probability that an electron incident on a target will interact with that target. This can be thought of as the `size' of the interaction. However, for a scattered electron to be measured it must be seen by a detector which measures within some acceptance $d\Omega$. Therefore, the differential cross section measures the probability that an electron will be scattered from a target into solid angle $d\Omega$.

Following the discussion of cross sections and form factors laid out in chapters five and six of \cite{Book:Povh}, and largely adopting their notation, let us begin with the classical case of particles scattering from a fixed target nuclei. The Rutherford scattering equation can be obtained classically as Equation \ref{eq:rutherford} or through non-relativistic quantum mechanics by assuming that the wave functions of the electron are plane waves (Born approximation).   

\begin{equation} \label{eq:rutherford}
	\left(\frac{d\sigma}{d\Omega}\right)_{\text{Rutherford}} = \frac{\left(zeZe\right)^2}{(4\pi\epsilon_0)^2(4E_0)^2\sin^4(\theta/2)}
	\text{\equationlabels{Rutherford Scattering as a Function of Angle}}
\end{equation}

\noindent Here $z$ ($Z$) is the atomic number of the incident particle (target), $e$ is the elementary charge, $\epsilon_0$ is the vacuum permittivity, and $E_0$ is the incident particle's initial energy. The Rutherford equation can also be written as Equation \ref{eq:rutherford_2} following the quantum mechanical derivation, where $\alpha$ is the fine structure constant, $\hbar$ is the reduced Planck's constant, and $c$ is the speed of light. Notice that the cross section falls off like $\frac{1}{q^4}$ indicating that the interaction probability falls off rapidly with increased momentum transfer \cite{Book:Povh}.

\begin{equation} \label{eq:rutherford_2}
	\left(\frac{d\sigma}{d\Omega}\right)_{\text{Rutherford}} = \frac{4Z^2\alpha^2\left(\hbar c\right)^2E'^2}{|qc|^4}
	\text{\equationlabels{Rutherford Scattering as a Function of Momentum Transfer}}
\end{equation}

The Rutherford equation does not account for relativity, spin, or target recoil. To begin accounting for these quantities let us continue the derivation by adding relativity. To do this we add a second term to the cross section that is scaled by a constant and introduces angular dependence to Equation \ref{eq:rutherford_2}. This yields the preliminary version of the Mott equation while still neglecting recoil as seen in Equation \ref{eq:mott_no_recoil}.

\begin{equation} \label{eq:mott_no_recoil}
	\left(\frac{d\sigma}{d\Omega}\right)_{\substack{ \text{Mott} \\ \text{No Recoil}}} = \left(\frac{d\sigma}{d\Omega}\right)_{\text{Rutherford}} \left( 1-\beta^2 \sin^2 \left( \frac{\theta}{2} \right) \right)
	\text{\equationlabels{Mott Equation No Recoil with Relativistic Term}}
\end{equation}

\noindent Here $\beta = \frac{v}{c}$ with $v$ being the velocity of the incident particle \cite{Book:Povh}. 

By taking the incident particle's velocity to its maximum value of $c$ we see $\beta$ goes to unity. Then Equation \ref{eq:mott_no_recoil} simplifies via trigonometric identity to Equation \ref{eq:mott_no_recoil_simple}.

\begin{equation} \label{eq:mott_no_recoil_simple}
	\left(\frac{d\sigma}{d\Omega}\right)_{\substack{ \text{Mott} \\ \text{No Recoil}}} = \left(\frac{d\sigma}{d\Omega}\right)_{\text{Ruth.}} \cos^2 \left( \frac{\theta}{2} \right) = \left(\frac{d\sigma}{d\Omega}\right)_{\text{Ruth.}} = \frac{4Z^2\alpha^2\left(\hbar c\right)^2E'^2}{|qc|^4} \cos^2 \left( \frac{\theta}{2} \right) 
	\text{\equationlabels{Mott Equation No Recoil Explicit}}
\end{equation}

\noindent Equation \ref{eq:mott_no_recoil_simple} now accounts for relativity, but also accounts for spin by suppressing scattering through 180$\degree$ for a spinless target which is forbidden by conservation of helicity \cite{Book:Povh}. 

\section{Nuclear Form Factors}
\label{sec:ffs}

So far we have only considered scattering off of a pointlike target. Real nuclear targets are made up of atoms with differing geometry. To quantize the spatial extent of a target we introduce the concept of form factors. Form factors contain all of the transverse spatial information about the target. Multiplying the Mott cross section, excepting recoil, by the form factor we get the experimental cross section as in Equation \ref{eq:xs_exp_ff}. By measuring the experimental cross section of a target at various angles for a single energy and dividing out the Mott cross section the form factor, $F(q^2)$, of a target can be determined. 

\begin{equation} \label{eq:xs_exp_ff}
	\left(\frac{d\sigma}{d\Omega}\right)_{\text{exp}} = \left(\frac{d\sigma}{d\Omega}\right)_{\substack{ \text{Mott} \\ \text{No Recoil}}} |F(q^2)|^2
	\text{\equationlabels{Form Factor Term in Cross Section}}
\end{equation}

Continuing to assume that recoil is negligible as well as the validity of the Born approximation the target's form factors can be written as the Fourier transform of a charge distribution, $\rho(x)$, as in Equation \ref{eq:fourier}. If the charge distribution is spherically symmetric then the form factor equals the right side of Equation \ref{eq:fourier} with the integral of $\rho(r)$ normalized to unity.

\begin{equation} \label{eq:fourier}
	F(q^2) = \int e^{\frac{iq \cdot x}{\hbar}} \rho(x) d^3x \xrightarrow{x \xrightarrow{} r} 4\pi \int \rho(r) \frac{\sin\left( |q|r/\hbar \right)}{|q|r/\hbar} r^2 dr
	\text{\equationlabels{Form Factor as a Fourier Transform}}
\end{equation}

\noindent This procedure can be inverted to find the charge distribution of a target from its form factor as in Equation \ref{eq:inverse_fourier} \cite{Book:Povh}.

\begin{equation} \label{eq:inverse_fourier}
	\rho(r) = \frac{1}{(2\pi)^3} \int F(q^2) e^{\frac{-iq \cdot x}{\hbar}} d^3q 
	\text{\equationlabels{Charge Density as an Inverse Fourier Transform}}
\end{equation}

Let us now examine a simple example of a charge distribution and its form factor. Assume that there is a charge distribution in the shape of a hard sphere, i.e. a solid ball of constant charge density that drops to zero beyond a certain radius. This is a reasonable first order model for the charge distribution of an atom. The top plot in Figure \ref{fig:hard_sphere} shows a hard sphere of charge density. The bottom plot shows the form factor, Fourier transform, of the upper hard sphere plot. Now we can see that the form factor of a hard sphere distribution of charge yields an oscillatory and decreasing form factor \cite{Book:Povh}.

\begin{figure}[!ht]
\begin{center}
\includegraphics[width=1.\linewidth]{Hard_Sphere_FT_Clean.png}
\end{center}
\caption[Hard Sphere Charge Distribution Form Factor]{
{\bf{Hard Sphere Charge Distribution Form Factor.}} Taking the Fourier transform of a hard sphere charge distribution (top plot) yields an oscillatory form factor (bottom plot).}
\label{fig:hard_sphere}
\end{figure} 

Next let us examine what a form factor for $^3$He, Figure \ref{fig:3he_rep_fit}, and a form factor for $^3$H, Figure \ref{fig:3h_rep_fit}, look like. They each have the oscillatory behavior predicted by the hard sphere of charge model indicating that their charge densities can be modelled similarly (in reality their charge densities decrease gradually with distance and not all at once). It is also interesting to study the form factors of individual nucleons like the proton. One might expect to find minima in the proton's form factors as well, however no such minima are observed. This is because the proton's form factors have an approximately dipole form, and the Fourier transform of a dipole is an exponential. This explains why we do not observe minima in the proton's form factors. However, it must be noted that proton's form factors cannot be found by taking the Fourier transform of the electric or magnetic charge densities since for the Fourier transform to be valid the recoil of the system must be small. Since the proton is relatively light the recoil is not negligible and the Fourier transform is not valid.   

As \cite{Book:Povh} points out the location of the diffractive minima can be used to learn about the size of the target. For a hard sphere of charge the charge radius, $R$, is roughly given by Equation \ref{eq:minima}, where $\frac{q}{\hbar}$ is the location of the first minima. One can also study the charge radius by examining the behavior of the charge density as it approaches zero. This can be seen by expanding the form factor in $q$ from the first integral in Equation \ref{eq:fourier}. Euler's formula, $e^{ix} = \cos(x)+i\sin(x)$, begins the expansion of the exponent, but we are still left with a troublesome $i\sin(x)$ term. This $i\sin(x)$ term can be eliminated by assuming that the wavelength of the electron, $\frac{\hbar}{q}$, is much larger than the charge radius, $R$, as in Equation \ref{eq:large_wavelength}. %Let us use this relation to compare the locations of the first minima in $F_{ch}$ for $^3$He (Figure \ref{fig:3he_fch_rep_fit}) and $^3$H (Figure \ref{fig:3h_fch_rep_fit}) ($F_{ch}$ is discussed in more detail below, but it represents the Fourier transform of the electric charge density). We see that the first minimum for $^3$He is found at $Q^2$ $\approx$ 11 fm$^{-2}$, and the first minimum for $^3$H is found at $Q^2$ $\approx$ 13 fm$^{-2}$. Equation \ref{eq:minima} then predicts that $^3$He will have a larger charge radius than $^3$H. 

\begin{equation} \label{eq:minima}
	R \approx \frac{4.5 \hbar}{q}
	\text{\equationlabels{Approximate Charge Radius of a Hard Sphere of Charge}}
\end{equation}

%Let us examine if this prediction agrees with our expectations. $^3$He contains two protons and a neutron, whereas $^3$H contains one proton and two neutrons. The nuclei of these atoms are held together by the strong force while the Coulomb force pushes against this binding force. $^3$He has more charge due to having two protons, and thus the Coulomb force for $^3$He will be stronger than that of $^3$H. Therefore, we expect $^3$He to have a larger charge radius than $^3$H as there is a larger Coulomb force pushing the charge radius out against the strong force. 

\begin{equation} \label{eq:large_wavelength}
	R \ll \frac{\hbar}{q} \implies \frac{Rq}{\hbar} \ll 1
	\text{\equationlabels{Small Electron Wavelength Approximation}}
\end{equation}

Now $i\sin\left(\frac{Rq}{\hbar}\right) \xrightarrow{} 0$ and we can drop the $i\sin(x)$ term leaving only $e^{ix} \approx \cos(x)$. The Taylor expansion of $\cos(x)$ is given by Equation \ref{eq:cos}.

\begin{equation} \label{eq:cos}
	\cos(x) = 1 - \frac{x^2}{2!} + \frac{x^4}{4!} - \frac{x^6}{6!} +...
	\text{\equationlabels{Cosine Taylor Expansion}}
\end{equation}

\noindent Keeping only the first two terms of the $\cos(x)$ expansion we can now rewrite the form factor equation as in Equation \ref{eq:ff_expanded}. Here we have inserted $q \cdot r = |q||r|\cos(\omega)$ where $\omega$ is the angle between $q$ and $r$.

\begin{equation} \label{eq:ff_expanded}
	F(q^2) = \int_0^\infty \int_{-1}^1 \int_0^{2\pi} \rho(r) \left( 1-\frac{1}{2} \frac{|q||r|\cos(\omega)}{\hbar} \right) r^2 d\phi \; d\cos(\omega) \; dr
	\text{\equationlabels{Form Factor (3D Integral)}}
\end{equation}

\noindent Integrating over $\phi$ and $\cos(\omega)$ we obtain Equation \ref{eq:ff_expanded_r} \cite{Book:Povh}.

\begin{equation} \label{eq:ff_expanded_r}
	F(q^2) = 4\pi \int_0^\infty \rho(r) r^2 dr - 4\pi \frac{q^2}{6\hbar^2} \int_0^\infty \rho(r) r^4 dr
	\text{\equationlabels{Form Factor (Radial)}}
\end{equation}

If we require that $\rho(r)$ be normalized such that $4\pi \int_0^\infty \rho(r) r^2 dr = 1$ then we can define the mean square charge radius as in Equation \ref{eq:rms}.

\begin{equation} \label{eq:rms}
	\langle r^2 \rangle = 4\pi \int_0^\infty r^2 \rho(r) r^2 dr
	\text{\equationlabels{Mean Square Charge Radius}}
\end{equation}

\noindent Now Equation \ref{eq:ff_expanded_r} can be rewritten as Equation \ref{eq:ff_rms}.

\begin{equation} \label{eq:ff_rms}
	F(q^2) = 1 - \frac{q^2}{6\hbar^2} \langle r^2 \rangle
	\text{\equationlabels{Form Factor relation to Mean Charge Radius}}
\end{equation}

\noindent Taking the derivative of Equation \ref{eq:ff_rms} with respect to $q^2$ we can extract $\langle r^2 \rangle$ as in Equation \ref{eq:rms_derivative}. So by measuring the form factor of a target at very low $q^2$ one can calculate the mean square radius by finding the slope of the form factor at $q^2=0$ \cite{Book:Povh}.

\begin{equation} \label{eq:rms_derivative}
	\langle r^2 \rangle = -6\hbar^2 \frac{dF(q^2)}{dq^2} |_{q^2=0}
	\text{\equationlabels{Mean Charge Radius from Form Factor First Derivative}}
\end{equation}

At this point we are still claiming that the recoil of the struck particle is negligible. Let us examine the truth of this assumption by comparing the electron energies needed to study nuclear structure to the mass of a typical target, $^3$He. First let us determine approximately what energy an electron needs to be able to probe a target's nuclear structure. The Planck-Einstein relation can guide us with this estimate, and is given in Equation \ref{eq:planck}, where $h$ is Planck's constant and $\nu$ is the electron's frequency. 

\begin{equation} \label{eq:planck}
	E = h\nu
	\text{\equationlabels{Planck-Einstein Relation}}
\end{equation}

\noindent Also remember the relationship between frequency and wavelength given in Equation \ref{eq:wavelength}

\begin{equation} \label{eq:wavelength}
	\nu = \frac{c}{\lambda}
	\text{\equationlabels{Wavelength}}
\end{equation}

\noindent Combining Equations \ref{eq:planck} and \ref{eq:wavelength} gives us Equation \ref{eq:energy_estimate} which allows us to estimate the required electron energies to study nuclear structure. 

\begin{equation} \label{eq:energy_estimate}
	E = \frac{hc}{\lambda}
	\text{\equationlabels{Energy}}
\end{equation}

Now plug in some reasonable values along with the constants. Nuclear targets are generally atoms made up of protons and neutrons so to glean any information about their structure one must use electrons with a wavelength approximately the size of, or smaller than that of, a proton. The proton's radius is about 0.84 fm. Plugging this into the wavelength value in Equation \ref{eq:energy_estimate} along with $h \approx 4.136 \times 10^{-15}$ eVs and $c\approx 3 \times 10^{8}$ m/s yields an electron energy of about 1.477 GeV. $^3$He has a mass of about 2.81 GeV. Clearly the electron energies required to study nuclear structure are no longer negligible when compared to target nuclei like $^3$He. This now requires us to account for target recoil in our cross section calculation from before.

Taking the recoil of the target in to account one finds that the recoil factor is given by Equation \ref{eq:recoil} from Equation \ref{eq:E'}. 

\begin{equation} \label{eq:recoil}
	\frac{E'}{E_0} = \frac{1}{1+\frac{2E_0}{M} \sin^2 \left( \frac{\theta}{2} \right)}
	\text{\equationlabels{Recoil Term}}
\end{equation}

\noindent Adding the factor of $\frac{E'}{E_0}$ to the Mott cross section gives Equation \ref{eq:mott_recoil}.

\begin{equation} \label{eq:mott_recoil}
	\left(\frac{d\sigma}{d\Omega}\right)_{\text{Mott}} = \frac{4Z^2\alpha^2\left(\hbar c\right)^2E'^3}{|qc|^4 E_0} \cos^2 \left( \frac{\theta}{2} \right)
	\text{\equationlabels{Mott Equation adding Recoil}}
\end{equation}

\noindent The Mott cross section can be written in a slightly different form by setting some constants equal to one and rearranging some of the energies as in Equation \ref{eq:mott}. We now have an equation that represents the scattering of electrons off of a pointlike particle \cite{Book:Povh}.

\begin{equation} \label{eq:mott}
	\left(\frac{d\sigma}{d\Omega}\right)_{\text{Mott}} = Z^2 \frac{E'}{E_0} \frac{\alpha^2 \cos^2\left( \frac{\theta}{2} \right)}{4E_0^2 \sin^4\left( \frac{\theta}{2} \right)}
	\text{\equationlabels{Mott Equation}}
\end{equation}

At this point in the analysis we have accounted for charge, spin, relativity, and recoil. However, we have still neglected the fact that many targets have a magnetic moment, $\mu$, that will also interact with the electrons it scatters. Equation \ref{eq:mu} shows the magnetic moment for a pointlike spin 1/2 particle of mass $M$ where the $g$ = 2 factor comes from Dirac theory.  

\begin{equation} \label{eq:mu}
	\mu = g \frac{e\hbar}{4M}
	\text{\equationlabels{Magnetic Moment}}
\end{equation}

To account for magnetic interactions we do as we did in Equation \ref{eq:mott_no_recoil} and introduce a second term to the cross section. This term is scaled by a constant and given an angular dependence of $\sin^2(\theta/2)$ as in Equation \ref{eq:mag}. The angular dependence arises from a need to conserve angular momentum and helicity. We obtain $\tan^2(\theta/2)$ by pulling out a $1/\cos^2(\theta/2)$.

\begin{equation} \label{eq:mag}
	\left(\frac{d\sigma}{d\Omega}\right)_{\substack{ \text{point} \\ \text{spin 1/2}}} = \left( \frac{d\sigma}{d\Omega} \right)_{\text{Mott}} \left( (1-2\tau \; \tan^2\left( \frac{\theta}{2} \right) \right)
	\text{\equationlabels{Cross Section Magnetic Term}}
\end{equation}

\noindent $\tau$ is given by Equation \ref{eq:tau} where $M$ is the target's mass and $Q^2$ is the kinematic variable measuring momentum transfer from Equation \ref{eq:Q^2}. From Equation \ref{eq:mag} it becomes clear that at large angles and large momentum transfers the magnetic interaction becomes significant and cannot be ignored. The effect of the magnetic term causes the cross section to fall off less rapidly than it would if only the electric interaction were relevant \cite{Book:Povh}.

\begin{equation} \label{eq:tau}
	\tau = \frac{Q^2}{4M}
	\text{\equationlabels{$\tau$}}
\end{equation}

Finally let us consider that nuclear targets are not pointlike particles and thus the $g$-factor from Dirac is not precisely equal to two. Instead the $g$-factor is replaced by the proton and neutron magnetic moments given in Equation \ref{eq:mag_mom} quantified in terms of the nuclear magneton, $\mu_N$ given in Equation \ref{eq:mu_n}, where $m_p$ is the proton mass.

 \begin{equation} \label{eq:mag_mom}
	\mu_p = 2.79 \mu_N \;\;\;\;\;\;\;\;\;\; \mu_n = -1.91 \mu_N 
	\text{\equationlabels{Proton and Neutron Magnetic Moments}}
\end{equation}

 \begin{equation} \label{eq:mu_n}
	\mu_N = \frac{e\hbar}{2m_p} = 3.1525 \times 10^{-8} \, eV \, T^{-1}
	\text{\equationlabels{Nuclear Magneton}}
\end{equation}

\noindent We once again need to introduce form factors to describe the structure of the electric and magnetic components of the cross section. In this case we use the Sach's form factors $G_E(Q^2)$ and $G_M(Q^2)$ for the electric and magnetic components respectively. With these form factors we finally arrive at the Rosenbluth Equation \ref{eq:rosenbluth_long} \cite{Book:Povh}.

\begin{equation} \label{eq:rosenbluth_long}
	\left(\frac{d\sigma}{d\Omega}\right) = \left( \frac{d\sigma}{d\Omega} \right)_{Mott} \left[ \frac{G_E^2\left(Q^2\right)+\tau G_M^2\left(Q^2\right)}{1+\tau} + 2 \tau G_M^2\left(Q^2\right) \tan^2\left( \frac{\theta}{2} \right) \right]
	\text{\equationlabels{Rosenbluth Equation with Sach's Form Factors}}
\end{equation}

The physical meaning of these form factors can be further explored by studying their behavior as $Q^2 \rightarrow 0$. $G_E(Q^2)$ describes the electric structure of the target and therefore equals the electric charge of the target at $Q^2 = 0$ in units of elementary charge. $G_M(Q^2)$ describes the magnetic structure of the target and therefore equals the magnetic moment of the target at $Q^2 = 0$ in units of the nuclear magneton. So we find that at $Q^2 = 0$ the proton's form factors are given by Equation \ref{eq:ge_0} and the neutron's form factors are given by Equation \ref{eq:gm_0}.

\begin{equation} \label{eq:ge_0}
	G_E^p\left(Q^2=0\right) = 1 \;\;\;\;\;\;\;\; G_M^p\left(Q^2=0\right) = 2.79
	\text{\equationlabels{Proton Sach's Form Factors}}
\end{equation}

\begin{equation} \label{eq:gm_0}
	G_E^n\left(Q^2=0\right) = 0 \;\;\;\;\;\;\;\; G_M^n\left(Q^2=0\right) = -1.91
	\text{\equationlabels{Neutron Sach's Form Factors}}
\end{equation}

In the literature one finds several other commonly used form factors related to $G_E$ and $G_M$. Being able to translate between these form factors is often necessary to compare the results reported by different groups. The first set of these form factors are the Dirac form factor, $F_1$, and the Pauli form factor, $F_2$. These two form factors are given by Equations \ref{eq:f1} and \ref{eq:f2} in relation to $G_E$ and $G_M$ which are more commonly used because of their physical interpretation.

\begin{equation} \label{eq:f1}
	G_E \left(Q^2\right) = F_1\left(Q^2\right) -\mu \tau F_2\left(Q^2\right)
	\text{\equationlabels{Dirac Form Factor}}
\end{equation}

\begin{equation} \label{eq:f2}
	G_M \left(Q^2\right) = F_1\left(Q^2\right) + \mu F_2\left(Q^2\right)
	\text{\equationlabels{Pauli Form Factor}}
\end{equation}

\noindent One final set of form factors are $F_{ch}$ and $F_m$. Their relations to $G_E$ and $G_M$ are given in Equations \ref{eq:fch} and \ref{eq:fm} \cite{Article:Hand}. These are the form factors used in the sum of Gaussians analysis of the data presented in this thesis. To see what the $^3$He and $^3$H $F_{ch}$ and $F_m$ form factors look like along with some world data see Figures \ref{fig:3he_rep_data} and \ref{fig:3h_rep_data} respectively. 

\begin{equation} \label{eq:fch}
	F_{ch}\left(Q^2\right) = G_E \left(Q^2\right) 
	\text{\equationlabels{Charge Form Factor (F$_{ch}$)}}
\end{equation}

\begin{equation} \label{eq:fm}
	F_{m}\left(Q^2\right) = \frac{G_M \left(Q^2\right)}{\mu} 
	\text{\equationlabels{Magnetic Form Factor (F$_{m}$)}}
\end{equation}

The form factors $G_E$ and $G_M$ can be separated out from experimental data according to the procedure of Rosenbluth separation laid out in \cite{Article:Hand}. First numerous cross section measurements at a fixed $Q^2$ and multiple angles must be taken. To see how to extract $G_E$ and $G_M$ it is helpful to rewrite the Rosenbluth equation in Equation \ref{eq:rosenbluth_long} as Equation \ref{eq:rosenbluth} with $\epsilon$ given in Equation \ref{eq:epsilon}. As an example of what a typical cross section looks like see Figure \ref{fig:3he_cross_section} which shows the $^3$He cross section derived in this thesis.

\begin{equation} \label{eq:rosenbluth}
	\left(\frac{d\sigma}{d\Omega}\right)_{exp} = \left( \frac{d\sigma}{d\Omega} \right)_{Mott} \frac{1}{1+\tau}\left[ G_E^2\left(Q^2\right) + \frac{\tau}{\epsilon} G_M^2\left(Q^2\right) \right]
	\text{\equationlabels{Rosenbluth Equation}}
\end{equation}

\begin{equation} \label{eq:epsilon}
	\epsilon = \left( 1 + 2(1+\tau)\tan^2\left( \frac{\theta}{2} \right) \right)^{-1}
	\text{\equationlabels{$\epsilon$}}
\end{equation}

Now we can define a value called the reduced cross section by dividing the experimental cross section by the Mott cross section and rearranging some kinematic factors as in Equation \ref{eq:rxs}.

\begin{equation} \label{eq:rxs}
	\left(\frac{d\sigma}{d\Omega}\right)_{r} = \frac{\left( \frac{d\sigma}{d\Omega} \right)_{exp}}{\left( \frac{d\sigma}{d\Omega} \right)_{Mott}} \epsilon (1+\tau) = \left[ \epsilon G_E^2\left(Q^2\right) + \tau G_M^2\left(Q^2\right) \right]
	\text{\equationlabels{Reduced Cross Section}}
\end{equation}

\noindent If one plots the reduced cross section against $\epsilon$ we see that we are plotting the equation of a line. Figure \ref{fig:rosenbluth_sep} shows an example of a Rosenbluth separation using data from \cite{Article:Alex}. The first point used was taken at 3.304 GeV and 27.24$^{\circ}$ with a $\frac{d\sigma}{d\Omega}$ of 2.77 $\pm$ 0.39 $\times$ 10$^{-13}$ cm$^2$/sr and the second point was at taken at 0.9893 GeV and 140.31$^{\circ}$ with a $\frac{d\sigma}{d\Omega}$ of 3.27 $\pm$ 0.13 $\times$ 10$^{-15}$ cm$^2$/sr. Both points have the same $Q^2$ value, 55.1 fm$^{-2}$, as required to perform a Rosenbluth separation. 

\begin{figure}[!ht]
\begin{center}
\includegraphics[width=1.\linewidth]{Rosenbluth_Separation.png}
\end{center}
\caption[Rosenbluth Separtation]{
{\bf{Rosenbluth Separtation.}} An example Rosenbluth separation using data at $Q^2$ = 55.1 fm$^{-2}$ from \cite{Article:Alex}. From this separation we find $F_{ch}$ = 6.98 $\times$ 10$^{-5}$ and $F_m$ = 3.33 $\times$ 10$^{-5}$, where we have converted $G_E$ and $G_M$ to the $F_{ch}$ and $F_m$ form factors used in this analysis.}
\label{fig:rosenbluth_sep}
\end{figure}

Immediately we can identify the slope of the line as $G_E^2\left(Q^2\right)$ and the $y$-intercept as $\tau G_M^2\left(Q^2\right)$ \cite{Book:Povh}. Taking the data from our example separation and converting $G_E$ and $G_M$ to $F_{ch}$ and $F_m$ we find that $F_{ch}$ = 6.98 $\times$ 10$^{-5}$ and $F_m$ = 3.33 $\times$ 10$^{-5}$, where we have neglected to treat the uncertainty for simplicity. These form factor values are in agreement with the calculations made in \cite{Article:Alex}. This method works well to extract the Sach's form factors if there are enough data points taken and they have sufficiently small error bars. Unfortunately, this method is very time consuming from an experimental standpoint. In Section \ref{sec:sog} we will examine a different method of extracting form factors by fitting experimental cross section data using a sum of Gaussians technique. 

\section{Motivation and Mirror Nuclei}
\label{sec:mirror_nuclei}

Why are we interested in finding the form factors of $^3$H and $^3$He? Knowing these form factors allows us to calculate numerous useful quantities including the cross section at different energies, the charge radii, and the charge densities. The form factors also teach us about the three-body physics inside of the two nuclei, and thus they provide information on the total wave function. This information can then be compared with theoretical predictions, e.g. comparing the fitted form factors to theory predictions as in Sections \ref{ssec:3he_comparison_with_theory} and \ref{ssec:3h_comparison_with_theory}, and then be used to improve the underlying theoretical models. If one wishes to study a `free' neutron one often uses $^3$He as a proxy system. These theoretical models are vital for correcting for three-body effects inside the nucleus so that a `free' neutron can be studied, and all of these corrections are dependent on a firm understanding of the $^3$He form factors.

This analysis has chosen to analyze both $^3$H and $^3$He as they are mirror nuclei. This means that each nuclei has the same number of nucleons, but the number of protons and neutrons in each nucleus is flipped (i.e. $^3$H has one proton and two neutrons and $^3$He has two protons and one neutron). The differences in the form factors between these mirror nuclei then teach us about the differences of the protons and the neutrons in the system by simply replacing one with the other. So if the two nuclei are in the ground state then each has three nucleons in the 1S$_{1/2}$ shells. $^3$H should have one proton in the 1S$_{1/2}$ shell and two neutrons in the 1S$_{1/2}$ shell for a full shell for the neutrons (so the angular momentum of $^3$H is derived from the unpaired proton). $^3$He should have two protons in the 1S$_{1/2}$ shell for a full shell and one neutron in the 1S$_{1/2}$ shell (so the angular momentum of $^3$He is derived from the unpaired neutron). Changing between these nuclei does not change the ground state nuclear wave function but merely fills a different shell. We believe that the strong force treats both protons and neutrons identically so any differences in the form factors for these mirror nuclei will be due to the different Coulomb forces in the nuclei from the electric charges of the protons and neutrons. 

Let us predict which nucleus, $^3$H or $^3$He, has a larger charge radius and then study how that influences the form factors. To do this we will employ the liquid drop model of the nucleus. In this model the binding energy for a nucleus is given by Equation \ref{eq:binding_energy} \cite{liquid_drop}. The terms from left to right are the volume, surface, Coulomb, asymmetry, and pairing terms and are derived from how the nucleon's configurations and ratios contribute to the binding energy. In Equation \ref{eq:binding_energy} $a_{1-4}$ are empirically derived constants, $A$ is the total number of nucleons, $Z$ is the number of protons, and $N$ is the number of neutrons.

\begin{equation} \label{eq:binding_energy}
	B(A,Z) = a_1A - a_2A^{2/3} - a_3 \frac{Z(Z-1)}{A^{1/3}} -a_4\frac{\left( N-Z \right)^2}{A} +\delta
	\text{\equationlabels{Binding Energy of a Nucleus}}
\end{equation}

Let us now apply this binding energy formula to our mirror nuclei. We can easily see that the volume and surface terms are equal for $^3$H and $^3$He since both nuclei have $A$ = 3. The asymmetry terms are also equal due to squaring the numerator. The $\delta$, pairing, term is zero for nuclei with odd numbers of nucleons. This leaves only the Coulomb term, $a_3 \frac{Z(Z-1)}{A^{1/3}}$, to differentiate the binding energies as we anticipated since the strong force does not distinguish between protons and neutrons. For $^3$H this term equals zero and for $^3$He this term equals $a_3 \frac{2}{A^{1/3}}$ with $a_3$ experimentally determined to equal 0.645 MeV \cite{liquid_drop}. This makes the binding energy of $^3$He 0.894 MeV lower than that of $^3$H (the total binding energies of both nuclei are on the order of 8 MeV for scale). So we find that the binding energies of these mirror nuclei are similar, but the nucleons in $^3$He are less tightly bound. 

Let us examine if this prediction agrees with our expectations. $^3$He contains two protons and a neutron, whereas $^3$H contains one proton and two neutrons. The nuclei of these atoms are held together by the strong force while the Coulomb force pushes against this binding force. $^3$He has more charge due to having two protons, and thus the Coulomb force for $^3$He will be stronger than that of $^3$H. Therefore, we expect $^3$He to have a larger charge radius than $^3$H as there is a larger Coulomb force pushing the charge radius out against the strong force.

%This makes good sense as the two protons in $^3$He create more Coulomb force pushing out against the binding strong force than the single proton in $^3$H. This excess Coulomb force pushes the charge radius of $^3$He outward and should make it larger than $^3$H's charge radius.

As \cite{Book:Povh} points out the location of the form factor's diffractive minima can be used to learn about the size of the charge radius (see Equation \ref{eq:minima}). Let us use this relation to compare the locations of the first minima in $F_{ch}$ for $^3$He (Figure \ref{fig:3he_fch_rep_fit}) and $^3$H (Figure \ref{fig:3h_fch_rep_fit}) ($F_{ch}$ is discussed in more detail in Section \ref{sec:ffs}, but it represents the Fourier transform of the electric charge density). We see that the first minimum for $^3$He is found at $Q^2$ $\approx$ 11 fm$^{-2}$, and the first minimum for $^3$H is found at $Q^2$ $\approx$ 13 fm$^{-2}$. Equation \ref{eq:minima} then predicts that $^3$He has a slightly larger charge radius than $^3$H exactly as we predicted from the liquid drop model. 

 

