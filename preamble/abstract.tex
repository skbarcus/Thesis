\documentclass[../main.tex]{subfiles}

%%%%
% Note this page should not exceed 3500 characters(including spaces), limited to single page!!!
% Must be 12 pt font
% No indentations between paragraphs
%%%%

% https://charactercounttool.com/ - copy text here and get character count

\begin{document}
\newpage
\thispagestyle{empty}
%This adds a bookmark in the PDF sidebar only
\phantomsection\pdfbookmark{Abstract}{Abstract}

\begin{center}
\fontsize{14}{16.8} \selectfont{ABSTRACT}
\end{center}

\begin{singlespace}
\fontsize{12}{14} \selectfont{
%\begin{flushleft}

{\parindent0pt % disables indentation for all the text between { and }

\indent By mining data from Jefferson Lab Hall A experiment E08-014 a new $^3$He elastic cross section was extracted from a large quasielastic background. This measurement was taken with an initial beam energy of 3.356 GeV and an angle of 20.51$^{\circ}$. The cross section was found to be 1.345 $\times$ 10$^{-6}$ $\mu$b/sr $\pm$ 0.086 $\times$ 10$^{-6}$ $\mu$b/sr at $Q^2$ = 34.19 fm$^{-2}$. This new data point falls approximately halfway between the first and second diffractive minima of the $^3$He form factors. When combined with recent high $Q^2$ $^3$He elastic cross section measurements from Jefferson Lab and MIT-Bates this new data point improves our knowledge of the cross section and form factors at large momentum transfers. \\

}
%\end{flushleft}

%\begin{flushleft}

{\parindent0pt % disables indentation for all the text between { and }

\indent The new high $Q^2$ data motivate a reanalysis of the $^3$He elastic cross section world data and provide an improved understanding of the magnetic form factor in particular. For this analysis the elastic cross section world data for $^3$He, and its mirror nuclei $^3$H, were collected. The world data spans a time frame from 1965 to 2016. The dataset contains electron energy ranges from tens of MeV to above 12 GeV for measurements performed at many different facilities. The world data were then fit using a sum of Gaussians parametrization which allowed for the extraction of both targets' magnetic and electric form factors which were then used to calculate charge densities and radii. %These new fit results were contrasted with past fit results and compared to modern theoretical predictions. 
\\

}
%\end{flushleft}

{\parindent0pt % disables indentation for all the text between { and }

\indent The new charge and magnetic form factors for $^3$H and the charge form factor for $^3$He are in good agreement with the 1994 fits from Amroun \textit{et al}. However, the addition of the new high $Q^2$ data has caused the $^3$He magnetic form factor's first diffractive minimum to shift up in $Q^2$ by 1-3 fm$^{-2}$ while also decreasing the magnitude of the magnetic form factor above $Q^2$ $\approx$ 20 fm$^{-2}$. The first diffractive minima for $^3$H are located at $Q^2$ $\approx$ 13 fm$^{-2}$ and $Q^2$ $\approx$ 23-24 fm$^{-2}$ for the charge and magnetic form factors respectively. The first diffractive minima for $^3$He are located at $Q^2$ $\approx$ 11 fm$^{-2}$ and $Q^2$ $\approx$ 17-20 fm$^{-2}$ for the charge and magnetic form factors respectively. \\

}

{\parindent0pt % disables indentation for all the text between { and }

\indent The charge radius for $^3$He was found to be 1.90 fm $\pm$ 0.00144 fm in reasonable agreement with past measurements, and the charge radius for $^3$H was found to be 2.02 fm $\pm$ 0.0133 fm which is much larger than past measurements. However, each of these charge radii has an additional uncertainty that must be applied to them due to allowing all parameters to float freely during the sum of Gaussians fitting procedure. This additional uncertainty should be small for $^3$He, but it is likely quite significant for $^3$H and would help bring this charge radius closer to agreement with past measurements that made different fitting choices. Unfortunately, this analysis was unable to quantify this additional uncertainty. \\

}

{\parindent0pt % disables indentation for all the text between { and }

\indent The new form factor fits were compared to modern theoretical predictions from the 2016 paper of Marcucci \textit{et al}. The `conventional' theoretical approach applied in this paper modelled two and three-body nucleon interactions with relativistic corrections and was reasonably successful at predicting the charge form factors of $^3$H and $^3$He. $\chi$EFT predictions were also often successful. However, while the `conventional' approach still performed best, theory failed to accurately predict the magnetic form factors of either $^3$H or $^3$He. The first diffractive minimum of the new $^3$He magnetic form factor fits actually moved further away from theory. This disagreement between theory and experiment provides motivation for new asymmetry measurements using polarized $^3$He and a polarized electron beam. When the beam is scanned in $Q^2$ on the target the sign of the asymmetry will flip at the form factor minima pinning down their true location.

}

}
\end{singlespace}

\end{document}