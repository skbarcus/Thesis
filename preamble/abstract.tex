\documentclass[../main.tex]{subfiles}

%%%%
% Note this page should not exceed 3500 characters(including spaces), limited to single page!!!
% Must be 12 pt font
% No indentations between paragraphs
%%%%

% https://charactercounttool.com/ - copy text here and get character count

\begin{document}
\newpage
\thispagestyle{empty}
%This adds a bookmark in the PDF sidebar only
\phantomsection\pdfbookmark{Abstract}{Abstract}

\begin{center}
\fontsize{14}{16.8} \selectfont{ABSTRACT}
\end{center}

\begin{singlespace}
\fontsize{12}{14} \selectfont{
\begin{flushleft}

{\parindent0pt % disables indentation for all the text between { and }

By mining data from Jefferson Lab Hall A experiment E08-014 a new measurement of the $^3$He elastic cross section at $Q^2$ $\approx$ 34 fm$^{-2}$ was extracted from a large quasielastic background. This new data point falls approximately halfway between the first and second diffractive minima of the $^3$He form factors. When combined with recent high $Q^2$ $^3$He elastic cross section measurements from Jefferson Lab this new point improves our knowledge of the cross section and form factors at large momentum transfers. 

}
\end{flushleft}

\begin{flushleft}

{\parindent0pt % disables indentation for all the text between { and }

The new high $Q^2$ data motivate a reanalysis of the $^3$He elastic cross section world data and provide an improved understanding of the magnetic form factor in particular. For this analysis the elastic cross section world data for $^3$He, and its mirror nuclei $^3$H, were collected. The world data spans a time frame from 1965 to 2016. The dataset contains electron energy ranges from tens of MeV to above 12 GeV for measurements performed at many different facilities. The world data were then fit using a sum of Gaussians parametrization which allowed for the extraction of both targets' magnetic and electric form factors, along with charge densities and radii. These new fit results were contrasted with past fit results and compared to modern theoretical predictions.

}
\end{flushleft}

}
\end{singlespace}

\end{document}