%Use geometry package to set margins
%WM requirement: Left 1.5in to 2in Top,Right,Bottom 1in to 1.25in
\usepackage[letterpaper,
            left=1.75in,
            top=1.125in,
            right=1.125in,
            bottom=1.125in]{geometry}
            
%Allow input of separate files. Using to load configuration files.
\usepackage{import}

%Allow separate files to be included into the main document.
\usepackage{subfiles}

%Allow fancy headers and footer
\usepackage{fancyhdr}
%\fancyhead{}

%\fancyfoot{}

%Allow doublespacing as required
\usepackage{setspace}
\doublespacing

%Allows for changing of fonts size for parts of doc
%\fontsize{size}{lineheight}\selectfont{blah blah}
%lineheight = 1.2*size
\usepackage{anyfontsize}

%Allow preamble tables to be formatted
\usepackage{tocloft}
%Dots in ToC
%\renewcommand{\cftchapleader}{\cftdotfill{\cftdotsep}}

%Import graphics package
\usepackage{graphicx}
%Allow for rotating boxes. (Lets me rotate my world data table.)
\usepackage[graphicx]{realboxes}
\usepackage{booktabs,rotating}


%Fix spacing issues in text
\usepackage{microtype}

%Import extra math environments
\usepackage{amsmath}

%Allows click-able reference in final pdf document.
%Also allows bookmarks for final pdf document.
%\usepackage[hidelinks]{hyperref}
\usepackage{hyperref}
\hypersetup{
    %colorlinks=true,
    colorlinks=false,
    linkcolor=blue,
    filecolor=magenta,      
    urlcolor=cyan,
    %bookmarks=true,
    bookmarksopen=false,
    pdftitle={Dissertation},
    pdfauthor={H. Potter}
}

%Allows for more clever referencing of items using \cref and \Cref
\usepackage{cleveref}

%Allow you to place To-do notes in colorful block in text.
 \usepackage[obeyFinal,
             colorinlistoftodos,
             textwidth=textwidth]{todonotes}
             
%Allow captions for figures
\usepackage{caption}
%Allow captions for subfigures
\usepackage{subcaption}

%Allow appendices
\usepackage[toc]{appendix}

%Allow Feynman Diagrams
%Activate later! Requires .sty file.
%\usepackage{tikz-feynman}

%Allow chemical & isotope notation
\usepackage[version=3]{mhchem}
\usepackage{isotope}

%Allow fancy tables 
\usepackage{booktabs}

%Always indent the first paragraph of a section or chapter.
\usepackage{indentfirst}

%Allows for the making of cells in tables.
\usepackage{makecell}

%Allows for the use of landscape oriented pages.
\usepackage{longtable}

%Allows for the use of long tables.
\usepackage{multirow,bigstrut}
\usepackage{pdflscape} % for 'landscape' environment. Use \usepackage{lscape} for nonrotated pages in the PDF and \usepackage{pdflscape} for rotated pages in the PDF.